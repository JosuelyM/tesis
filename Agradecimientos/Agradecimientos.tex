%\chapter*{}
%\pagenumbering{Roman}

\begin{acknowledgementsacademic}

Gracias a la UNAM por la formación académica recibida, y por el ejemplo profesional que me brindó en su personal, en la Preparatoria 9 con los profesores Norma Ramírez, Patricia Huerta, Elsa Cano, Arturo Palafox, José Carbajal y Arturo Jiménez; y en la Facultad de Ciencias con  Rosario Paredes, Patricia Goldstein, Pablo Barrera, Mirna Villavicencio, Valentín Porta, Ricardo Méndez, Roxana del Castillo y Catalina Stern. Gracias a todos por su labor, su ética y moral profesional,  y por todo el apoyo brindado. 

\vspace{0.5cm}
Gracias a todos los miembros del  grupo de idiomas, por formarme y enseñarme que se puede hacer física en las cosas más cotidianas, no sólo en los fenómenos abstractos. 

\vspace{0.5cm}
Gracias a mi asesor el Dr. Carlos Pineda, por sus consejos, su compresión, su apoyo y paciencia en las múltiples reuniones para elaborar este trabajo. También porque me enseño que en la ciencia,  lo más importante no son las formulaciones y los desarrollos complicados, sino la explicación y la sencillez con la que nos expresamos. 

\vspace{0.5cm}
Finalmente, gracias a los apoyos de los proyectos PAPIIT IG100518 y CONACYT CB-285754, ya que con ellos fue posible cada parte del desarrollo de está tesis. 

\end{acknowledgementsacademic}




