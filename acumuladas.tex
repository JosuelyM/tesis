\chapter{Palabras acumuladas}
% Intro {{{

La búsqueda para cuantificar la influencia, ha llevado a contabilizar las
palabras que son nuevas en los distintos receptores y a partir de ellas ligar contextos que sustenten la aparición de palabras.  Se ha puesto énfasis en el conjunto de búsqueda, pero  el conjunto base  también tiene información sobre las palabras que han migrado, además  abarca más años (159 comprendidos entre 1740 y 1899), por lo que su información es más basta en contenido. 
  

$\rightarrow$ 

\fxwarning{Rescribi  de forma mas compacta casi todo el contenido de esta pagina, tratando de ser mas claro respecto a las notas que me dejaste, deje la felcha para indicar y no borrar lo que habia escrito y donde me dejaste notas. }

Para no repetir el proceso de contabilizar a los préstamos nuevos,  se propone pensar que en el primer año del conjunto de búsqueda (1900), el idioma receptor ya contenia  cierta cantidad de palabras que provenian de otros origenes; 
\cpnote{A que te refieres con orígenes?}
de tal manera que  ya forman parte de él, es decir estos préstamos ``conviven'' con las palabras propias de el receptor y son empleadas
indistintamente\cpnote{No entiendo lo que quieres decir en esta frase. 
Sugiero que la redactes mejor o me la expliques en persona.}.


Así el conjunto base proporcionará un sostén de aquellas palabras que han permeado en un idioma
y son utilizadas en los años del conjunto de búsqueda,  cabe decir que este
sostén crecerá conforme se localicen nuevas palabras\cpnote{acá tampoco entiendo}. 

Es necesario hacer una nueva definición para estos préstamos, dados un idioma  origen  \textit{A} y la lista para un año  de las palabras más comunes en el receptor \textit{B}, se definen como: 

\begin{description}
	\item[préstamos acumulados:] Son las palabras con origen \textit{A} que ya habían aparecido en alguna lista de \textit{B}, y para ese año lo volvieron a hacer.  \cpnote{Es decir que volvieron a aparecer, o que siguen estando? no me queda claro. En la frase despues, lo aclaras, pero siento que la definicion está deficiente. Igual porfa no quites la siguiente frase.}
\end{description}
\cpnote{No entiendo porque cambias el formato. Sugiero seguir con el parrafo como va. De hecho veo que en el capitulo 3, al principio tienes un formato similar. Cambia ambos e integralos en el texto. En la seccion 2.2 me gusta como tienes esas primeras 4 definiciones. Las
primeras dos, integradas en una frase, y las siguientes dos, como parte de una frase dentro de items. }

$\rightarrow$

\fxwarning{aqui empieza el texto corregido}

Se propone pensar en un origen \textit{A} y un receptor \textit{B}, y se continua a encontrar los  préstamos nuevos de \textit{A} en \textit{B} que surgieron en los años  comprendidos en el conjunto base (1740-1899).  Estos préstamos serán en principio el nuevo cimiento de la investigación,
se pueden definir como \textbf{préstamos acumulados} a aquellas palabras  con origen \textit{A} que habían aparecido anteriormente  en  \textit{B} y que para un determinado año vuelven a aparecer, por lo que continúan siendo relevantes en \textit{B}



La diferencia entre los nuevos y los acumulados es que un préstamo será nuevo sólo en el año de aparición, posteriormente se convertirá en acumulado, además la cantidad de acumulados ira creciendo conforme los nuevos vuelvan a aparecer. 

El objetivo  de trabajar con los acumulados es ver cómo se comportan las palabras que ya han migrado a un receptor y si hay periodos donde su empleo se vea alterado. 


\cpnote{Que quieres decir con esa ultima frase? Que comportamiento? No se
que podemos decir con los datos que tenemos}.  
\fxwarning{Para ver el comportamiento,  trate de establecer cantidades con las cuales relacionar el comportamiento,  si el comportamiento es alterado esta cifra aumento o disminiyuo.  Movere esta frase despues de que defina las cantidades. solo como ejemplo esta cifra entre el ingles y el aleman era baja antes de 1945, postriroemnte fue aumentando debido a que el aleman comenzo a utilizar mas  las palabras del ingles}

En el capitulo anterior, la atención se enfocaba en la cantidad de palabras, nunca se trato con su frecuencia o su rango,  por lo que en esta sección se destacarán estas propiedades para llegar a una cantidad que cuantifique la influencia. 

Tomando la lista de las cinco mil palabras más usadas de \textit{B}, y se
distinguen en ella los préstamos acumulados con origen \textit{A}, el curso a seguir es el siguiente. 

\cpnote{Oye, no me queda claro que estas usando el doble punto de manera
correcta. Puedes porfa verificar en un manual de ortografía y gramática que lo estas haciendo bien?}

\fxwarning{No lo estaba haciendo bien,  para enumerar o detallar un proceso no se usan los dos puntos,    lo comprobe en una pagina de la UAM, corrijo en donde haya hecho lo mismo http://www.uamenlinea.uam.mx/materiales/lengua/puntuacion/html/puntos.htm 
}


\begin{enumerate}
	
	\item En un año determinado $t$ del idioma \textit{B}, se sumarán las
	frecuencias $f(k)$ de las cinco mil palabras más usadas, donde $k$ determina el rango  de cada palabra. Esta nueva cantidad se llamará \textbf{frecuencia total}  $\underset{\text{\tiny B}}{F}(t)$ , siendo su expresión la siguiente

	\begin{equation}
	\label{ec.ftot}
	%F^{y}_{\text{B}} = \sum_{k=1}^{5000} f(k)
	\underset{\text{\tiny B}}{F}(t) = \sum_{k=1}^{5000} f(k)
	\end{equation}
	
	\cpnote{el texto en una ecuaccion se pone de otra manera. Además, no
	tienes necesitad de ponerlo ahi, mejor en lo que sigue. Ademas, tu
	notacion esta chafa. Si tenemos $f_5$ es a tiempo 5 o para el rango 5?
	Dale una iterada a todo este capitulo. Veo que 
	no esta tan pulida como otros capitulos.} 
	\fxwarning{Listo, corregida y el texto lo puse dentro del punto y ya no en la ecuacion}


	\item Si en la lista de las palabras más usadas de \textit{B} se distinguen los préstamos acumulados con origen \textit{A} que ocupan un rango $j$,  se procede a sumar la frecuencia asociada a estas palabras $f(j)$. Esta cantidad será la  \textbf{frecuencia de préstamo} $\underset{ \text{\tiny A} \to  \text{\tiny B} }{P}(t)$   de \textit{A} en \textit{B} para el año $t$.
	
	\begin{equation}
	\label{ec.fpres}
	%P^{y}_{\text{A} \to \text{B}} = \sum_{j} f(j)
	\underset{ \text{\tiny A} \to  \text{\tiny B} }{P}(t) = \sum_{j} f(j)
	\end{equation}
	
	
	\item   Se divide la frecuencia de préstamo entre la frecuencia total , obteniendo una cantidad que se detonará como  \textbf{uso} $\underset{ \text{\tiny A} \to  \text{\tiny B} }{U}(t)$  siendo la porción que representa \textit{A} en \textit{B} en términos de frecuencia.  Como en un año hay más palabras propias de B, esta cantidad es muy pequeña, para tener cifras manejables, se tomará un porcentaje al multiplicar el cociente por cien.  

	\begin{equation}
	\label{ec.fuso}
	 \underset{ \text{\tiny A} \to  \text{\tiny B} }{U}(t) = \frac{	\underset{ \text{\tiny A} \to  \text{\tiny B} }{P}(t)}{\underset{\text{\tiny B}}{F}(t) } * 100
	\end{equation}
	
	\fxwarning{deje los sub indices para saber que opinabas,  a mi gusto se ve saturado y  creo que se sobrentiende el tener el subindice sobre la U,  P y F tambien lo deberian de tener }
	
	
	\item Se empleó la ecuación \ref{ec.fuso} en todos los años del conjunto de búsqueda, obteniendo 109 valores.
	
	\item El proceso se repitió para todas las combinaciones de orígenes y receptores.

\end{enumerate}

Al igual que en los préstamos nuevos,  se agruparon los préstamos acumulados por cada pareja de origen y receptor,  y dentro de ella se listaron por cada año del conjunto de búsqueda los prestamos encontrados en el receptor. Estos datos serán de utilidad al interpretar los datos, y en los capítulos siguientes,  asimismo se proporciona en \cite{prestamos_acumulados} la tablas con las agrupaciones hechas, mientras que en el Apéndice 1 se especifica la forma de leerlas. 

El objetivo  de trabajar con los acumulados es ver cómo se comportan las palabras que ya han migrado a un receptor y si hay periodos donde su empleó (a través del uso) se vea alterado. Para aclarar esta ultima idea, dada una lista del receptor \textit{B} para un año $y$, se encuentran en ella préstamos acumulados con diferentes orígenes \textit{A}, \textit{C} o \textit{D},  existiendo un valor de uso por cada origen. Ya que esta cantidad depende del tiempo (en este caso los  años $y$),  se podrá observar el comportamiento de los idiomas presentes en  \textit{B} en diferentes épocas y determinar por cada periodo cual fue el idioma más influyente. 





\section {El uso entre idiomas} 

Por cada idioma se presentarán dos graficas, la primera tomando al idioma como origen y observar el uso que tiene en los demás, la segunda mostrará el caso contrario al tomarlo ahora como receptor y graficando el uso de los demás en el.  Como la influencia esta ligada al significado de las palabras, se provee información de cuales palabras son relevantes para que el uso varié en el tiempo así como el evento, área o campo semántico al cual se lograron relacionar. 


Una tercera grafica es posible, al fijar una pareja de origen y receptor, manifestando el uso de uno sobre otro,  estas graficas se agregaran a la seccion XXXXX del apendice 1, sirviendo para complementar los resultados expuestos en esta sección.


Adicionalmente, la tabla \ref{tab.cantidad_acumulados} muestra la cantidad promedio de préstamos acumulados encontrados en cada año del conjunto de búsqueda. La idea de la tabla y del valor del uso, es observar que el idioma que más palabras aporta a un receptor no es siempre el más utilizado,  el uso es mayor si  si los préstamos tienen rangos mas bajos(frecuencias altas), sin importar la cantidad de préstamos que se esten tratando. 


\begin{table}
	\centering
	\begin{tabular}{lcccccc}
		\multicolumn{7}{c}{R E C E P T O R}                                                                                                                                             \\
		\multirow{6}{*}{\begin{tabular}[c]{@{}l@{}}O\\ R\\ \,I\\ G\\ E\\ N\end{tabular}} &             & \textbf{EN} & \textbf{FR} & \textbf{GE} & \textbf{IT} & \textbf{SP} \\
		& \textbf{EN} & -           & 324.43      & 164.33      & 77.5        & 73.61       \\
		& \textbf{FR} & 297.36      & -           & 94.06       & 118.55      & 66.31       \\
		& \textbf{GE} & 63.87       & 48.06       & -           & 34.92       & 16.61       \\
		& \textbf{IT} & 77.82       & 100.62      & 47.9        & -           & 219.45      \\
		& \textbf{SP} & 118.43      & 84.22       & 29.85       & 311.97      & -          
	\end{tabular}
	\caption{Promedio de préstamos acumulados entre idiomas. Se aprecian dos relaciones reciprocas entre el inglés con el francés y el español con el italiano, donde no importa cual actué como receptor, el otro idioma es el origen del que provienen la mayor cantidad de palabras.}
	\label{tab.cantidad_acumulados}
\end{table}




\subsection{Inglés} % {{{


\begin{figure}[h!]
	\centering
	\includegraphics[scale=.36]{PF1_S2_EN.png}
	\label{fig.ST_a_EN}
	\caption{El inglés en los demás. El francés es el idioma donde más se ha empleado inglés, sin embargo en el español ha sido el de mayor crecimiento comparado en su uso en principios y en final de siglo.}
\end{figure} 



\begin{figure}[h!]
	\centering
	\includegraphics[scale=.36]{PF2_S2_EN.png}
	\label{fig.ST_b_EN}
	\caption{Los demás en el inglés. En los últimos 50 años, el español ha sido el idioma que es más utilizado en el ingles, seguido del francés, debido a las relaciones comerciales entre países de ambas lenguas.  Tras la segunda guerra mundial, alemán e italiano decayeron consistente en ser lengua de los países vencidos. }
\end{figure} 



El uso del ingles en los demás ha visto un continuo incremento posterior a 1945 en francés, en italiano y en español, mientras que en  alemán se da posterior a 1990, año donde culmina la guerra fría y se da la finalización del socialismo en Europa con la re-unificación de Alemania. El significado común de los   préstamos acumulados que aparecen en los cuatro conjuntos y que con los años descienden en rango son términos económicos y referentes a la industria como  \textit{capital}, \textit{dollar}, \textit{invesment}, \textit{relations}, \textit{market}, \textit{company}, \textit{development}, \textit{financial},  \textit{institutions}, \textit{internet} y \textit{software}. Otra característica relevante es la aparición continua de los apellidos de los presidentes de los Estados Unidos (posteriores a la guerra) durante el periodo en el cual gobernaron.  Apoyado de la información de los préstamos nuevos, se puede confirmar que el inglés se ha beneficiado del crecimiento de los Estados Unidos para ser exportado a las demás lenguas y ser el idioma común para transmitir información.   


En los últimos cincuenta años, los idiomas mas comunes en el ingles han sido el español y el francés,  nombres de países latinoamericanos como \textit{México}, \textit{Cuba}, \textit{Chile}, \textit{Nicaragua} y \textit{Argentina}, caracterizan a los acumulados del español, mientras que en el francés en su mayoría son palabras que bien podrian catalogarse de origen inglés, entre ellas \textit{royals}, \textit{religion}, \textit{saint}, \textit{passage} o \textit{court}. Tras brevemente ver ambos conjuntos se infiere que el español ha logrado instaurarse en el inglés por la relevancia de estos países en las relaciones o conflictos que tuvieron en el siglo pasado y donde intervinieron países de habla inglesa, contrario  al francés que prevalece por las relaciones culturales y etimológicas que existen entre ambas lenguas.

Por parte de las palabras con origen alemán e italiano no se logró relacionarlas a un campo semántico común. 


% }}}
\subsection{Francés} % {{{

\begin{figure}[h!]
	\centering
	\includegraphics[scale=.36]{PF1_S2_FR.png}
	\label{fig.ST_a_FR}
	\caption{El francés en los demás idiomas. El italiano empleó más al francés durante todo el siglo XX, caracterizado por palabras comunes en la industria vitivinícola.}
\end{figure}


\begin{figure}[h!]
	\centering
	\includegraphics[scale=.36]{PF2_S2_FR.png}
	\label{fig.ST_b_FR}
	\caption{Los demás en el francés. El español ha resultado el de mayor presencia en el francés, en su mayoria son palabras con etimologías grecolatinas, comunes para ambas lenguas al provenir de la misma familia lingüística.}
\end{figure}
		


A pesar de que el idioma que más préstamos toma del francés es el inglés,  el idioma que más utiliza el francés ha sido el italiano,  aspecto que se mantuvo durante todo el siglo del análisis,  la industria vitivinícola, surgen como un conectores entre ambas lenguas al estar presente términos como  \textit{raisins}, \textit{vin}, \textit{vignoble} y \textit{recolte},  siendo una actividad común entre Francia e Italia.  Los préstamos hacia los demás idiomas son de carácter religioso o politico, destacando que tuvo el francés en estos ámbitos, a pesar de que la búsqueda se centre en el siglo XX, las mayores migraciones del francés surgen a partir de 1800, posteriores a la revolución francesa; entre las palabras que se han mantenido desde este acontecimiento están  \textit{saint}, \textit{eglise}, \textit{dime}, \textit{reine}, \textit{guerre}, \textit{imperiale}, \textit{royals} o \textit{bourgeois}.  


Para los prestamos usados en el francés el español y el inglés se muestran como los idiomas con mayor presencia, la característica común de los vocablos de ambos idiomas es que son palabras con etimología grecolatina, 
\textit{depression}, \textit{canal}, \textit{proceso}, \textit{services}, \textit{justice} entre otras,  siendo razonable la aparición de estas palabras por tener las tres lenguas una composición grecolatina. 


% }}}
\subsection{Alemán} % {{{

\begin{figure}[h!]
	\centering
	\includegraphics[scale=.36]{PF1_S2_GE.png}
	\label{fig.ST_a_GE}
	\caption{El alemán en los demás. La familiaridad por provenir de la misma rama lingüística hace posible que el inglés sea el idioma  donde los préstamos del alemán sean continuamente utilizados, siendo evidente la diferencia con las lenguas romances.}

\end{figure}


\begin{figure}[h!]
	\centering
	\includegraphics[scale=.36]{PF2_S2_GE.png}
	\label{fig.ST_b_GE}
	\caption{Los demás en el alemán. Cada lengua ha tenido un periodo de  crecimiento posterior a 1950, mostrando al alemán como un idioma susceptible (al menos en la literatura) donde los demás idiomas han impactado y perdurado.}
\end{figure}




La característica principal de los términos en alemán  en los demás idiomas  son principalmente personajes germano-parlantes que sobresalieron en algún ámbito; siendo además los encontrados en los préstamos nuevos, \textit{Hitler}, \textit{Marx}, \textit{Einstein}, \textit{Freud}, \textit{Engels}, \textit{Heidegger}, \textit{Mozart}, \textit{Hegel} y  \textit{Nietzsche}. 

Como idioma receptor, el alemán adoptó palabras de diferentes campos, tecnológicos y de desarrollo por parte del inglés,  religiosos  por el francés, históricos en el italiano y médicos por el español.  Cada idioma presentó un periodo de crecimiento posterior a 1950  o a la segunda guerra mundial, donde al ser vencido alemania, el idioma tuvo que adaptarse a las tendencias donde los demás destacaban. 

% }}}
\subsection{Italiano} % {{{

\begin{figure}
	\centering
	\includegraphics[scale=.36]{PF1_S2_IT.png}
	\label{fig.ST_a_IT}
	\caption{El italiano en los demás. A pesar de ser fonéticamente similares y de provenir de la familia de las lenguas romances,  el español persiste en ser el idioma con menor uso de italiano. La migración italiana a los Estados Unidos posterior a la primera guerra mundial coincide con el aumento en el inglés. }
\end{figure}
		
\begin{figure}[h!]
	\centering
	\includegraphics[scale=.36]{PF2_S2_IT.png}
	\label{fig.ST_b_IT}
	\caption{Los demás en el italiano. La proximidad geográfica  entre italia con países cuya lengua es el francés y el alemán ayudó a incrementar el uso de estos en el italiano. A pesar de que el inglés se difundió como un idioma universal para la comunicación, su uso en el italiano no ha mostrado un incremento considerable en todo el siglo XX. }
\end{figure}



El uso del italiano se vio caracterizado en la segunda mitad del siglo por la constante aparición y ascenso en rango de \textit{Mussolini}, siendo el personaje más relevante en el siglo pasado cuya lengua materna es el italiano.  Salvo por Mussolini, los demás prestamos italianos presentes en las otras lenguas no se lograron asociar a un único campo, sin embargo esto muestra la diversidad de temas en los cuales el italiano fue relevante, como términos políticos \textit{sociale}, \textit{liberale}, religiosos \textit{santo}, \textit{suora}, \textit{cattedrale} o  bélicos \textit{battaglia}, \textit{regime}.

El sentido inverso al considerar el tipo de palabras que utiliza el italiano de los demás idiomas, es igualmente variado por parte del ingles son conceptos ligados a la tecnología, por el francés a la industria vitivinícola, con el alemán a personajes relevantes de esta lengua,  términos que ya se han mencionado. Finalmente entre la variedad de palabras que toma del español se encuentran nombres de países o ciudades hispanohablantes \textit{México}, \textit{Chile}, \textit{Argentina}, \textit{Montevideo} o \textit{Peru}. 

% }}}
\subsection{Español} % {{{

\begin{figure}[h!] % {{{
	\centering
	\includegraphics[scale=.36]{PF1_S2_SP.png}
	\label{fig.ST_a_SP}
	\caption{El español en los demás. Los idiomas que más emplean español son aquellos con los que comparte una relación etimológica, francés e italiano por ser lenguas romances y con el ingles al tener este idioma una base de palabras grecolatinas.     }
\end{figure}


		
\begin{figure}[h!] % {{{
	\centering
	\includegraphics[scale=.36]{PF2_S2_SP.png}
	\label{fig.ST_b_SP}
	\caption{Los demás en el español.  Durante el último siglo, el contenido francés e inglés ha aumentado en el español siendo su uso equiparable, el surgimiento del inglés como el idioma universal y la relación etimológica con el francés  hacen posible los incrementos.}
\end{figure}



Entre los gráficos del apéndice 1, referentes al uso entre el español y un determinado idioma, se observo que el uso de los del español en los demás idiomas es mayor que el uso de los otros en él.  Ya se han comentado las características principales de las palabras interfieren en el uso, destacando el ámbito de la medicina, \textit{terapia}, \textit{lepra}, \textit{tumor}, \textit{syphilis}, \textit{virus} o \textit{renal}. Con ellos se infiere la productividad y la importancia de la medicina en países de lengua española antes de 1900; incluso en lenguas como el alemán se encontraron este tipo de préstamos. 


En las discusiones anteriores, se ha mencionado el tipo de palabras que los idiomas aportan,  en el caso de las contribuciones al español,   estas siguen la misma tendencia; del ingles los préstamos son de carácter tecnológico y del desarrollo industrial, del alemán son apellidos de personajes destacados en un campo especifico, mientras que del francés y el italiano son de condición religiosa. 

% }}}
% }}}
\section{Comentarios y complementos del método} % {{{


El determinar la influencia entre idiomas a través del uso de los préstamos, ha mostrado primeramente que el idioma que mas cantidad de palabras tiene en otro no siempre es el más utilizado,  radicando el mayor uso en aquel idioma cuyas préstamos tengan menores rangos en la lista de un receptor. 

En todo el siglo XX y la primer década del XXI, el inglés y el alemán han sido los idiomas más cambiantes en los papeles de origen y receptor respectivamente.
El inglés al ser el que más creció en tres idiomas (francés, alemán y español), complementando los resultados del capitulo anterior, al ser el idioma que más palabras nuevas exportó.  El alemán como el receptor donde los diferentes orígenes aumentaron su usó tras la segunda guerra mundial; el uso ha sido semejante a los préstamos nuevos, ha sido el receptor que más recibió. 

Ambos análisis se complementan,  el idioma más influyente ha aportado más palabras nuevas y aquellas que se van acumulando resultan las de mayor incremento en el uso. El idioma más influenciado recibió la mayor cantidad de palabras nuevas y el uso que han tenido los demás ha sido también el del mayor incremento. 

Por el momento sólo es posible describir que originó las variaciones en el uso o en la cantidad de nuevas palabras, no es posible predecir como se comportaran los idiomas en el futuro, ya que la principal característica que  hace fluir a las palabras entre idiomas han sido los eventos, reflejado en que las palabras de su campo semántico  se muevan a diferentes idiomas y continúen apareciendo o desapareciendo tras el suceso. 

Una mejor información de como los eventos alteran a los idiomas se podría extraer si se compararán las características de los prestamos con  datos de los países de alguna habla como lo pueden ser  el crecimiento economizo, el producto interno bruto, la alfabetización, la mortalidad, las migraciones de personas, entre otros.





% }}}

