\chapter{Conclusiones}

En el presente trabajo, se ha tratado de estimar una forma para cuantificar la influencia que un idioma ejerce sobre otro,  a partir de la construcción de dos métodos. 

En el primero método al contar las palabras nuevas, se expuso que las palabras que migran de un idioma a otro son parte de un mismo campo semántico, y las migraciones ocurren tras un suceso que tiene relación con el campo semántico. Con ello, se refleja la influencia que tiene un idioma en determinadas áreas.  

El segundo método para cuantificar la influencia, por medio del uso de un idioma en otro, mostró que las palabras migrantes que continuamente son empleadas por los demás idiomas, también son parte de un mismo campo semántico. Además, estas descienden su rango (aumentan su frecuencia), en los años posteriores al evento que las hizo migrar. 

Con ambos métodos, se concluyó que las áreas donde un idioma es más influyente, y cuyas palabras son continuamente empleadas, son en el inglés la guerra, la economía, la tecnología, la política y la globalización; en el francés la guerra, la Revolución Francesa, la religión y la industria vitivinícola; en el alemán la guerra y los apellidos de personajes germano parlantes que destacaron el algún área académica; en el italiano  la guerra, la política y la religión; y en el español la medicina y la cultura de los países Latinoamericanos. 

\jmnote{hablar de la forma logarítmica de la diversidad}

En la parte estadística del trabajo, se destaca que las palabras migrantes de cada pareja de idiomas, siguen una regla común si estas se ordenan de acuerdo a su frecuencia de aparición, y es que en cada año, la cantidad de distintas palabras que pueden ocupar un lugar en el ordenamiento, aumentará de forma logarítmica conforme se avancen lugares en el ordenamiento. 

En el ultimo capitulo del trabajo, se trataron de justificar los resultados, pese al escaso rigor lingüístico que se tuvo. En esta justificación, se mostró que a pesar que una palabra (o un conjunto de ellas) no pertenezca con exactitud a un idioma,  el uso de un idioma en otro no se ve alterado si está es excluida del conjunto. La eliminación de palabras, ejemplifica que mientras se considere a las palabras como un conjunto,  estas mantendrán sus características, sin importar cuantos elementos conformen el conjunto. 





