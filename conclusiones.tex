\chapter{Conclusiones}

En el presente trabajo, se ha tratado de estimar una forma de cuantificar la influencia que un idioma ejerce sobre otro, proceso que no resultó sencillo al no existir un antecedente en el cual basarse o tomarse como referencia, además las clasificaciones hechas no contaron con el suficiente rigor lingüístico, no obstante, la cuantificación llegó a través de dos métodos, con resultados importantes en ambos. 

El primero al contar las palabras nuevas que surgían en los idiomas, expuso que un grupo de palabras migra de un idioma a otro si existió un suceso en cual el grupo esta involucrado, además los propios eventos, modifican la tendencia de los idiomas, cambiando su papel de ser más portadores (receptores) a ser mayormente receptores (portadores) de palabras nuevas, .  

El segundo método, por medio del valor llamado uso, permitió ver que los eventos tienen relevancia después de suscitarse, interfiriendo además la propia cultura de los países, por ejemplo las palabras que surgieron tras la revolución francesa aun siguen siendo utilizadas, mientras que los mayores aportes del alemán son referentes a personajes destacados en algún área profesional, ejemplificando el impacto de la academia germano parlante en las demás culturas.  

A pesar del escaso rigor lingüístico que se tuvo en el trabajo, en los últimos capítulos se mostró que a pesar de que una palabra (o un conjunto de ellas) no pertenezca con exactitud a un idioma, las propiedades estadísticas (matemáticas) como el uso y la diversidad de rango no se ven alteradas si esta es excluida del conjunto.  La eliminación de palabras ejemplifica que siempre y cuando no se desaparezca el conjunto de palabras migrantes,  el uso (la influencia) de un idioma en otro no cambia; mientras que la diversidad de rango obedece a una curva logarítmica, sin importar si el conjunto es grandes (más de 200 elementos) o pequeño (20 elementos).

Las migraciones de palabras al tratarlas como conjuntos, mantienen las propiedades anteriores, sin importar cuantos elementos conformen el conjunto ni cuales sean.  



