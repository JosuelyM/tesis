\chapter{La Base de datos}

\fxnote{En el titulo del capitulo, solo la primera palabra tiene mayusculas, 
a menos que sea un nombre, etc. Ya la cambie, pero revisa en otross
capitulos, que la capitalizacion este bien.}

\section{La base de datos y su interpretación} % {{{

Para la elaboración del trabajo,  se dispuso de la base de datos de los los
$n$-grams de Google Books\fxnote{Acá tienes que citar}. Esta base de datos
tiene hasta el momento alrededor del 4$\%$ de todas las publicaciones en
diferentes idiomas,  y se caracteriza por en listar  por cada año y por cada
idioma de publicación los ``$n$-gramas'' \fxnote{Mira como se ponen las
comillas en LaTex. Es un poco latoso y sin sentido, pero asi es. Otra
correccion: como $n$ representa un número, prefiero escribir $n$-gramas a
$n$-gramas. Nota el cambio en el símbolo.} más utilizados.   Los $n$-gramas son
las palabras o conjunto de palabras que forman el texto de un libro, donde el
número $n$ indicará la cantidad de palabras que forman el grama.  Es decir, los
1-grama son palabras individuales (una sola palabra o un signo), los 2-gramas
son frases compuestas por dos 1-gramas, los 3-gramas el conjunto de tres 1-gramas
y así sucesivamente.   Con la herramienta del \textit{n-gram viewer}
\cite{ngramv}, los usuarios pueden acceder a una forma visual el comportamiento
a lo largo del tiempo de un $n$-grama en los diferentes idiomas en que se
publicó.  

A partir de esta base, se extrajeron los datos de los 1-grama en cinco
diferentes idiomas, inglés, francés, alemán, italiano y español,
correspondientes a las publicaciones de 269 años (1740-2009).  De acuerdo a
\cite{iplosone},  el kernel de un idioma lo componen entre 1500 a 3000 palabras
más comunes del mismo,  cantidad que basta para conocer al idioma~\fxnote{Este no es
el claim. Reformula}.  Para el
estudio del trabajo se tomaron por cada año y por cada idioma,  las cinco mil
palabras más usadas (cantidad que abarca palabras dentro y fuera del kernel),
teniendo conjuntos homogéneos en tamaño.

Cada palabra está asociada a una frecuencia, que es la cantidad de veces en que
que la  palabra apareció en las publicaciones de un año y un idioma, y a su vez
cada frecuencia se vincula a un rango, que es la posición en un ordenamiento.
Cada listado está ordenado de manera descendente en la frecuencia y de manera
ascendente en el rango, la palabra más utilizada tendrá mayor frecuencia y le
corresponde el rango uno,  la siguiente será la segunda más frecuente
correspondiente al rango dos,  la tercera más frecuente con rango tres y así
sucesivamente para todos los elementos.  Con ello se afirma que las palabras
más usadas tienen una mayor frecuencia y un rango menor. 

% }}}
\section{Forma de búsqueda} % {{{
\fxnote{Creo que el nombre de esta sección no es acertado. Tambien me gustaria
que dijeras por que vas a ahcer lo qeu vas a hacer. }

Se identifican palabras que sean iguales en escritura, carácter por carácter y
que estén presentes en dos o más idiomas.  Si se define como
\textbf{migración}, al movimiento de palabras de un idioma a otro, entonces las
\textbf{palabras migrantes}  son aquellas que están presentes en al
menos dos idiomas\fxnote{no debes capitalizar las deficiones, proque no son nombres
propios}.  

Conviene en este punto realizar dos definiciones que serán
útiles.
% \hfill \break
\fxnote{Acá pusiste un hfill y un break. Para que? Los comenté porque no 
les encuentro sentido. }
\begin{itemize}
\item 
Se define como  \textbf{idioma origen}, al idioma en el cual la palabra
apareció por primera vez dentro de la lista de las cinco mil palabras más
usadas.  

% \hfill \break 
\item 
Se define como \textbf{idioma receptor}, a  aquel donde la palabra está
presente, siendo un conjunto diferente al idioma origen.  
\end{itemize}

% \hfill \break

Para que existan las palabras migrantes, se necesita un origen y al menos un
receptor. Si una palabra está presente en más de dos idiomas, alguno es el
origen y los demás son los receptores. 

Una vez que se tiene certeza de cuáles palabras son migrantes, y en cuáles
idiomas está presente se procede a determinar el idioma origen y los idiomas
receptores.   Si se suponen dos idiomas $\textit{A}$ y $\textit{B}$  donde la
palabra se encuentra, y se sabe con seguridad el año de aparición en cada
idioma y el rango que ocupó para ese año,  el criterio para determinar el
origen es el siguiente: 
\begin{enumerate}
\item  Si la palabra apareció en años anteriores en el idioma $\textit{A}$
(dentro de las 5 mil más usadas) que en el idioma $\textit{B}$ , se establece
al idioma $\textit{A}$  como el idioma origen y $\textit{B}$  como el receptor.
\item Si la palabra apareció en el mismo año en ambos idiomas, se establece el
idioma origen a aquel donde la palabra tuvo un menor rango, es decir, si el
rango en la lista de las más usadas en el idioma $\textit{A}$  es menor que el
rango en la lista del idioma $\textit{B}$ , entonces $\textit{A}$  es el idioma
origen, en caso contrario, el origen es $\textit{B}$ .
\item Se descartan palabras que contengan un solo carácter, las que son
compuestas por letras y números y aquellas que aparecen en el receptor sólo una
vez. La intención es tener una  nueva base de datos con el mayor contenido
limpio. 
\end{enumerate}
\fxnote{Creo que convendría tener un par de ejemplos, como tablas o algo asi, 
que incluyan palabras que migran razonablemente y otras que no, para uqe se entienda
las limitaciones de la definicion. Esto recuerda que dio mucha lata.}
Los argumentos anteriores, se pueden ampliar si la palabra está presente en
tres o más.

% \newpage
\fxnote{Aca pusiste un newpage. Este tipo de comandos normalmente no se usan. }
La manera de clasificar el idioma origen de las palabras puede carecer de otras
pautas para ser más preciso, sin embargo, a lo largo de toda la investigación
se optó por utilizar lo más posible los datos de los $n$-gramas y  con ellos
crear reglas para obtener resultados.  Una forma más precisa sería tomando otro
tipo de base de datos, con información etimológica de las palabras y las
diferentes escrituras que tomó la palabra hasta que prevaleció una estructura.  

Conocidos los idiomas origen y receptor de las palabras migrantes, se llamarán
\textbf{préstamos} a las palabras con un mismo origen y que están  presentes en
un receptor.  Un receptor $\textit{B}$, puede tener palabras con diferentes
orígenes $\textit{A}$, $\textit{C}$ o $\textit{D}$, también una palabra con
origen $\textit{A}$ puede llegar a diferentes receptores.  Los préstamos de
$\textit{A}$  en $\textit{B}$  son el conjunto de palabras con origen
$\textit{A}$  y con receptor $\textit{B}$.     


\subsubsection*{Ventajas}
\fxnote{Esto no es una subsubseccion. Redacta mejor algo que introduzca la lista. 
Tambien quité la linea, para que seas homogeneo cuando ahces una lista. Si quieres
cambiar el simbolo del item en todas las listas, mejor si lo haces de manera global. 
debe haber una forma de hacerlo.}


\begin{itemize}
\item 
% [$-$] 
Determina el idioma donde la palabra fue más popular al comienzo de
la base de datos (1740), y hacia donde se esparció el vocablo, dando un
carácter histórico de los idiomas mas populares en diferentes épocas. 
\item Localiza las palabras que conservaron su escritura al pasar de un
idioma a otro. 
\item Valora a un idioma como importante e influyente si sus vocablos son
transmitidos a los demás y perduran por un periodo de tiempo. También el mismo
idioma es influyente si a través de él se han esparcido palabras a los demás, a
pesar de que no sea el idioma origen. 
\end{itemize}


\subsubsection*{Desventajas}
\begin{itemize}
\item Al encontrar palabras con igual escritura, se obtienen casos donde
el significado en cada idioma es diferente.  Por ejemplo, la palabra
$\textit{MAYOR}$, que está presente en el inglés y en el español,  significa
alcalde en inglés, mientras que en español es más grande que.
\item No se localizan palabras que han sufrido transformaciones en la
escritura al pasar de un idioma a otro, consecuente de que la palabra se adapta
a la gramática de los diferentes receptores.  Por ejemplo, las palabras
$\textit{imagine}$ e $\textit{imaginar}$, son similares en los primeros
caracteres, teniendo el mismo significado en el inglés y en el español, no
obstante la terminación de  sus últimas letras se modificó al estar en  la
lengua inglesa y la española. 
\item Puede definir un origen distinto al verdadero.  Muchos de estos casos se
presentan al no tener una base de datos con más idiomas, y el verdadero origen
puede estar en estas exclusiones. Por ejemplo la palabra natural, que proviene
del latín y se encuentra en inglés y español con igual escritura,  el programa
la identifica proveniente del inglés, más su verdadera procedencia es el latín. 
\end{itemize}

Se han revisado las palabras encontradas con el algoritmo descrito, en su
mayoría los resultados son aceptables. Los mayores inconvenientes resultaron al
clasificar las palabras y las migraciones entre idiomas de la misma familia
lingüística.  En los capítulos siguientes se proporcionara un vinculo para
poder observar todas las palabras encontradas. \fxnote{Ok, veo que pones algunos ejemplos. 
Discutamos si ponemos algo aca como una tabla o no.}

Ya que  la base de datos es amplia y se busca tener palabras cuyo significado
refleja la composición de los idiomas y no a la estructura escrita de los
mismos\fxnote{Esta frase esta mal construida. ni se lo que quieres decir.}. Las
palabras de acuerdo a\cite{contenidopal}, se clasifican en \fxnote{no entiendo 
que uso le das a las italicas y al bold face. Me puedes por favor aclarar eso?}
\textbf{\textit{palabras funcionales,}} aquellas que auxilian a las demás a
estructurar un mensaje de acuerdo a la gramática del idioma, y en
\textbf{\textit{palabras de contenido}} que llevan la información y significado
del mensaje. Para obtener la esencia de las palabras que fluyen entre los
idiomas, se han eliminado de las listas a los artículos, pronombres,
preposiciones y conjunciones, correspondientes a las palabras funcionales,
quedando solo palabras de contenido. \fxnote{Aca estas cambiando totalmente de
tema. Creo que este parrafo no va aca. El siguiente parrafo, tambien esta mal
conectado. El otro que le sigue tambien. }

Al tratarse de cinco diferentes idiomas, las posibles palabras funcionales se
obtuvieron de diferentes diccionarios y páginas \cite{englishdic, frenchdic,
germandic, italiandic, spanishdic}.

En secciones posteriores, se realizará un estudio sobre el cómo afecta a los
resultados la eliminación de palabras a partir de reglas arbitrarias.  Por el
momento todo el trabajo se apoya en la eliminación de las palabras funcionales.
% }}}
\section{Partición de datos} % {{{

\fxnote{No entiendo esta seccion }

Al ser limitada la base de datos por no haber registros (suficiente
información) para los cinco idiomas (inglés, francés, alemán, italiano y
español) antes de 1740,  existe un periodo de tiempo durante los primeros
treinta o cincuenta años, donde se encuentran una mayor cantidad de préstamos
de un idioma a otro. Este periodo desfasa a  las migraciones ya que sin
importar cual sea el año del comienzo y los idiomas utilizados,  la mayor
cantidad de migraciones sigue ocurriendo años después del inicio.   

Para estabilizar el flujo de palabras entre idiomas, se decidió partir la
información en dos conjuntos que ayuden al análisis.

\textbf{Conjunto base.} (1740-1900) todos los movimientos encontrados en esta
partición se tomarán como verdaderos,  obteniendo un \textit{background} de
las palabras que migraron de un lado a otro.

\textbf{Conjunto de búsqueda.} (1901-2009) será donde se muestren los
resultados de cada técnica elaborada y donde mayor discusión se realizará,
apoyados de los resultados del conjunto base. 
% }}}

