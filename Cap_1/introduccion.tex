
% this file is called up by thesis.tex
% content in this file will be fed into the main document
%----------------------- introduction file header -----------------------
%%%%%%%%%%%%%%%%%%%%%%%%%%%%%%%%%%%%%%%%%%%%%%%%%%%%%%%%%%%%%%%%%%%%%%%%%
%  Capítulo 1: Introducción- DEFINIR OBJETIVOS DE LA TESIS              %
%%%%%%%%%%%%%%%%%%%%%%%%%%%%%%%%%%%%%%%%%%%%%%%%%%%%%%%%%%%%%%%%%%%%%%%%%

\chapter{Introducción}




\newpage

\section{Primeros objetivos del análisis}

Una premisa que puede explicar el por qué las palabras se alteran o no al pasar de un idioma a otro, se puede vincular a la adaptabilidad de la palabra por parte de los hablantes de la lengua.  Por ejemplo, la palabra \textit{internet} con un claro origen en el inglés, y año de “invención” alrededor de 1990,  migró y cobró relevancia en otros idiomas por la facilidad y la rapidez con la que las sociedades  aprovecharon el fenómeno de internet, beneficiándose de la revolución que trajo en aspectos como el desarrollo de las telecomunicaciones,  el avance tecnológico,  la sofisticación de aparatos, el cambio positivo en el modo de vivir de las personas, entre otros beneficios.  Tal adaptación por parte de la sociedad hizo cotidiano el concepto, y en consecuencia la palabra.  

Otro palabra que sirve como ejemplo para este razonamiento es \textit{wi-fi}, donde la mayoría de las personas entiende el significado y el concepto que describe la palabra. Probablemente  exista una traducción para describir al fenómeno en cada idioma, sin embargo la cotidianidad que existe entre la escritura original y la comunidad que lo emplea, hace inviable una modificación que resulte en la pérdida de la familiaridad.


Un caso donde un concepto y fenómeno ha logrado un cambio en la forma de vida de las comunidades,  y la palabra que se asocia a él ha sido transformada por los receptores es el del \textit{teléfono}; su escritura original \textit{teletrofono} es proveniente del italiano,  pero su apogeo se dio en el inglés, modificando la escritura a \textit{telephone}; los diferentes idiomas también hicieron una modificación, siendo \textit{téléphone} en francés, \textit{telefon} en alemán, \textit{teléfono} en español, e incluso el italiano adopta la escritura del español   (el n-gram viewer es útil para comprobar esto), a pesar de ser el idioma origen. Pocos cuestionan el cambio que originó el teléfono en la sociedad desde su aparición hasta  tiempos recientes, pero contrario al internet su escritura no prevaleció.  

El tiempo que le toma a las palabras el pasar de un idioma a otro, es un factor que influye en la prevalencia de las palabras. El ejemplo del internet en los años recientes,  donde  la globalización ha permitido que la información pase de un medio a otro con mayor fluidez y el fenómeno que conlleva la palabra sea asimilado por las diferentes comunidades en un tiempo relativamente corto entre el origen de la palabra y su uso en los demás idiomas. 

El encontrar factores o sucesos  que expliquen el por qué las palabras fluyen de un lado a otro, será uno de los objetivos del trabajo. Más objetivos se irán planteando conforme se avance en el texto, para llegar al propósito del análisis, permitiendo establecer una forma de cuantificar la influencia entre los idiomas. Antes de llegar a este punto, habrá que hacer más clasificaciones para un mejor manejo de los datos. 

\newpage

\section{Clasificaciones de las palabras}

La primera clasificación  hecha fue el identificar  el  idioma origen y los diferentes idiomas receptores para las palabras migrantes, para establecer los préstamos de un idioma en otro, tras esta organización de datos, y como primer paso para llegar a la influencia de una lengua sobre otra,  es útil conocer qué tan importantes han sido los préstamos de un mismo origen en los demás idiomas, y también en qué año o años un idioma receptor ha adoptado recibido más préstamos de diferentes orígenes. 

Si se supone un idioma receptor B, un año cualquiera que esté dentro de los datos y  la lista  de las 5 mil palabras más usadas de B para tal año, entonces dentro del registro habrá préstamos con idioma  origen A en alguna posición de rango.  Estos préstamos se han clasificado como:


\begin{description}

	\item [Préstamos Nuevos:] Son palabras que aparecen por primera vez en las más usadas del idioma B.
	
	\item [Préstamos Repetidos:] Palabras que ya habían aparecido en alguna lista del idioma B, y para ese año lo volvieron a hacer.
	
	\item [Préstamos Acumulados:] Conjunto de préstamos nuevos y repetidos.
	
	
\end{description}


Por lo tanto para un determinado año dentro de la lista de las más usadas, existirán palabras que aparecieron en la lista por primera vez y palabras que ya habían estado en la lista en al menos un año anterior.  Asimismo  habrá años en los que no existieron préstamos nuevos,  por lo que todos los préstamos por año serán repetidos.




\newpage
\section{Interpretación de la influencia}

El medir la influencia que un evento tiene sobre otro, no es un proceso que esté mecanizado, como lo puede ser el determinar la distancia entre dos puntos o evaluar el tamaño de un objeto.  No existe un conjunto de reglas que afirmen o refuten si un acontecimiento es influyente en otro. En cada evento existe una cantidad diferente de variantes que intervienen entre ellas para  conducir a una respuesta sobre la influencia. 

En el caso de los idiomas, las variables que se conocen son el tiempo (manifestado en los años de las listas), la frecuencia y el rango de las palabras.  Un dato importante es que la cantidad de libros que fueron digitalizados es mayor en los últimos años que en los primeros,  la frecuencia de la palabra con rango uno en las lista de 1740 puede ser la diezmilésima parte de la del mismo rango en la lista de 2009, inclusive tratándose de la misma palabra,  entonces se tendrán que normalizar los datos para que tengan la misma proporción. 

De nuevo, se piensa en los préstamos de A, y los de C que están presentes en B.  Una manera preliminar de conocer la influencia, es contar cuántos préstamos de cada idioma están en las listas de cada año en B,  y el idioma más influyente será el que tenga más elementos. La idea es buena, pero no suficiente, para demostrarlo, se piensa en la lista de un año en B,  y al contar los préstamos, se obtiene una cantidad X para los que tienen origen A y  una cantidad Y para los de origen C,  con X $>$ Y. 

Al utilizar el criterio anterior,  el idioma A es en principio más influyente que el C por tener más cantidad de elementos en B, pero si los préstamos de C ocupan rangos más pequeños que los de A, como el tener un rango pequeño es similar a tener una frecuencia mayor,  si se suma cada frecuencia  de los préstamos de A y C,  la suma para las de C puede ser mayor que la suma para las de A, en consecuencia las palabras de C son más importantes dentro de B que las palabras de A, ya que son más frecuentadas o son utilizadas más veces.   

El utilizar la frecuencia es otra herramienta con la que se logra sustentar la respuesta a la influencia, puede ser alternativa o complementaria a la basada en la cantidad, en el trabajo se empleara cada una para describir un caso particular, aún falta añadir la normalización,  de manera formal se describen los dos métodos y se recuerda que las listas son de las cinco mil palabras más usadas. 

\begin{description}
	
	\item[Influencia por cantidad:] Se toma como influencia a la cantidad de préstamos de A en la lista de un año en B.
	
	\item[Influencia por frecuencia:] La influencia estará determinada por relacion porcentual entre las frecuencias de los préstamos de A en la lista de B y la frecuencia de todas las palabras de la lista de B. 
	
\end{description}


Para entender mejor  la influencia por frecuencia, se describira a detalle el algoritmo empleado:

\begin{enumerate}
	
	\item En un año determinado del idioma B, se sumarán las frecuencias $f_{t}$ de cada una de las cinco mil palabras más usadas.  Esta cantidad se llamará \textbf{frecuencia  total por año.}
	
	\begin{equation}
	\label{ec.ftot}
	f_{t} = \sum_{i=1}^{5000} f_{i} \,\,\,\,\,\,\,\,\, i = posici\'{o}n \,\,\, de \,\,\,cada \,\,\,palabra
	\end{equation}
	
	\item Como se conocen las posiciones $j$,  que ocupan los préstamos A en la lista de B, se procede a sumar sólo las frecuencias asociadas a estas palabras. Esta cantidad será la \textbf{frecuencia de préstamo} $f_{p}$,  esta cantidad es siempre menor que la frecuencia total por año.
	
	\begin{equation}
	\label{ec.fpres}
	f_{p} = \sum_{j} f_{j} \,\,\,\,\,\,\,\,\, j = posici\'{o}n \,\,\, de \,\,\,cada \,\,\,prestamo
	\end{equation}
	
	
	\item  Se divide la frecuencia de préstamo entre la frecuencia total por año, esta cantidad es la indicada para medir la influencia, se denotará como \textbf{frecuencia de uso} $F$ y es la porción que representan las frecuencias de los préstamos de A en B.  Como la influencia por cantidad se expresa como un porcentaje,  bastará multiplicar el cociente por cien para obtener el porcentaje.  
	
	\begin{equation}
	\label{ec.fuso}
	F = \frac{f_{p}}{f_{t}} * 100
	\end{equation}
	
	
	Entre más cercana a 100 $\%$ sea la frecuencia de uso, los préstamos del idioma A serán más relevantes en B.
	
	
\end{enumerate}


La frecuencia de uso está ya normalizada, siendo el parámetro de normalización la frecuencia total por año. Otra posible normalización sería al considerar la cantidad de libros que fueron registrados para obtener la lista de las palabras más usadas en cada año,  sin embargo se desconoce esta cifra.  


\newpage
\section{La influencia en 109 años (1900-2009)}

Como se tienen dos maneras de medir influencia,  hay casos en los que conviene utilizar solo una forma o ambas.  Para los préstamos nuevos, se decidió emplear el método de influencia por cantidad, ya que mostrara los años de mayor intercambio de palabras que se asientan en el idioma receptor; el método de influencia por frecuencia no proporciona información relevante en estos préstamos ya que el impacto que representan las palabras nuevas en la frecuencia de uso del idioma receptor es muy pequeña, ya que generalmente las palabras nuevas entran a las listas en posiciones muy altas de rango.  

La influencia por frecuencia brinda mayor información si se emplea en los préstamos por año, al evaluar la relevancia de todos los préstamos del idioma origen en el receptor. En este punto es importante el background que se obtuvo del conjunto base, ya que también se contó la influencia de los préstamos acumulados que surgieron en el background, debido a que para el primer año del conjunto búsqueda (1901) ya forman parte del idioma receptor, y tienen un impacto en él. 

De manera resumida se hicieron por el momento dos análisis, el primero obteniendo la influencia por cantidad de los préstamos entre dos idiomas entre 1901 y 2009, y el segundo al contabilizar la influencia por frecuencia de los préstamos acumulados durante los mismos años.  Los resultados de cada prueba se presentarán en las siguientes secciones. 


