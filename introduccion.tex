\chapter{Introducción}
\jmnote{corregir sin tanto rollo}

%Entre los sistemas que existen en la naturaleza, se conocen como  \textbf{sistemas complejos} a aquellos que  están conformados por una gran cantidad de elementos, interactuando mutuamente entre si. Una de las características de los sistemas complejos, es que al haber tantas interacciones como componentes, se pueden obtener diversas descripciones para su fenomenología,  si alguna componente es modificada.

Entre los sistemas que se encuentran en la naturaleza, los \textbf{sistemas complejos} están conformados por una gran cantidad de componentes, interactuando mutuamente entre sí. Estas interacciones no se pueden separar y estudiar de forma aislada, ya que el estado del sistema lo determinará el estado colectivo. 

%Entre los sistemas que se encuentran en la naturaleza, los \textbf{sistemas complejos} están conformados por una gran cantidad de componentes, interactuando mutuamente entre sí. Estas interacciones no se pueden separar y estudiar de forma aislada, ya que el estado del sistema, lo determinarán el estado de todas las posibles interacciones. 


%Entre los sistemas que existen en la naturaleza, se encuentran los \textbf{sistemas complejos}; estos están conformados por una gran cantidad de elementos, interactuando mutuamente entre si. En los sistemas complejos al haber tantas interacciones como componentes, se pueden obtener diversas descripciones para la fenomenología del sistema,  si alguna componente es modificada. 

%En la historia del pensamiento científico, se han estudiado los fenómenos en sistemas de la naturaleza, a partir de desarrollar leyes con carácter lógico matemático que se acoplen a las propiedades del sistema. Estas leyes son idealizadas, ya que con ellas se logra entender y predecir el comportamiento y la evolución del sistema. 

%También se ha idealizado que el desarrollo de tales leyes, comienza por simplificar las propiedades del sistema para reducirlas a diferentes causas, donde la primera de ellas origina a las demás. Este método linealizado no siempre es óptimo, ya hay sistemas con tantas componentes, interactuando todas entre si, que pueden tener diferentes descripciones si alguna es modificada; este tipo de sistemas se denominan como \textbf{sistemas complejos}.

Los sistemas complejos se pueden encontrar en diferentes áreas, en las ciencias atmosféricas al estudiar el clima \cite{complex_climate};  en la medicina y en la biología al estudiar las funciones de los organismos vivos  \cite{complex_medicine}; y también en la lingüística al estudiar a los idiomas  \cite{complex_language}.

El tratamiento de los idiomas como sistemas complejos, comenzó con el trabajo de George Zipf \cite{zipf}. En él se plantea que al ordenar las palabras de un texto a partir de su frecuencia de aparición, donde el rango uno lo ocupa la palabra más frecuente, el rango dos la segunda palabra más frecuente, y así sucesivamente; entonces la frecuencia relativa $f$ de cada palabra y el rango $k$ de la misma, están relacionados por la forma $f\sim1/k$. La expresión anterior se conoce como \textit{ley de Zipf}, y su validez se ha demostrado en corpus donde se ordenan las palabra más usadas en diferentes idiomas \cite{tesis.sergio}, en clasificaciones de deportes y juegos \cite{epj}, en el producto interno bruto de distintos países \cite{zipf_gdp}, entre otros. 

Pese a que la ley de Zipf abrió diversos estudios estadísticos en lingüística, aun no se han realizado suficientes trabajos para entender cómo en el vocabulario de un idioma, están mezclados términos del idioma mismo y de otros lenguajes. 

Actualmente en la lengua española, se encuentran palabras de la lengua inglesa que no tienen una traducción o que en ocasiones, desplazan a las ya existentes en el español. Por ejemplo, es común escuchar la palabra \textit{marketing} en vez de \textit{mercadeo} cuando se tratan temas económicos o relacionados a los negocios; también el termino \textit{online} ha sustituido a \textit{en linea}, al referirse a temas relacionados con el \textit{internet}; una palabra sin traducción en el español. 

Esta tendencia no solo ha afectado al español, sino también a los demás idiomas que han sido influidos, por ámbitos donde el inglés es el idioma común para la comunicación.  No obstante,  en diferentes periodos de tiempo, el flujo mayoritario de palabras provenía de otras lenguas. En \cite{influencia_mutua} se discute con un rigor lingüístico e histórico, 
el intercambio de palabras entre el inglés y el español, además se menciona la influencia del árabe en el español y del francés en el inglés durante la época renacentista.  

En este trabajo, se emplearan herramientas estadísticas para entender el cómo se relacionan las palabras de diferentes idiomas, propiamente del inglés, del francés, del alemán, del italiano y del español;  además se construirán distintos métodos para cuantificar la influencia que un idioma ha tenido en los demás, en diferentes periodos de tiempo; así como explicar las causas que originaron la influencia y el flujo de palabras. 




