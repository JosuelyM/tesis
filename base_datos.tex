\chapter{La base de datos}

\fxnote{En las subsecciones aun tienes mayúsculas. Por ejemplo, la 6.1. Se puede ver del índice. Corrige.}

\fxwarning{listo}

\section{La base de datos y su interpretación} % {{{

Para la elaboración del trabajo,  se dispuso de la base de datos de los los
$n$-grams de Google Books  \cite{ngramv}. Esta base de datos
tiene hasta el momento alrededor del $4\%$ de todas las publicaciones en
diferentes idiomas,  y se caracteriza por en listar  por cada año y por cada
idioma de publicación los ``$n$-gramas'' más utilizados.   Los $n$-gramas son
las palabras o conjunto de palabras que forman el texto de un libro, donde el
número $n$ indicará la cantidad de palabras que forman el grama.  Es decir, los
1-grama son palabras individuales (una sola palabra o un signo), los 2-gramas
son frases compuestas por dos 1-gramas, los 3-gramas el conjunto de tres 1-gramas
y así sucesivamente.   Con la herramienta del \textit{n-gram
viewer}~\cite{ngramv}, los usuarios pueden acceder a una forma visual el
comportamiento a lo largo del tiempo de un $n$-grama en los diferentes idiomas
en que se publicó.  

A partir de esta base, se extrajeron los datos de los 1-grama en cinco
diferentes idiomas, inglés, francés, alemán, italiano y español,
correspondientes a las publicaciones de 269 años (1740-2009).  De acuerdo a
\cite{languagesascool},  el kernel de un idioma lo componen entre 1500 a 3000
palabras más comunes del mismo,  cantidad que basta para conocer lo esencial
del idioma. Para el
estudio del trabajo se tomaron por cada año y por cada idioma,  las cinco mil
palabras más usadas (cantidad que abarca palabras dentro y fuera del kernel),
teniendo conjuntos homogéneos en tamaño.

Cada palabra está asociada a una frecuencia, que es la cantidad de veces en que
que la  palabra apareció en las publicaciones de un año y un idioma, y a su vez
cada frecuencia se vincula a un rango, que es la posición en un ordenamiento.
Cada listado está ordenado de manera descendente en la frecuencia y de manera
ascendente en el rango, la palabra más utilizada tendrá mayor frecuencia y le
corresponde el rango uno,  la siguiente será la segunda más frecuente
correspondiente al rango dos,  la tercera más frecuente con rango tres y así
sucesivamente para todos los elementos.  Con ello se afirma que las palabras
más usadas tienen una mayor frecuencia y un rango menor. 

% }}}
\section{Palabras comunes entre idiomas} % {{{

Al tratar de identificar la presencia de un idioma en otro, una forma
de lograrlo es reconocer aquellas palabras que son comunes entre dos o más
idiomas, y a partir de ellas se determinar como se alteran los idiomas por la presencia y el uso de vocablos cuyo origen proviene de un idioma externo. \fxnote{La redacción  de esta frase te quedo muy mal.}
\fxwarning{listo, corregida!}

Para ello se identifican palabras que sean iguales en escritura, carácter por carácter y que estén presentes en dos o más idiomas.  Si se define como \textbf{migración}, al movimiento de palabras de un idioma a otro, entonces las \textbf{palabras migrantes}  son aquellas que están presentes en al menos dos idiomas.  

Conviene en este punto realizar dos definiciones que serán útiles.
\begin{itemize}
\item 
Se define como  \textbf{idioma origen}, al idioma en el cual la palabra
apareció por primera vez dentro de la lista de las cinco mil palabras más
usadas.  
\item 
Se define como \textbf{idioma receptor}, a  aquel donde la palabra está
presente, siendo un conjunto diferente al idioma origen.  
\end{itemize}

Para que existan las palabras migrantes, se necesita un origen y al menos un
receptor. Si una palabra está presente en más de dos idiomas, alguno es el
origen y los demás son los receptores. 

Una vez que se tiene certeza de cuáles palabras son migrantes, y en cuáles
idiomas está presente se procede a determinar el idioma origen y los idiomas
receptores.   Si se suponen dos idiomas $\textit{A}$ y $\textit{B}$  donde la
palabra se encuentra, y se sabe con seguridad el año de aparición en cada
idioma y el rango que ocupó para ese año,  el criterio para determinar el
origen es el siguiente: 
\begin{enumerate}
\item  Si la palabra apareció en años anteriores en el idioma $\textit{A}$
(dentro de las 5 mil más usadas) que en el idioma $\textit{B}$ , se establece
al idioma $\textit{A}$  como el idioma origen y $\textit{B}$  como el receptor.
\item Si la palabra apareció en el mismo año en ambos idiomas, se establece el
idioma origen a aquel donde la palabra tuvo un menor rango, es decir, si el
rango en la lista de las más usadas en el idioma $\textit{A}$  es menor que el
rango en la lista del idioma $\textit{B}$ , entonces $\textit{A}$  es el idioma
origen, en caso contrario, el origen es $\textit{B}$ .
\item Se descartan palabras que contengan un solo carácter, las que son
compuestas por letras y números y aquellas que aparecen en el receptor sólo en un año. La intención es tener una  nueva base de datos con el mayor contenido
limpio con palabras que perduran al menos dos años en el receptor.

\end{enumerate}
Los argumentos anteriores, se pueden ampliar si la palabra está presente en tres o más.

La manera de clasificar el idioma origen de las palabras puede carecer de otras
pautas para ser más preciso, sin embargo, a lo largo de toda la investigación
se optó por utilizar lo más posible los datos de los $n$-gramas y  con ellos
crear reglas para obtener resultados.  Una forma más precisa sería tomando otro
tipo de base de datos, con información etimológica de las palabras y las
diferentes escrituras que tomó la palabra hasta que prevaleció una estructura.  

Conocidos los idiomas origen y receptor de las palabras migrantes, se llamarán
\textbf{préstamos} a las palabras con un mismo origen y que están  presentes en
un receptor.  Un receptor $\textit{B}$, puede tener palabras con diferentes
orígenes $\textit{A}$, $\textit{C}$ o $\textit{D}$, también una palabra con
origen $\textit{A}$ puede llegar a diferentes receptores.  Los préstamos de
$\textit{A}$  en $\textit{B}$  son el conjunto de palabras con origen
$\textit{A}$  y con receptor $\textit{B}$.     



Los criterios empleados para establecer orígenes ha mostrado mayormente
resultados adecuados, sin embargo se presentaron situaciones donde el algoritmo
no logra asentar el origen adecuado. Dentro de las ventajas ($\bullet$) y
desventajas ($\ast$) del método se encuentran las siguientes:
\begin{itemize}
\item [$\bullet$] 
% [$-$] 
Determina el idioma donde la palabra fue más popular al comienzo de
la base de datos (1740), y hacia donde se esparció el vocablo, dando un
carácter histórico de los idiomas mas populares en diferentes épocas. 
\item [$\bullet$] 
Localiza las palabras que conservaron su escritura al pasar de un
idioma a otro. 
\item [$\bullet$] 
Valora a un idioma como importante e influyente si sus vocablos son
transmitidos a los demás y perduran por un periodo de tiempo. También el mismo
idioma es influyente si a través de él se han esparcido palabras a los demás, a
pesar de que no sea el idioma origen. 
\item [$\ast$] Al encontrar palabras con igual escritura, se obtienen casos donde
el significado en cada idioma es diferente.  Por ejemplo, la palabra
$\textit{mayor}$, que está presente en el inglés y en el español,  significa
alcalde en inglés, mientras que en español es más grande que.
\item [$\ast$]
No se localizan palabras que han sufrido transformaciones en la
escritura al pasar de un idioma a otro, consecuente de que la palabra se adapta
a la gramática de los diferentes receptores.  Por ejemplo, las palabras
$\textit{imagine}$ e $\textit{imaginar}$, son similares en los primeros
caracteres, teniendo el mismo significado en el inglés y en el español, no
obstante la terminación de  sus últimas letras se modificó al estar en  la
lengua inglesa y la española. 
\item [$\ast$]
 Puede definir un origen distinto al verdadero.  Muchos de estos casos se
presentan al no tener una base de datos con más idiomas, y el verdadero origen
puede estar en estas exclusiones. Por ejemplo la palabra natural, que proviene
del latín y se encuentra en inglés y español con igual escritura,  el programa
la identifica proveniente del inglés, más su verdadera procedencia es el latín. 
\end{itemize}

Se han revisado las palabras encontradas con el algoritmo descrito, en su
mayoría los resultados son aceptables. Los mayores inconvenientes resultaron al
clasificar las palabras y las migraciones entre idiomas de la misma familia
lingüística.  En los capítulos siguientes se proporcionara un vinculo para
poder observar todas las palabras encontradas.


Las palabras que se encontraron se pueden agrupar en dos conjuntos de acuerdo
a~\cite{contenidopal}. Las \textbf{palabras funcionales} son aquellas que
auxilian a las demás a estructurar un mensaje de acuerdo a la gramática del
idioma y las \textbf{palabras de contenido} son las que llevan la información
y el significado del mensaje. Ya  que se busca tener palabras cuyo significado
refleje un concepto o situación donde estas son utilizadas, se decidió eliminar
de las listas a los artículos, pronombres, preposiciones y conjunciones,
correspondientes a las palabras funcionales, quedando solo palabras de
contenido. Para eliminar las posibles palabras funcionales se emplearon
diferentes paginas y diccionarios especializados en cada lengua
\cite{englishdic, frenchdic, germandic, italiandic, spanishdic}. 

% }}}
\section{Partición de datos} % {{{

El tener una base de datos que comprende 269 años (1740-2009) no garantiza ser
lo suficientemente completa para determinar el año preciso en que una palabra
apareció en otro idioma, hay palabras cuyo primer empleo es más antiguo que el
primer año en que se tienen registros. Este tipo de palabras tienen una
característica común y es que migran en los primeros treinta años, sin importar
el año en que comiencen las mediciones; a pesar de ser una característica
importante el flujo de palabras siempre es mayor en este periodo, posterior a
el las migraciones son menos frecuentes. Para estabilizar el flujo de palabras,
se decidió partir la información en dos conjuntos que cuyo contenido se emplee
de forma diferente y se complemente uno a otro.  Se llamarán como:

\textbf{Conjunto base.} (1740-1899) Todas las migraciones que ocurran dentro de
estos años se tomarán como verdaderas, obteniendo un antecedente o historial de
las palabras que migraron de un lado a otro y se han asentado en el idioma
receptor. Con ello se establece una base de las palabras que ya forman parte de
un receptor y que son relevantes al estar en las listas de sus palabras mas
comunes. \fxnote{Porque usas esos años? Y además, porque no hasta 1899?}
\fxwarning{Me equivoque y si es hasta 1899,  lo de los años es para ver las migraciones en el siglo pasado y los años que abarco de este}

\textbf{Conjunto de búsqueda.} (1900-2009) Las migraciones en estos años se
cuestionarán tratando de dar un significado al por que ha ocurrido el
movimiento,  siendo en este periodo las migraciones más estables. Para las
posteriores técnicas se utilizará siempre al conjunto de búsqueda.
% }}}

