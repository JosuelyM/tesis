%%%%%%%%%%%%%%%%%%%%%%%%%%%%%%%%%%%%%%%%%%%%%%%%%%%%%%%%%%%%%%%%%%%%%%%%%%%%%%%%
%                         FORMATO DE TESIS UMSNH                               %
%%%%%%%%%%%%%%%%%%%%%%%%%%%%%%%%%%%%%%%%%%%%%%%%%%%%%%%%%%%%%%%%%%%%%%%%%%%%%%%%
% based on Harish Bhanderi's PhD/MPhil template, then Uni Cambridge
% http://www-h.eng.cam.ac.uk/help/tpl/textprocessing/ThesisStyle/
% corrected and extended in 2007 by Jakob Suckale, then MPI-iCBG PhD programme
% and made available through OpenWetWare.org - the free biology wiki
% forked from https://github.com/Tepexic/Tesis-UNAM on July 2017
% modifications made by Arturo Lopez Pineda

%                     Under GNU License v3

% ADAPTADO PARA UMSNH:  @arturolp

\documentclass[twoside,11pt]{Latex/Classes/thesisUMSNH}
%         PUEDEN INCLUIR EN ESTE ESPACIO LOS PAQUETES EXTRA, O BIEN, EN EL ARCHIVO "PhDthesisPSnPDF.cls" EN "./Latex/Classes/"
\usepackage{blindtext}                        % Para insertar texto dummy, de ejemplo, pues.
\usepackage[sort, numbers]{natbib}    % Personalizar la bibliografía a gusto de cada quien
\usepackage{url}

\usepackage{multirow}
\usepackage{tcolorbox}
\usepackage{colortbl}
\usepackage{amsmath}
\usepackage{caption}
% Note:
% The \blindtext or \Blindtext commands throughout this template generate dummy text
% to fill the template out. These commands should all be removed when 
% writing thesis content.
\include{Latex/Macros/MacroFile1}           % Archivo con funciones útiles
% \usepackage[draft,inline,nomargin]{fixme} \fxsetup{theme=color}
% \FXRegisterAuthor{cp}{cfp}{\color{blue}Carlos}
% \FXRegisterAuthor{jm}{jmb}{\color{red}Josué}


\definecolor{C1-EN}{RGB}{0,132,170}
\definecolor{C1-FR}{RGB}{255,117,0}
\definecolor{C1-GE}{RGB}{134,0,179}
\definecolor{C1-IT}{RGB}{49,129,6}
\definecolor{C1-SP}{RGB}{162,0,0}

\definecolor{bueno}{RGB}{131, 246, 234}
\definecolor{malo}{RGB}{254, 112, 11}

\graphicspath{{images/},{figures/}}

\usepackage{lineno}
\linenumbers

%%%%%%%%%%%%%%%%%%%%%%%%%%%%%%%%%%%%%%%%%%%%%%%%%%%%%%%%%%%%%%%%%%%%%%%%%%%%%%%%
%                                   DATOS                                      %
%%%%%%%%%%%%%%%%%%%%%%%%%%%%%%%%%%%%%%%%%%%%%%%%%%%%%%%%%%%%%%%%%%%%%%%%%%%%%%%%
\title{Análisis estadístico del flujo de 1-gramas entre lenguajes indoeuropeos}
\author{JOSUÉ ELY MOLINA BECERRA} 
\facultad{FACULTAD DE CIENCIAS}                 % Nombre de la facultad/escuela
\escudofacultad{Latex/Classes/Escudos/fc_negro.pdf} % Aquí ponen la ruta y nombre del escudo de su facultad, actualmente, la carpeta Latex/Classes/Escudos cuenta con los siguientes escudos:
% "fi_azul" Facultad de ingenieria en color azul
% "fi_negro" Facultad de ingenieria en color negro
% "fc_azul" Facultad de ciencias en color azul
% "fc_negro" Facultad de ciencias en color negro
% Se agradecen sus aportaciones de escudos a jebus.velazquez@gmail.com

\degree{FÍSICO}       % Carrera
\director{CARLOS FRANCISCO PINEDA ZORRILLA}                   % Director de tesis
%\tutor{Nombre  Tutor }                    % Tutor de tesis, si aplica
\degreedate{2019}                                     % Año de la fecha del examen
\lugar{CIUDAD DE MÉXICO}                        % Lugar

\portadafalse                              % Portada en NEGRO, descomentar y comentar la línea siguiente si se quiere utilizar
%\portadatrue                                % Portada en COLOR



%% Opciones del posgrado (descomentar si las necesitan)
	%\posgradotrue                                                    
	%\programa{programa de maestría y doctorado en ingeniería}
	%\campo{Ingeniería Eléctrica - Control}
	%% En caso de que haya comité tutor
	%\comitetrue
	%\ctutoruno{Dr. Emmet L. Brown}
	%\ctutordos{Dr. El Doctor}
%% Datos del jurado                             
	%\presidente{Dr. 1}
	%\secretario{Dr. 2}
	%\vocal{Dr. 3}
	%\supuno{Dr. 4}
	%\supdos{Dr. 5}
	%\institucion{el Instituto de Ingeniería, UNAM}

\keywords{tesis,autor,tutor,etc}            % Palablas clave para los metadatos del PDF
\subject{tema_1,tema_2}                     % Tema para metadatos del PDF  

%%%%%%%%%%%%%%%%%%%%%%%%%%%%%%%%%%%%%%%%%%%%%%%%%%%%%
%                   PORTADA                         %
%%%%%%%%%%%%%%%%%%%%%%%%%%%%%%%%%%%%%%%%%%%%%%%%%%%%%
\begin{document}

\maketitle									% Se redefinió este comando en el archivo de la clase para generar automáticamente la portada a partir de los datos

%%%%%%%%%%%%%%%%%%%%%%%%%%%%%%%%%%%%%%%%%%%%%%%%%%%%%
%                  PRÓLOGO                          %
%%%%%%%%%%%%%%%%%%%%%%%%%%%%%%%%%%%%%%%%%%%%%%%%%%%%%
\frontmatter
\begin{dedication}

Sólo se puede combatir lo que se ama, y se tiene que conocer lo que combate como experiencia, por dentro, para poder escribir sobre ello.

Carlos Fuentes
\end{dedication}
          % Comentar línea si no se usa
\begin{acknowledgementspersonal}
	
Principalmente a mi abuela Nieves, por haberme cuidado y procurado tantos años sin ser su responsabilidad. 

\vspace{0.5cm}
A mis tios, padrinos y en muchas ocasiones mis padres, Dolores y Oscar; mis primas Flor y Yoalli, también Yurai. Gracias por apoyarme, aceptarme y darme una convivencia familiar.

\vspace{0.5cm}
A mi madre María de la Luz, por darme en mi niñez el aprecio a los libros y a la lectura; y porque en los últimos años, con su desinterés y su desconfianza, me enseñó a sólo depender en mi mismo. 

\vspace{0.5cm}
A mis primos Román, Alfonso y Sinar, por mostrarme que el trabajo duro nos lleva más lejos de la realidad donde nos toco nacer. 

\vspace{0.5cm}
A mi familia paterna, mis primos Liz, Carlos, Griss, Sandy, Elias, Martha y Yadi; mis tias Paty, Cata, Lupe y Xila; y a toda la \textit{Molinada sin fin}, por aceptarme dentro de una gran familia y por ayudarme a cumplir mi deseo.  

\vspace{0.5cm}
A Pamela, Diego, Zoé, Hilary y Yultzín, por ser mi mejor recuerdo de Fundación;  a Maffer, Cabral, Gaby y Alejandro por los \textit{buenos aires} del primer año de preparatoria; a Mauricio, Charly y Alex Califas por los mejores días en las canchas y en \textit{Potrero Dome}; a Brenda, Hodek y Kas \textit{Panda} por las platicas de humanidades, artes, historia y filosofía.  

\vspace{0.5cm}
Nuevamente a Pamela, Zoé, Maffer, Cabral, Mauricio y Charly, porque ustedes se han vuelto más que mis mejores amigos, gracias a ustedes y a sus familias por lo vivido y lo aprendido.

\vspace{0.5cm}
A Carla, por los años en la universidad en los que me mostró amistad, cariño y comprensión; por enseñarme a trabajar en equipo y también por llevarme a \textit{lo más alto y lo más bajo}.

\vspace{0.5cm}
A mis hermanos Carlos, Pavlos, George y Jack,  y a mi hermana Aila, porque ustedes son el mejor regalo y la mejor ilusión que el tiempo me brindó, y porque su existencia es la motivación para cruzar el mundo.

\vspace{0.5cm}
Finalmente a mi padre Carlos Molina Jiménez, porque su ausencia y su ejemplo, me hicieron ver que la formación académica no lo es todo,  y a la vez es la camino para superarnos y llegar a lo más alto. 


\end{acknowledgementspersonal}   % Comentar línea si no se usa 
%\chapter*{}
%\pagenumbering{Roman}

\begin{acknowledgementsacademic}

Gracias a la UNAM por la formación académica recibida, y por el ejemplo profesional que me brindó en su personal, en la Preparatoria 9 con los profesores Norma Ramírez, Patricia Huerta, Elsa Cano, Arturo Palafox, José Carbajal y Arturo Jiménez; y en la Facultad de Ciencias con  Rosario Paredes, Patricia Goldstein, Pablo Barrera, Mirna Villavicencio, Valentín Porta, Ricardo Méndez, Roxana del Castillo y Catalina Stern. Gracias a todos por su labor, su ética y moral profesional,  y por todo el apoyo brindado. 

\vspace{0.5cm}
Gracias a todos los miembros del  grupo de idiomas, por formarme y enseñarme que se puede hacer física en las cosas más cotidianas, no sólo en los fenómenos abstractos. 

\vspace{0.5cm}
Gracias a mi asesor el Dr. Carlos Pineda, por sus consejos, su compresión, su apoyo y paciencia en las múltiples reuniones para elaborar este trabajo. También porque me enseño que en la ciencia,  lo más importante no son las formulaciones y los desarrollos complicados, sino la explicación y la sencillez con la que nos expresamos. 

\vspace{0.5cm}
Finalmente, gracias a los apoyos de los proyectos PAPIIT IG100518 y CONACYT CB-285754, ya que con ellos fue posible cada parte del desarrollo de está tesis. 

\end{acknowledgementsacademic}





%\include{Declaracion/Declaracion}           % Comentar línea si no se usa

\begin{abstracts}        

En los últimos años, el desarrollo científico y tecnológico,  y el crecimiento económico de países como los Estados Unidos,  han hecho del idioma inglés el lenguaje común para la comunicación y para la difusión de información. Esto ha provocado que algunos vocablos del inglés comiencen a surgir en los demás idiomas,  mezclándose con las palabras típicas del idioma, y en ocasiones desplazándolas.  

Está tendencia no ha sido una característica única del inglés, a lo largo del tiempo, diferentes idiomas han aportado y modificado el vocabulario de otras lenguas. En este trabajo se estudia la forma en que los idiomas inglés, francés, alemán, italiano y español, se han relacionado durante el siglo XX a través de las palabras que son comunes entre ellos, llamadas  \textit{palabras migrantes};  y a partir de ellas, se han propuesto dos formas para cuantificar la influencia que un idioma ha tenido en otro. 

También se presenta un estudio estadístico llamado \textit{diversidad de rango}, con el que se cuantifican las distintas palabras migrantes que pueden ocupar lugar en un listado, si estas se ordenan  a partir de su frecuencia.  

Finalmente se justifica la falta de un rigor lingüístico, al eliminar cierta cantidad de palabras migrantes, para volver a obtener la influencia entre idiomas, y medir su similitud con los resultados previos. 



 
\end{abstracts}
%\end{abstractlongs}


% ----------------------------------------------------------------------                   % Comentar línea si no se usa

%%%%%%%%%%%%%%%%%%%%%%%%%%%%%%%%%%%%%%%%%%%%%%%%%%%%%
%                   ÍNDICES                         %
%%%%%%%%%%%%%%%%%%%%%%%%%%%%%%%%%%%%%%%%%%%%%%%%%%%%%
%Esta sección genera el índice
\setcounter{secnumdepth}{3} % organisational level that receives a numbers
\setcounter{tocdepth}{3}    % print table of contents for level 3
\tableofcontents            % Genera el índice 
%: ----------------------- list of figures/tables ------------------------
\listoffigures              % Genera el ínidce de figuras, comentar línea si no se usa
\listoftables               % Genera índice de tablas, comentar línea si no se usa


%%%%%%%%%%%%%%%%%%%%%%%%%%%%%%%%%%%%%%%%%%%%%%%%%%%%%
%                   CONTENIDO                       %
%%%%%%%%%%%%%%%%%%%%%%%%%%%%%%%%%%%%%%%%%%%%%%%%%%%%%
% the main text starts here with the introduction, 1st chapter,...
\mainmatter
\def\baselinestretch{1.5}         % Interlineado de 1.5

% this file is called up by thesis.tex
% content in this file will be fed into the main document
%----------------------- introduction file header -----------------------
%%%%%%%%%%%%%%%%%%%%%%%%%%%%%%%%%%%%%%%%%%%%%%%%%%%%%%%%%%%%%%%%%%%%%%%%%
%  Capítulo 1: Introducción- DEFINIR OBJETIVOS DE LA TESIS              %
%%%%%%%%%%%%%%%%%%%%%%%%%%%%%%%%%%%%%%%%%%%%%%%%%%%%%%%%%%%%%%%%%%%%%%%%%

\chapter{Introducción}




\newpage

\section{Primeros objetivos del análisis}

Una premisa que puede explicar el por qué las palabras se alteran o no al pasar de un idioma a otro, se puede vincular a la adaptabilidad de la palabra por parte de los hablantes de la lengua.  Por ejemplo, la palabra \textit{internet} con un claro origen en el inglés, y año de “invención” alrededor de 1990,  migró y cobró relevancia en otros idiomas por la facilidad y la rapidez con la que las sociedades  aprovecharon el fenómeno de internet, beneficiándose de la revolución que trajo en aspectos como el desarrollo de las telecomunicaciones,  el avance tecnológico,  la sofisticación de aparatos, el cambio positivo en el modo de vivir de las personas, entre otros beneficios.  Tal adaptación por parte de la sociedad hizo cotidiano el concepto, y en consecuencia la palabra.  

Otro palabra que sirve como ejemplo para este razonamiento es \textit{wi-fi}, donde la mayoría de las personas entiende el significado y el concepto que describe la palabra. Probablemente  exista una traducción para describir al fenómeno en cada idioma, sin embargo la cotidianidad que existe entre la escritura original y la comunidad que lo emplea, hace inviable una modificación que resulte en la pérdida de la familiaridad.


Un caso donde un concepto y fenómeno ha logrado un cambio en la forma de vida de las comunidades,  y la palabra que se asocia a él ha sido transformada por los receptores es el del \textit{teléfono}; su escritura original \textit{teletrofono} es proveniente del italiano,  pero su apogeo se dio en el inglés, modificando la escritura a \textit{telephone}; los diferentes idiomas también hicieron una modificación, siendo \textit{téléphone} en francés, \textit{telefon} en alemán, \textit{teléfono} en español, e incluso el italiano adopta la escritura del español   (el n-gram viewer es útil para comprobar esto), a pesar de ser el idioma origen. Pocos cuestionan el cambio que originó el teléfono en la sociedad desde su aparición hasta  tiempos recientes, pero contrario al internet su escritura no prevaleció.  

El tiempo que le toma a las palabras el pasar de un idioma a otro, es un factor que influye en la prevalencia de las palabras. El ejemplo del internet en los años recientes,  donde  la globalización ha permitido que la información pase de un medio a otro con mayor fluidez y el fenómeno que conlleva la palabra sea asimilado por las diferentes comunidades en un tiempo relativamente corto entre el origen de la palabra y su uso en los demás idiomas. 

El encontrar factores o sucesos  que expliquen el por qué las palabras fluyen de un lado a otro, será uno de los objetivos del trabajo. Más objetivos se irán planteando conforme se avance en el texto, para llegar al propósito del análisis, permitiendo establecer una forma de cuantificar la influencia entre los idiomas. Antes de llegar a este punto, habrá que hacer más clasificaciones para un mejor manejo de los datos. 

\newpage

\section{Clasificaciones de las palabras}

La primera clasificación  hecha fue el identificar  el  idioma origen y los diferentes idiomas receptores para las palabras migrantes, para establecer los préstamos de un idioma en otro, tras esta organización de datos, y como primer paso para llegar a la influencia de una lengua sobre otra,  es útil conocer qué tan importantes han sido los préstamos de un mismo origen en los demás idiomas, y también en qué año o años un idioma receptor ha adoptado recibido más préstamos de diferentes orígenes. 

Si se supone un idioma receptor B, un año cualquiera que esté dentro de los datos y  la lista  de las 5 mil palabras más usadas de B para tal año, entonces dentro del registro habrá préstamos con idioma  origen A en alguna posición de rango.  Estos préstamos se han clasificado como:


\begin{description}

	\item [Préstamos Nuevos:] Son palabras que aparecen por primera vez en las más usadas del idioma B.
	
	\item [Préstamos Repetidos:] Palabras que ya habían aparecido en alguna lista del idioma B, y para ese año lo volvieron a hacer.
	
	\item [Préstamos Acumulados:] Conjunto de préstamos nuevos y repetidos.
	
	
\end{description}


Por lo tanto para un determinado año dentro de la lista de las más usadas, existirán palabras que aparecieron en la lista por primera vez y palabras que ya habían estado en la lista en al menos un año anterior.  Asimismo  habrá años en los que no existieron préstamos nuevos,  por lo que todos los préstamos por año serán repetidos.




\newpage
\section{Interpretación de la influencia}

El medir la influencia que un evento tiene sobre otro, no es un proceso que esté mecanizado, como lo puede ser el determinar la distancia entre dos puntos o evaluar el tamaño de un objeto.  No existe un conjunto de reglas que afirmen o refuten si un acontecimiento es influyente en otro. En cada evento existe una cantidad diferente de variantes que intervienen entre ellas para  conducir a una respuesta sobre la influencia. 

En el caso de los idiomas, las variables que se conocen son el tiempo (manifestado en los años de las listas), la frecuencia y el rango de las palabras.  Un dato importante es que la cantidad de libros que fueron digitalizados es mayor en los últimos años que en los primeros,  la frecuencia de la palabra con rango uno en las lista de 1740 puede ser la diezmilésima parte de la del mismo rango en la lista de 2009, inclusive tratándose de la misma palabra,  entonces se tendrán que normalizar los datos para que tengan la misma proporción. 

De nuevo, se piensa en los préstamos de A, y los de C que están presentes en B.  Una manera preliminar de conocer la influencia, es contar cuántos préstamos de cada idioma están en las listas de cada año en B,  y el idioma más influyente será el que tenga más elementos. La idea es buena, pero no suficiente, para demostrarlo, se piensa en la lista de un año en B,  y al contar los préstamos, se obtiene una cantidad X para los que tienen origen A y  una cantidad Y para los de origen C,  con X $>$ Y. 

Al utilizar el criterio anterior,  el idioma A es en principio más influyente que el C por tener más cantidad de elementos en B, pero si los préstamos de C ocupan rangos más pequeños que los de A, como el tener un rango pequeño es similar a tener una frecuencia mayor,  si se suma cada frecuencia  de los préstamos de A y C,  la suma para las de C puede ser mayor que la suma para las de A, en consecuencia las palabras de C son más importantes dentro de B que las palabras de A, ya que son más frecuentadas o son utilizadas más veces.   

El utilizar la frecuencia es otra herramienta con la que se logra sustentar la respuesta a la influencia, puede ser alternativa o complementaria a la basada en la cantidad, en el trabajo se empleara cada una para describir un caso particular, aún falta añadir la normalización,  de manera formal se describen los dos métodos y se recuerda que las listas son de las cinco mil palabras más usadas. 

\begin{description}
	
	\item[Influencia por cantidad:] Se toma como influencia a la cantidad de préstamos de A en la lista de un año en B.
	
	\item[Influencia por frecuencia:] La influencia estará determinada por relacion porcentual entre las frecuencias de los préstamos de A en la lista de B y la frecuencia de todas las palabras de la lista de B. 
	
\end{description}


Para entender mejor  la influencia por frecuencia, se describira a detalle el algoritmo empleado:

\begin{enumerate}
	
	\item En un año determinado del idioma B, se sumarán las frecuencias $f_{t}$ de cada una de las cinco mil palabras más usadas.  Esta cantidad se llamará \textbf{frecuencia  total por año.}
	
	\begin{equation}
	\label{ec.ftot}
	f_{t} = \sum_{i=1}^{5000} f_{i} \,\,\,\,\,\,\,\,\, i = posici\'{o}n \,\,\, de \,\,\,cada \,\,\,palabra
	\end{equation}
	
	\item Como se conocen las posiciones $j$,  que ocupan los préstamos A en la lista de B, se procede a sumar sólo las frecuencias asociadas a estas palabras. Esta cantidad será la \textbf{frecuencia de préstamo} $f_{p}$,  esta cantidad es siempre menor que la frecuencia total por año.
	
	\begin{equation}
	\label{ec.fpres}
	f_{p} = \sum_{j} f_{j} \,\,\,\,\,\,\,\,\, j = posici\'{o}n \,\,\, de \,\,\,cada \,\,\,prestamo
	\end{equation}
	
	
	\item  Se divide la frecuencia de préstamo entre la frecuencia total por año, esta cantidad es la indicada para medir la influencia, se denotará como \textbf{frecuencia de uso} $F$ y es la porción que representan las frecuencias de los préstamos de A en B.  Como la influencia por cantidad se expresa como un porcentaje,  bastará multiplicar el cociente por cien para obtener el porcentaje.  
	
	\begin{equation}
	\label{ec.fuso}
	F = \frac{f_{p}}{f_{t}} * 100
	\end{equation}
	
	
	Entre más cercana a 100 $\%$ sea la frecuencia de uso, los préstamos del idioma A serán más relevantes en B.
	
	
\end{enumerate}


La frecuencia de uso está ya normalizada, siendo el parámetro de normalización la frecuencia total por año. Otra posible normalización sería al considerar la cantidad de libros que fueron registrados para obtener la lista de las palabras más usadas en cada año,  sin embargo se desconoce esta cifra.  


\newpage
\section{La influencia en 109 años (1900-2009)}

Como se tienen dos maneras de medir influencia,  hay casos en los que conviene utilizar solo una forma o ambas.  Para los préstamos nuevos, se decidió emplear el método de influencia por cantidad, ya que mostrara los años de mayor intercambio de palabras que se asientan en el idioma receptor; el método de influencia por frecuencia no proporciona información relevante en estos préstamos ya que el impacto que representan las palabras nuevas en la frecuencia de uso del idioma receptor es muy pequeña, ya que generalmente las palabras nuevas entran a las listas en posiciones muy altas de rango.  

La influencia por frecuencia brinda mayor información si se emplea en los préstamos por año, al evaluar la relevancia de todos los préstamos del idioma origen en el receptor. En este punto es importante el background que se obtuvo del conjunto base, ya que también se contó la influencia de los préstamos acumulados que surgieron en el background, debido a que para el primer año del conjunto búsqueda (1901) ya forman parte del idioma receptor, y tienen un impacto en él. 

De manera resumida se hicieron por el momento dos análisis, el primero obteniendo la influencia por cantidad de los préstamos entre dos idiomas entre 1901 y 2009, y el segundo al contabilizar la influencia por frecuencia de los préstamos acumulados durante los mismos años.  Los resultados de cada prueba se presentarán en las siguientes secciones. 


            % ~10 páginas - Explicar el propósito de la tesis
\chapter{La Base de datos}

\section{La base de datos y su interpretación} % section headings are 

Para la elaboración del trabajo,  se dispuso de la base de datos de los los n-grams de Google Books. Esta base de datos tiene hasta el momento alrededor del 4$\%$ de todas las publicaciones en diferentes idiomas,  y se caracteriza por en listar  por cada año y por cada idioma de publicación los “n-gramas” más utilizados.   Los n-gramas son las palabras o conjunto de palabras que forman el texto de un libro, donde el número n indicará la cantidad de palabras que forman el grama.  Es decir, los  1-grama son palabras individuales (una sola palabra o un signo), los 2-grama son frases compuestas por dos 1-grama, los 3-grama el conjunto de tres 1-grama y así sucesivamente.   Con la herramienta del \textit{n-gram viewer} \cite{ngramv}, los usuarios pueden acceder a una forma visual el comportamiento a lo largo del tiempo de un n-grama en los diferentes idiomas en que se publicó.  

A partir de esta base, se extrajeron los datos de los 1-grama en cinco diferentes idiomas, inglés, francés, alemán, italiano y español, correspondientes a las publicaciones de 269 años (1740-2009).  De acuerdo a \cite{iplosone},  el kernel de un idioma lo componen entre 1500 a 3000 palabras más comunes del mismo,  cantidad que basta para conocer al idioma.  Para el estudio del trabajo se tomaron por cada año y por cada idioma,  las cinco mil palabras más usadas (cantidad que abarca palabras dentro y fuera del kernel), teniendo conjuntos homogéneos en tamaño.

Cada palabra está asociada a una frecuencia, que es la cantidad de veces en que que la  palabra apareció en las publicaciones de un año y un idioma, y a su vez cada frecuencia se vincula a un rango, que es la posición en un ordenamiento.  Cada listado está ordenado de manera descendente en la frecuencia y de manera ascendente en el rango, la palabra más utilizada tendrá mayor frecuencia y le corresponde el rango uno,  la siguiente seŕa la segunda más frecuente correspondiente al rango dos,  la tercera más frecuente con rango tres y así sucesivamente para todos los elementos.  Con ello se afirma que las palabras más usadas tienen una mayor frecuencia y un rango menor. 


\section{Forma de búsqueda}

Se identifican palabras que sean iguales en escritura, carácter por carácter y que estén presentes en dos o más idiomas.  Si se define como  \textbf{Migración}, al movimiento de palabras de un idioma a otro, entonces las palabras \textbf{Palabras Migrantes}  son aquellas que están presentes en al menos dos idiomas.  Conviene en este punto realizar dos definiciones que serán útiles.

\hfill \break

Se define como  \textbf{idioma origen}, al idioma en el cual la palabra apareció por primera vez dentro de la lista de las cinco mil palabras más usadas.  

\hfill \break 

Se define como \textbf{idioma receptor}, a  aquel donde la palabra está presente, siendo un conjunto diferente al idioma origen.  

\hfill \break

Para que existan las palabras migrantes, se necesita un origen y al menos un receptor. Si una palabra está presente en más de dos idiomas, alguno es el origen y los demás son los receptores. 

Una vez que se tiene certeza de cuáles palabras son migrantes, y en cuáles idiomas está presente se procede a determinar el idioma origen y los idiomas receptores.   Si se suponen dos idiomas $\textit{A}$ y $\textit{B}$  donde la palabra se encuentra, y se sabe con seguridad el año de aparición en cada idioma y el rango que ocupó para ese año,  el criterio para determinar el origen es el siguiente: 


\begin{enumerate}
	
	\item  Si la palabra apareció en años anteriores en el idioma $\textit{A}$  (dentro de las 5 mil más usadas) que en el idioma $\textit{B}$ , se establece al idioma $\textit{A}$  como el idioma origen y $\textit{B}$  como el receptor.
	
	\item Si la palabra apareció en el mismo año en ambos idiomas, se establece el idioma origen a aquel donde la palabra tuvo un menor rango, es decir, si el rango en la lista de las más usadas en el idioma $\textit{A}$  es menor que el rango en la lista del idioma $\textit{B}$ , entonces $\textit{A}$  es el idioma origen, en caso contrario, el origen es $\textit{B}$ .
	
	\item Se descartan palabras que contengan un solo carácter, las que son compuestas por letras y números y aquellas que aparecen en el receptor sólo una vez. La intención es tener una  nueva base de datos con el mayor contenido limpio. 
	
\end{enumerate}

Los argumentos anteriores, se pueden ampliar si la palabra está presente en tres o más .

\newpage
La manera de clasificar el idioma origen de las palabras puede carecer de otras pautas para ser más preciso, sin embargo, a lo largo de toda la investigación se optó por utilizar lo más posible los datos de los n-gramas y  con ellos crear reglas para obtener resultados.  Una forma más precisa sería tomando otro tipo de base de datos, con información etimológica de las palabras y las diferentes escrituras que tomó la palabra hasta que prevaleció una estructura.  

Conocidos los idiomas origen y receptor de las palabras migrantes, se llamarán \textbf{Préstamos} a las palabras con un mismo origen y que están  presentes en un receptor.  Un receptor $\textit{B}$, puede tener palabras con diferentes orígenes $\textit{A}$, $\textit{C}$ o $\textit{D}$, también una palabra con origen $\textit{A}$ puede llegar a diferentes receptores.  Los préstamos de $\textit{A}$  en $\textit{B}$  son el conjunto de palabras con origen  $\textit{A}$  y con receptor $\textit{B}$.     


\subsubsection*{Ventajas}


\begin{itemize}
	
	\item [$-$] Determina el idioma donde la palabra fue más popular al comienzo de la base de datos (1740), y hacia donde se esparció el vocablo, dando un carácter histórico de los idiomas mas populares en diferentes épocas. 
	
	\item [$-$] Localiza las palabras que conservaron su escritura al pasar de un idioma a otro. 
	
	\item [$-$] Valora a un idioma como importante e influyente si sus vocablos son transmitidos a los demás y perduran por un periodo de tiempo. También el mismo idioma es influyente si a través de él se han esparcido palabras a los demás, a pesar de que no sea el idioma origen. 
	
\end{itemize}


\subsubsection*{Desventajas}


\begin{itemize}
	
	\item [$-$] Al encontrar palabras con igual escritura, se obtienen casos donde el significado en cada idioma es diferente.  Por ejemplo, la palabra  $\textit{MAYOR}$, que está presente en el inglés y en el español,  significa alcalde en inglés, mientras que en español es más grande que.
	
	\item [$-$] No se localizan palabras que han sufrido transformaciones en la escritura al pasar de un idioma a otro, consecuente de que la palabra se adapta a la gramática de los diferentes receptores.  Por ejemplo, las palabras $\textit{imagine}$ e $\textit{imaginar}$, son similares en los primeros caracteres, teniendo el mismo significado en el inglés y en el español, no obstante la terminación de  sus últimas letras se modificó al estar en  la lengua inglesa y la española. 
	
	\item [$-$] Define un origen distinto al verdadero.  Muchos de estos casos se presentan al no tener una base de datos con más idiomas, y el verdadero origen puede estar en estas exclusiones. Por ejemplo la palabra natural, que proviene del latín y se encuentra en inglés y español con igual escritura,  el programa la identifica proveniente del inglés, más su verdadera procedencia es el latín. 
	
\end{itemize}


Se han revisado las palabras encontradas con el algoritmo descrito, en su mayoría los resultados son aceptables. Los mayores inconvenientes resultaron al clasificar las palabras y las migraciones entre idiomas de la misma familia lingüística.  En los capítulos siguientes se proporcionara un vinculo para poder observar todas las palabras encontradas. 


Ya que  la base de datos es amplia y se busca tener palabras cuyo significado refleja la composición de los idiomas y no a la estructura escrita de los mismos. Las palabras de acuerdo a  \cite{contenidopal}, se clasifican en \textbf{\textit{palabras funcionales,}} aquellas que auxilian a las demás a estructurar un mensaje de acuerdo a la gramática del idioma, y en \textbf{\textit{palabras de contenido}} que llevan la información y significado del mensaje. Para obtener la esencia de las palabras que fluyen entre los idiomas, se han eliminado de las listas a los artículos, pronombres,
preposiciones y conjunciones, correspondientes a las palabras funcionales, quedando solo palabras de contenido. 

Al tratarse de cinco diferentes idiomas, las posibles palabras funcionales se obtuvieron de diferentes diccionarios y páginas \cite{englishdic, frenchdic, germandic, italiandic, spanishdic}.

En secciones posteriores, se realizará un estudio sobre el cómo afecta a los resultados la eliminación de palabras a partir de reglas arbitrarias.  Por el momento todo el trabajo se apoya en la eliminación de las palabras funcionales.


\section{Partición de datos}

Al ser limitada la base de datos por no haber registros (suficiente información) para los cinco idiomas (inglés, francés, alemán, italiano y español) antes de 1740,  existe un periodo de tiempo durante los primeros treinta o cincuenta años, donde se encuentran una mayor cantidad de préstamos de un idioma a otro. Este periodo desfasa a  las migraciones ya que sin importar cual sea el año del comienzo y los idiomas utilizados,  la mayor cantidad de migraciones sigue ocurriendo años después del inicio.   

Para estabilizar el flujo de palabras entre idiomas, se decidió partir la información en dos conjuntos que ayuden al análisis.

\textbf{Conjunto base.} (1740-1900) todos los movimientos encontrados en esta partición se tomarán como verdaderos,  obteniendo un \textit{“background”} de las palabras que migraron de un lado a otro.

\textbf{Conjunto de búsqueda.} (1901-2009) será donde se muestren los resultados de cada técnica elaborada y donde mayor discusión se realizará, apoyados de los resultados del conjunto base. 





% this file is called up by thesis.tex
% content in this file will be fed into the main document
%----------------------- introduction file header -----------------------
%%%%%%%%%%%%%%%%%%%%%%%%%%%%%%%%%%%%%%%%%%%%%%%%%%%%%%%%%%%%%%%%%%%%%%%%%
%  Capítulo 1: Introducción- DEFINIR OBJETIVOS DE LA TESIS              %
%%%%%%%%%%%%%%%%%%%%%%%%%%%%%%%%%%%%%%%%%%%%%%%%%%%%%%%%%%%%%%%%%%%%%%%%%

\chapter{Análisis de nuevas}

Para las gráficas del trabajo se utilizaron diferentes colores para caracterizar las influencias de un idioma, además en cada gráfica se especifica los idiomas que intervienen y para su mejor lectura se utilizaron abreviaciones. Los colores y abreviaciones de cada idioma son:

\hfill\break

inglés   $\rightarrow$  EN $\rightarrow$  azul

francés  $\rightarrow$  FR $\rightarrow$  amarillo

alemán   $\rightarrow$  GE $\rightarrow$  violeta

italiano $\rightarrow$  IT $\rightarrow$  verde

español  $\rightarrow$  SP $\rightarrow$  guinda

\hfill\break

Para las gráficas comparando la influencia entre dos idiomas, la primera abreviación que aparezca en la leyenda será el idioma origen de los préstamos, y la segunda corresponderá al idioma receptor, el color de cada una indicará el idioma origen. Por ejemplo, si se grafica la influencia entre el inglés y el francés,  la leyenda EN-FR seran los préstamos que van del inglés al francés y se graficarán con color azul,  en cambio la leyenda FR-EN son los préstamos que van en sentido contrario, del francés al inglés en color amarillo.  

\hfill\break

En todas las gráficas, el eje horizontal está representado por los años del conjunto de búsqueda 1900-2009,  mientras que en el eje vertical se presenta la medición de la influencia ya sea en cantidad o en frecuencia. 



\newpage

\section{Palabras nuevas entre dos idiomas}

Se estima la influencia por cantidad en los  préstamos nuevos que entraron a un idioma receptor provenientes de un idioma origen; el año de entrada del préstamo es aquel donde apareció en la lista de las cinco mil palabras más usadas del receptor por primera vez. 

La motivación de cuantificar así la influencia es notar si algún idioma ha crecido más en los demás durante el siglo XX y la primera década del siglo XXI, y asociar los crecimientos a eventos políticos, culturales, históricos o sociales. 

Para disposición del lector, se puede consultar la lista de los préstamos nuevos de un idioma a otro, en la referencia \cite{prestamos_nuevos}.  Las instrucciones de cómo se deben interpretar las palabras se encuentran en el apéndice 1. 

Analizar el contenido de las listas, permitirá hacer conclusiones sobre la causa de las migraciones. 



\subsection{Inglés y Francés}


\begin{figure}[h!]
	\centering
	\includegraphics[scale=.38]{Cap_2/NC_1_S2_EN.png}
	\label{NC_EF}
	\caption{}
\end{figure}


La primera conclusión tras ver la gráfica, es que el inglés aportó mayor cantidad de palabras al francés que el caso contrario, los aportes no fueron constantes, sino que existieron períodos de tiempo 1910-1920, 1935-1960, 1965-1975 donde la migración fue más evidente (periodos donde hay picos en las trazas).  En el contexto histórico, en los primeros dos periodos ocurrieron las guerras mundiales, donde intervinieron países de habla inglesa y francesa.   Este argumento puede ser comprobado al consultar la lista de préstamos nuevos del inglés al francés; específicamente para los años 1944 y 1945, entre el contenido de la lista se encuentran \textit{Churchill}, \textit{territories,} \textit{nazis} y \textit{catastrophe} palabras que están relacionadas con la segunda guerra mundial por los años de aparición. Para el último periodo,  el contenido de la lista tiene palabras como \textit{Nixon} \textit{dollar} y \textit{Johnson}; dos de estas palabras aluden a apellidos que fueron importantes en el periodo, posiblemente ya existían estas palabras en el francés, pero hasta los años entre 1965 y 1975 fue que cobraron notoriedad por algún personaje que fue importante en esa época,  específicamente Lyndon B. Johnson y Richard Nixon, presidentes de Estados Unidos entre 1963-1969 y 1969-1974 respectivamente.

Los periodos más prolongados donde existieron aportes del francés al inglés,  ocurrieron entre 1930-1950 y 1975-1990,  sin embargo la lista de los préstamos no muestra palabras que se puedan asociar a un evento,   pero si hay palabras comunes en el inglés y el programa las detecto como provenientes del francés, como lo son \textit{diagnostic,} \textit{clients,} \textit{placement,} \textit{adaptation,} \textit{diffusion,} \textit{amplitude,} entre otras, estas son palabras que el programa clasificó como de origen francés, ya que fue el primer idioma donde tuvieron relevancia,  y que después las retomo el inglés. Esta propuesta puede sugerir que en años anteriores el francés era rico o basto de palabras de otros idiomas, o que los idiomas utilizaban al francés para poder llegar a más conjuntos. En un análisis posterior, se mostrará el uso que tenía el francés en el siglo XIX, comparado con el que tuvo en el siglo XX. 

De manera general para todos los idiomas, existirán casos donde las préstamos de un idioma en otro serán nombres, apellidos, ciudades o países, como el caso que ya se comentó anteriormente. Estas palabras pueden mostrar un evento histórico al estar relacionada con personajes o lugares del evento y también por la fecha en la que aparecieron. 


\newpage


\subsection{Inglés y Alemán}


\begin{figure}[h!]
	\centering
	\includegraphics[scale=.38]{Cap_2/NC_2_S2_EN.png}
	\label{NC_EG}
	\caption{}
\end{figure}


El mayor aporte ocurrió del inglés al alemán, hay muchos periodos donde la cantidad de préstamos sobresalen, el primero entre 1920 y 1935,  el inglés aportó  palabras como \textit{economic} (1929), \textit{depression} (1931),  \textit{investment} (1933), \textit{Roosevelt} (1935) , que pertenecen al campo semántico de la gran depresión, evento que tuvo origen en los Estados Unidos  con consecuencias en muchos países, en este caso en Alemania, ya que fue uno de los motivos para que se originara la segunda guerra mundial.  Durante los años de la segunda guerra las palabras encontradas fueron \textit{Churchill} (1940), \textit{Stalin} (1943), \textit{nazi} (1945),  \textit{justice} (1947),  las cuales si están relacionadas con el suceso.  Finalmente alrededor de 1990, el inglés ha aportado de manera creciente más palabras al alemán, el desarrollo de la tecnología y la globalización son responsables de esta tendencia al estar involucradas palabras como \textit{standards} (1933), \textit{market} (1994),  \textit{internet} (1996), \textit{online} (1988). 

Para el préstamo en sentido inverso, el alemán presentó muchos años donde no llevó palabras de él hacia el inglés,  pero las pocas que lo hicieron muestran información histórica, como \textit{Lenin} (1931), \textit{Marx} y \textit{Hitler} (1934),  \textit{reich} (1939) y  \textit{Mao} (1967),  en alusión a la segunda guerra mundial y a la revolución china que ocurrió durante la guerra fría.

La relación entre el inglés y el alemán en el ámbito de préstamos se vio marcada por palabras que van de un lado a otro con relaciones a un hecho histórico que involucró a países cuyas lenguas son estos idiomas.  Por el momento se está haciendo evidente que los eventos políticos o históricos si permiten que exista una migración de palabras entre idiomas. 

\newpage
\subsection{Inglés e Italiano}

\begin{figure}[h!]
	\centering
	\includegraphics[scale=.38]{Cap_2/NC_3_S2_EN.png}
	\label{NC_EI}
	\caption{}
\end{figure}

El sentido de mayor cantidad de migraciones ocurrió del inglés hacia el italiano, donde es evidente periodos alrededor de las guerras como en el caso de las migraciones del inglés hacia el francés y el alemán,  nuevamente palabras como \textit{Roosevelt} (1941) o \textit{Stalin} (1949) son de carácter histórico; \textit{Mussolini} (1935)  presente en las migraciones del italiano al inglés es otro ejemplo de la relación de la guerra con estos idiomas y en esos años.A pesar de que la gráfica del inglés al italiano se vea favorecida en cantidad, muchas de las palabras no se consideraron préstamos de carácter histórico, ya que son palabras que no portan información que las ligue a un evento, sin embargo se asentaron en el italiano y han sido usadas en al menos dos años dentro de la lista de cinco mil. 

Los préstamos del italiano en el inglés, fueron nulos durante muchos años, los pocos en los que existieron, las palabras no se lograron  ligar a un suceso que explicara el por qué ocurrieron las migraciones,  salvo el ejemplo ya mencionado.

\newpage


\subsection{Inglés y Español}
\begin{figure}[h!]
	\centering
	\includegraphics[scale=.38]{Cap_2/NC_4_S2_EN.png}
	\label{NC_ES}
	\caption{}
\end{figure}

El español ha sido el único idioma que ha aportado en algunas épocas igual cantidad de palabras al inglés como el inglés aportó al español,  en los demás idiomas la tendencia era que el inglés aportaba más que los demás a él.  Las dos gráficas muestran similar crecimiento en periodos de la primera mitad de siglo,  como de 1905 - 1925, y en 1935-1950.  En la segunda mitad del siglo XX,  se presentaron mayor cantidad de préstamos del inglés al español, y el sentido opuesto tuvo muchos periodos donde no aportó alguna palabra,  siendo el más relevante entre 1975 y 1985. 

Entre los préstamos encontrados en el sentido inglés-español, se encuentran \textit{oil} (1931), \textit{standard} (1933),  referentes a la compañía de petróleos de Standart Oil de John D.  Rockefeller,  \textit{unesco} (1955); de estas palabras es curioso que aparecieron en el español tras un letargo de años en el que eran importantes en el inglés,  por ejemplo, la empresa Standard Oil, cobró relevancia por ser el primer gran monopolio entre finales del siglo XIX y su disolución en 1911,  transcurrieron cerca de treinta años para que el español se comenzará a hablar acerca de ello. Un letargo menos extenso ocurrió con la palabra Unesco, esta organización se fundó diez años antes (1945)  de que entrara a la lista de español.  

Conforme se avanza en los años, aparecen palabras como \textit{Kennedy} (1961), \textit{nuclear} (1962) (sin acento al provenir del ingles), \textit{Nixon} (1972),  \textit{Bush} (1990), \textit{internet} (1996) e  \textit{Irak} (2003),  de este conjunto, tres son apellidos de presidentes de los Estados Unidos, pero no hay un desfase prolongado entre la fecha de relevancia en el conjunto original (para este caso los años en que fueron presidentes) y el año donde aparecen en el receptor. Palabras como nuclear e Irak, también son de carácter histórico aludiendo a los conflictos bélicos que se dieron alrededor de su fecha de aparición, igualmente no tuvieron un desfase de años, mientras que internet alude al desarrollo de esta herramienta. 


Por parte de los préstamos que van del español hacia el inglés,  se lograron ligar palabras al campo de la medicina,  en el año de 1943 aparecieron las palabras \textit{virus} y \textit{anemia}, años antes en 1934 George Richards Minot, Parry Murphy y George Hoiyt Whipple, habían recibido el premio nobel de medicina por su descubrimiento de la terapia de hígado en las anemias.  También migraron nombres de países como \textit{Chile} (1920), \textit{Argentina} (1941),  \textit{Venezuela} (1942).

La conclusión general  de estos argumentos es que la “velocidad” con la que se transmiten palabras de un idioma a otro,  ha ido en aumento conforme se avanza en el tiempo, permitiendo que cada vez sea más inmediata, además las palabras ligadas a sucesos históricos no son solo en el ámbito político, sino también al desarrollo científico y tecnológico. 


\hfill\break

\subsection{Francés y Alemán}

\begin{figure}[h!]
	\centering
	\includegraphics[scale=.38]{Cap_2/NC_2_S2_FR.png}
	\label{NC_FG}
	\caption{}
\end{figure}


Las aportaciones entre ambos idiomas han sido equitativas,  teniendo un comportamiento similar en diferentes plazos, el alemán logra su punto más alto entre 1940, mientras que el francés lo hace  en 1945.  Las primeras dos clafisicaciones que se hicieron para los préstamos que van del francés al alemán son a partir de términos políticos como \textit{diplomatie} (1917), \textit{bourgeoisie} (1919),  \textit{guerre} (1925),  \textit{empire} (1937), y por nombres de  países como \textit{Ukraine} (1918),  \textit{Allemagne} y \textit{Russie} (1925) y \textit{Vietnam} (1965).  Dentro de un entorno histórico,  ambas clasificaciones pueden simplificarse en una, ya que para los años de aparición, es posible que en una misma oración o párrafo coexistan palabras de las dos clasificaciones.  La última clasificación hecha para esté sentido de migración son términos del campo semántico del desarrollo tecnológico como \textit{technologie} (1969),  \textit{innovation} (1996), \textit{informations} (1997),  \textit{mobile} y \textit{communication} (2007).  Estas palabras también tienen en común que su origen es el inglés y que aparecieron en la segunda mitad de siglo, posiblemente gracias a la globalización. 

Para la gráfica de las palabras que parten del alemán hacia el francés, el punto más alto se logró alrededor de 1944,  dentro del contenido para esté año se encuentran  \textit{regierung},  \textit{deutschen},\textit{minister} y  \textit{bestimmungen} (traducciones de gobierno, alemán, ministro y reglamentos);   si se añaden palabras en años previos como \textit{Hitler} (1933),  \textit{tirailleurs} (1931),  e incluso \textit{kaiser} (1915) y \textit{reich} (1921), son conceptos que muestran parte de la historia bélica que vivieron los hablantes del alemán, hechos de gran importancia que fueron adoptados por los francoparlantes. 

El objetivo no es solo identificar sucesos de carácter militar en esta dirección de los préstamos, el identificar un nombre o apellido facilita encontrar las relaciones con un ámbito,  además de Hitler se encontralos los siguientes apellidos:  \textit{Nietzsche} (1905),  \textit{Marx} (1923), \textit{Heidegger} (1987),  \textit{Mozart} (1956), \textit{Freud} (1965) y \textit{Engels} (1970), enlazados a la filosofía, la música y la medicina,  además todos ellos de personajes nacidos en países germanohablantes.


\newpage

\subsection{Francés e Italiano}

\begin{figure}[h!]
	\centering
	\includegraphics[scale=.38]{Cap_2/NC_3_S2_FR.png}
	\label{NC_FI}
	\caption{}
\end{figure}

La mayoría de los préstamos que ocurrieron entre estos idiomas provienen del italiano como el idioma origen, logrando puntos contrastantes con las migraciones provenientes del francés. Aunque países como Italia y Francia  fueron relevantes durante los años del análisis,  la mayoría de las palabras encontradas no brindan información para familiarizar a algún hecho,  en las pocas que se lograron conectar del francés al italiano están \textit{Versailles} y \textit{Poincaré} (1924) aludiendo al tratado de paz de la primera guerra mundial y al matemático francés.  También se encontró  \textit{Vietnam} (1966) y \textit{URSS} (1975), aunque no son términos franceses, si fue el francés el primero en hablar de ellos e importarlos a los demás idiomas.

En las migraciones con origen en el italiano, se encuentra \textit{Mussolini} (1935), la cual ya se había mencionado en las migraciones del italiano al inglés, tras revisar las listas de migraciones con origen italiano  a los demás idiomas, Mussolini siempre se encuentra en la lista de préstamos y con el mismo año de aparición en los demás idiomas.  Aunque solo sea una palabra, el estar en todos los idiomas del análisis ejemplifica la importancia de la misma en el siglo XX. 

\newpage
\subsection{Francés y Español}

\begin{figure}[h!]
	\centering
	\includegraphics[scale=.38]{Cap_2/NC_4_S2_FR.png}
	\label{NC_FS}
	\caption{}
\end{figure}

La tendencia de los préstamos del español hacia el francés presentó pocos años donde la cantidad de ellos fue nula, en comparación con los del francés al español donde se aprecian periodos sin una migración.  Por parte de los préstamos con origen en el francés y receptor hay una ausencia de palabras que representen información de la época, la más relevante es \textit{euros}, referente a la moneda de la Unión Europea, ya que el año de aparición en el español y el año en el que empezó a circular de forma oficial coinciden, siendo el 2002.  La palabra \textit{ONU} (1995)  también se detectó en este sentido de migración; a pesar de ser la abreviación de  una organización mundial,  a la cual no se le puede asignar el francés como idioma origen, pero sí fue el francés el primer idioma ( entre los idiomas del trabajo)  en hablar de ella, y aunque la ONU tenga al español como uno de sus seis idiomas oficiales,  dentro de las publicaciones en español, la palabra ONU entró a la lista de las cinco mil más usadas  cincuenta años después de su fundación en 1945. Otras palabras encontradas son \textit{URSS} (1962) y  \textit{Vietnam} (1965),  las cuáles ya se habían mencionado en las anteriores migraciones del francés. 

El primer préstamo que resultó importante del español hacia el francés es \textit{Panamá}\,\,\, (1913), su trascendencia se liga al año de inauguración del canal de Panamá en 1914, siendo la obra una de las más importantes para el comercio de la época al conectar los océanos pacífico y atlántico, además el primer gobierno que impulsó económicamente la construcción del canal fue el francés,  aunque su conclusión y administración pasó a los Estados Unidos.  Las demás palabras incluidas las que migraron entre 1965 y 1985 no se logró identificar su relación con suceso  en el cual hayan sido importantes.

\hfill\break
\subsection{Alemán e Italiano}

\begin{figure}[h!]
	\centering
	\includegraphics[scale=.38]{Cap_2/NC_3_S2_GE.png}
	\label{NC_GI}
	\caption{}
\end{figure}

La tendencia de las migraciones de palabras entre ambos idiomas se ve equitativa entre ambos idiomas,  siendo el punto más destacado alrededor de 1940 por parte del alemán como idioma origen. 

Tras la revisión de los préstamos del alemán hacia el inglés y el francés, se han destacado apellidos de personajes germano hablantes que destacaron en alguna disciplina. El apellido que se asentaron únicamente en el  italiano fue \textit{Berchtold} en 1943,  un año después de la muerte Leopold Berchtold (21 de noviembre de 1942) ministro de exteriores del Imperio Austro-Húngaro de 1912 a 1915, fecha coincidente con el inicio de la primera guerra mundial.  La relación entre la palabra y la fecha con los dos idiomas se puede explicar por las relaciones que tuvieron el imperio Austro-Húngaro con Italia, y además la parte germano parlante del imperio (actualmente Austria) tenía  fronteras con Italia, sumado a que una de las consecuencias de la primera guerra mundial fue la disolución del Imperio.  El argumento de las fronteras físicas entre países con diferentes idiomas oficiales, ejemplifica una de las posibles causas de la migración de palabras entre idiomas, ya que el intercambio cultural  y las relaciones entre ambas naciones pueden ser más activas o dinámicas permitiendo un mayor flujo de términos. 

Los préstamos del Italiano al Alemán permitieron relacionar con contextos bélicos de la época de la segunda guerra mundial a \textit{regime} (1938), \textit{panzer} (1941), \textit{duce} (1942),  traducciones de régimen, blindado y líder, además de \textit{Mussolini} (1935).  Uno de los errores que se encontró fue la palabra \textit{Franco} (1951), que por la fecha hace referencia al dictador español Francisco Franco; al ser un apellido, se registró como origen Italiano, pero mostrando como un término que es popular en un idioma diferente al origen, emigra a los demás idiomas  por un conjunto auxiliar.


\subsection{Alemán y Español}

\begin{figure}[h!]
	\centering
	\includegraphics[scale=.38]{Cap_2/NC_4_S2_GE.png}
	\label{NC_GS}
	\caption{}
\end{figure}

La diferencia más marcada en las gráficas corresponde a las migraciones del español hacia el alemán entre 1930 y 1945, fechas que coinciden con el desarrollo de la segunda guerra mundial, sin embargo no se encontraron palabras que se relacionen con el evento.  La investigación hecha para ligar las palabras del español hacia el alemán con sucesos, muestra que la palabra \textit{lepra} (1901) fue globalmente importante a partir de 1874,  ya que en ese año el científico noruego Gerhard Armauer Hansen descubrió el bacilo de Hansen Mycobacterium Leprae \cite{lepra} que origina la enfermedad,  por el carácter médico de la palabra es probable que se hiciera más investigación sobre la enfermedad en diferentes idiomas, en este caso el alemán,  tras el descubrimiento del bacilo.  Este es otro caso de una palabra que se origina en un idioma (español), pierde  relevancia al transcurrir los años y tras un suceso (el descubrimiento del bacilo)  retoma importancia y migra a los demás idiomas (alemán).   Entre las otras palabras que pasaron al alemán están \textit{virus} (1939) que ya se ha tratado en los préstamos del español al inglés, \textit{China} (1955) referente al país y su relevancia en la segunda mitad del siglo XX por la revolución China de 1949, e \textit{India} (1970) igualmente se conecta con el país ya que en 1965 ocurrió la guerra indo-pakistaní,  también en 1968 hubo una extensa cobertura mediática a la banda británica The Beatles que visitó el norte de la India para meditar.

Las palabras que van en el sentido alemán-español,  presentaron muchos años sin una migración, el incremento de ellos se dio posterior a 1950  donde los años sin intercambio disminuyeron. En el contenido de la lista se encuentran palabras que ya se han tratado como \textit{Marx} (1932), \textit{kaiser} (1938), \textit{Hitler} (1940), \textit{Lenin} (1970), \textit{Hegel} (1971),  \textit{Nietzsche} (2000) y \textit{Freud} (2002).  Es peculiar que las dos últimas hayan aparecido en el español muchos años después que en los demás idiomas, por ejemplo en el francés Nietzsche apareció 1905 y Freud en 1965. El letargo de años puede ser una característica de el tiempo que les lleva  a las  palabras de un idioma  pasar hacia el español,  sobretodo si son palabras con un origen antiguo.

\hfill\break
\subsection{Italiano y Español}

\begin{figure}[h!]
	\centering
	\includegraphics[scale=.38]{Cap_2/NC_4_S2_IT.png}
	\label{NC_IS}
	\caption{}
\end{figure}

Entre estos idiomas se dio la mayor cantidad de intercambios de palabras durante el siglo XX, durante la primera mitad, las aportaciones fueron equitativas con dos periodos donde dominó el italiano y dos donde dominó el español; a partir de 1960 el dominio fue italiano, ya que los aportes del español  fueron nulos hasta 1900. 

Por parte de los préstamos del italiano al español, se encuentran muchos con tendencias hacia ideologías políticas como \textit{socialista} (1914), \textit{comunista} (1932), \textit{capitalismo} (1935), \textit{fascismo} (1937),  \textit{marxismo} (1963) y \textit{terrorismo} (1986), palabras que por las fechas fueron importantes no solo en el italiano y el español, sino globalmente, ya que estas ideologías repercutieron en eventos como la guerra fría y la economía.  Una palabra que no se había tratado es realismo (1948) que tuvo un impulso por la corriente literaria del realismo mágico originado en varios países de Latinoamérica, cuyas obras trascendieron en la literatura. 

En las palabras que se originan en el español y llegan al italiano, se encontró \textit{idealismo} (1920) otra ideología política,  y palabras de términos médicos como \textit{virus} (1922), \textit{colesterina} (1928),  \textit{sintomatología} (1931), \textit{anestesia} (1932), \textit{vitamina} (1935), \textit{anemia}, \textit{metabolismo} y \textit{gástrica} en 1936  y \textit{endovenosa} (1937).  Este grupo de términos apareció en un periodo de veinte años, probablemente por la publicación de un texto que se popularizó, ya que muchas de estas palabras tienen su traducción al italiano. En los demás años, no se encontraron más palabras con un tema común al cual relacionar. 

\hfill\break
\hfill\break
\hfill\break
\section{Conclusiones del método}

El método de buscar préstamos nuevos mostró el diverso contenido de los préstamos que se lograron asociar un hecho,  por parte del inglés, las mayores aportaciones fueron apellidos de los presidentes de los Estados Unidos, y por el papel que tuvo este país en diferentes campos como la economía, la ideología política, el desarrollo industrial, la globalización y su participación en las grandes guerras, logró que sus jefes de estado fuesen relevantes en cada ámbito que se suscitó en la época que gobernaron.  Por parte del inglés como idioma receptor,  las palabras que emigraron al inglés, no exhibieron una tendencia clara sobre el campo al cual relacionarlas,  muchas de ellas son palabras con origen etimológico en el inglés pero que fueron más relevantes  en otros idiomas, antes de tomarlas como un error, estas palabras muestran que antes del siglo XX el inglés no tenía la importancia que actualmente posee, y que se apoyaba en otros idiomas para que sus vocablos fuesen conocidos. 

El Alemán fue otro conjunto donde los préstamos que llegan a otros idiomas se caracterizan por ser apellidos de personajes, a diferencia del inglés, los campos donde destacaron los germanoparlantes fueron ciencias, filosofía, medicina y  por la historia de alemania en las guerras mundiales,  palabras de carácter bélico también migraron a los distintos receptores.

El francés, aportó endónimos a los otros idiomas, principalmente en las primeras décadas del siglo, reflejando la importancia que tuvo este idioma en el siglo XIX y principios del XX, al migrar palabras que tienen su propio significado y escritura en los demás idiomas.  Además de endónimos,  una porción significativa de los aportes son palabras etimológicas del inglés,  este punto ya fue discutido, pero es otra muestra de la relación que ha mantenido el inglés y el francés  desde antes de 1800, al tener el inglés muchas palabras provenientes de la familia del francés, es decir las lenguas grecolatinas. 

El italiano migró términos políticos a los demás idiomas, principalmente al español,  mientras que el español fue diverso en el contenido de sus préstamos,  desde expresiones médicas, nombres de países y  literatura. La diversidad del español se puede deber a la cantidad de países donde es un idioma oficial,  y hay muchos elementos de diferentes culturas que se exportan a los demás idiomas. 

Tanto el francés, el alemán, el italiano y el español,  se comportaron más como idiomas receptores que como portadores,  donde la mayoría de palabras provienen del inglés.  Entonces en términos de cantidad, el inglés ha sido el idioma que más ha influenciado a los demás, por los vocablos nuevos que de él han llegado.  


En términos de préstamos nuevos,  la influencia por cantidad mostró información histórica de los países en el siglo pasado,  pero no el cómo afectan las nuevas palabras a los idiomas receptores. Si los préstamos nuevos cobran relevancia en el idioma receptor, el rango que ocupan disminuye en cada lista tras el año en que se localizaron,  sin embargo se encontraron años donde no hay préstamos nuevos, y al ser pocos en comparación a las cinco mil palabras, su frecuencia de uso, ecuación \ref{ec.fuso} es casi nula, por ello la mejor forma de utilizar a la frecuencia de uso es usando el background de préstamos acumulados. 
















	

%\include{Capitulo2/marco_teorico}           % ~20 páginas - Poner un contexto a la tesis, hacer referencia a trabajos actuales en el tema
\chapter{Palabras acumuladas}
% Intro {{{
La búsqueda para cuantificar la influencia, ha llevado a contabilizar las
palabras que son nuevas en los distintos receptores y a partir de ellas ligar
contextos que sustenten la aparición de palabras.  Se ha puesto énfasis en el
conjunto de búsqueda, pero  el conjunto base  también tiene información sobre
las palabras que han migrado, además  abarca más años (160 comprendidos entre
1740 y 1900), por lo que su información es más basta en contenido. 

Para no repetir el proceso de contabilizar a los préstamos nuevos,  se propone
pensar que en el primer año del conjunto de búsqueda (1900),  el idioma
receptor ya contenía cierta cantidad de palabras que provenían de otros
orígenes\cpnote{A que te refieres con orígenes?},  de tal manera que ya forman parte de él, es decir estos préstamos
``conviven'' con las palabras propias de el receptor y son empleadas
indistintamente\cpnote{No entiendo lo que quieres decir en esta frase. 
Sugiero que la redactes mejor o me la expliques en persona.}. Así el conjunto
base proporcionará un sostén de aquellas palabras que han permeado en un idioma
y son utilizadas en los años del conjunto de búsqueda,  cabe decir que este
sostén crecerá conforme se localicen nuevas palabras\cpnote{acá tampoco entiendo}. 

Es necesario hacer una nueva definición para estos préstamos, dados un idioma  origen  \textit{A} y la lista para un año  de las palabras más comunes en el receptor \textit{B}, se definen como: 

\begin{description}
	\item[préstamos acumulados:] Son las palabras con origen \textit{A} que ya habían aparecido en alguna lista de \textit{B}, y para ese año lo volvieron a hacer.  \cpnote{Es decir que volvieron a aparecer, o que siguen estando? no me queda claro. En la frase despues, lo aclaras, pero siento que la definicion está deficiente. Igual porfa no quites la siguiente frase.}
\end{description}
\cpnote{No entiendo porque cambias el formato. Sugiero seguir con el parrafo como va. De hecho veo que en el capitulo 3, al principio tienes un formato similar. Cambia ambos e integralos en el texto. En la seccion 2.2 me gusta como tienes esas primeras 4 definiciones. Las
primeras dos, integradas en una frase, y las siguientes dos, como parte de una frase dentro de items. }

La diferencia entre los nuevos y los acumulados es que un préstamo será nuevo
sólo en el año de aparición, posteriormente se convertirá en acumulado. El
objetivo  de trabajar con los acumulados es ver cómo se comportan las palabras
que ya han migrado a un receptor y si hay tendencias donde su empleo se vea
alterado\cpnote{Que quieres decir con esa ultima frase? Que comportamiento? No se
que podemos decir con los datos que tenemos}.  

En el capitulo anterior, la atención se enfocaba en la cantidad de palabras,
nunca se trato con su frecuencia o su rango, ahora se utilizarán estas
propiedades  para llegar a una cantidad que cuantifique la influencia. Si se
tienen la lista de las cinco mil palabras más usadas  de \textit{B}, y se
distinguen en ella los préstamos acumulados con origen \textit{A}, entonces:
\cpnote{Oye, no me queda claro que estas usando el doble punto de manera
correcta. Puedes porfa verificar en un manual de ortografía ygramática que lo
estas haciendo bien?}


\begin{enumerate}
	
	\item En un año determinado del idioma \textit{B}, se sumarán las
	frecuencias $f_{k}$ de las cinco mil palabras más usadas.  Esta
	cantidad se llamará \textbf{frecuencia total} $f_{t}$ y es distinta año
	con año. 
	\begin{equation}
	\label{ec.ftot}
	f_{t} = \sum_{k=1}^{5000} f_{k} \,\,\,\,\,\,\,\,\, k = rango\,\, de \,\,cada \,\,palabra
	\end{equation}
	\cpnote{el texto en una ecuaccion se pone de otra manera. Además, no
	tienes necesitad de ponerlo ahi, mejor en lo que sigue. Ademas, tu
	notacion esta chafa. Si tenemos $f_5$ es a tiempo 5 o para el rango 5?
	Dale una iterada a todo este capitulo. Veo que 
	no esta tan pulida como otros capitulos. }
	\item Como se conocen los rangos $j$,  que ocupan los préstamos \textit{A} en la lista de \textit{B}, se procede a sumar sólo las frecuencias asociadas a estas palabras. Esta cantidad será la \textbf{frecuencia de préstamo} $f_{p}$,  siempre será menor que la frecuencia total.
	
	\begin{equation}
	\label{ec.fpres}
	f_{p} = \sum_{j} f_{j} \,\,\,\,\,\,\,\,\, j = rango\,\, de \,\,cada \,\,pr\acute{e}stamo\,\,acumulado
	\end{equation}
	
	
	\item  Se divide la frecuencia de préstamo entre la frecuencia total , esta cantidad se llamará  \textbf{Uso} $U$ y es la porción que representa \textit{A} en \textit{B} en teŕminos de frecuencia.  Como en un año hay más palabras propias de B, esta cantidad es muy pequeña, para tener cifras manejables, se tomara un porcentaje al multiplicar el cociente por cien.  

	\begin{equation}
	\label{ec.fuso}
	 U = \frac{f_{p}}{f_{t}} * 100
	\end{equation}
	
	
	Entre más cercana a 100$\%$ sea el \textit{Uso de A en B}, los préstamos de \textit{A} son más relevantes en \textit{B}.

\end{enumerate}

Lo relevante de trabajar con esta cantidad es que en una lista de un determinado receptor existen préstamos acumulados con distintos orígenes, cada uno tendrá un valor diferente de uso, con ellos se puede inferir el origen que ha sido más relevante para el receptor. 


% }}}
\section{La influencia en 109 años} % {{{

Descritas el tipo de palabras a emplear y la forma de trabajar con ellas, el proceso que se siguió para obtener resultados es el siguiente: 

\begin{itemize}
	
	\item Elegidos un  origen \textit{A} y un receptor \textit{B}, se localizaron los prestamos acumulados de \textit{A} en \textit{B}.
	
	\item Se empleó la ecuación \ref{ec.fuso} en todos los años del conjunto de búsqueda, obteniendo 109 valores. 
	
	\item El proceso se repitió para todas las combinaciones de orígenes y receptores.
	
	\item  Tras cada año del conjunto de búsqueda y por cada pareja de origen y receptor, se elaboraron  listas con los préstamos acumulados, ordenándolos de forma descendente a partir de su frecuencia. 
	
	\item Para observar los datos como una cantidad que varia en el tiempo, se  hicieron tres tipos de graficas con tres tipos de agrupaciones.
	
	\begin{description}
		
		\item[\textit{A} como origen común.] Graficando el uso de \textit{A} en todos los demás.
		
		\item[\textit{A} como receptor común.] Graficando el uso de los demás en \textit{A}.
		
		\item[Alternando \textit{A} y \textit{B}.]  Graficando de manera conjunta el uso de \textit{A} en \textit{B}, y el de \textit{B} en \textit{A}.
				
	\end{description}	

\end{itemize}

Las listas elaboradas se emplearán en los siguientes capítulos, en el apéndice 1 se explica la forma de leerlas así como un vinculo para su consulta. 

\subsection*{Presentación de resultados} % {{{

Por cada idioma se presentarán dos graficas, la primera será tomando al idioma como origen y la segunda al tomarlo como receptor. Se seguirá utilizando la nomenclatura descrita en el capítulo anterior sobre las abreviaciones y los colores.  Además se provee información de los campos semánticos comunes que hacen posible la prevalencia de los préstamos. 

En el apéndice A se agregarán las graficas de uso entre dos idiomas, estas servirán para complementar los resultados expuestos en esta sección.


Adicionalmente, la tabla \ref{tab.cantidad_acumulados} muestra la cantidad promedio de préstamos acumulados encontrados en cada año del conjunto de búsqueda. La idea de la tabla y el método del uso, es observar que el idioma que más palabras aporta a un receptor no es siempre el más utilizado,  el uso es mayor si las préstamos tienen rangos mas bajos (frecuencias altas). 


\begin{table}
	\centering
	\begin{tabular}{lcccccc}
		\multicolumn{7}{c}{R E C E P T O R}                                                                                                                                             \\
		\multirow{6}{*}{\begin{tabular}[c]{@{}l@{}}O\\ R\\ \,I\\ G\\ E\\ N\end{tabular}} &             & \textbf{EN} & \textbf{FR} & \textbf{GE} & \textbf{IT} & \textbf{SP} \\
		& \textbf{EN} & -           & 324.43      & 164.33      & 77.5        & 73.61       \\
		& \textbf{FR} & 297.36      & -           & 94.06       & 118.55      & 66.31       \\
		& \textbf{GE} & 63.87       & 48.06       & -           & 34.92       & 16.61       \\
		& \textbf{IT} & 77.82       & 100.62      & 47.9        & -           & 219.45      \\
		& \textbf{SP} & 118.43      & 84.22       & 29.85       & 311.97      & -          
	\end{tabular}
	\caption{Promedio de préstamos acumulados entre idiomas. Se aprecian dos relaciones reciprocas entre el inglés con el francés y el español con el italiano, donde no importa cual actué como receptor, el otro idioma es el origen del que provienen la mayor cantidad de palabras.}
	\label{tab.cantidad_acumulados}
\end{table}




% }}}
\subsection{Inglés} % {{{


\begin{figure}[h!]
	\centering
	\includegraphics[scale=.36]{PF1_S2_EN.png}
	\label{fig.ST_a_EN}
	\caption{El inglés en los demás. El francés es el idioma donde más se ha empleado inglés, sin embargo en el español ha sido el de mayor crecimiento comparado en su uso en principios y en final de siglo.}
\end{figure} 



\begin{figure}[h!]
	\centering
	\includegraphics[scale=.36]{PF2_S2_EN.png}
	\label{fig.ST_b_EN}
	\caption{Los demás en el inglés. En los últimos 50 años, el español ha sido el idioma que es más utilizado en el ingles, seguido del francés, debido a las relaciones comerciales entre países de ambas lenguas.  Tras la segunda guerra mundial, alemán e italiano decayeron consistente en ser lengua de los países vencidos. }
\end{figure} 



El uso del ingles en los demás ha visto un continuo incremento posterior a 1945 en francés, en italiano y en español, mientras que en  alemán se da posterior a 1990, año donde culmina la guerra fría y se da la finalización del socialismo en Europa con la re-unificación de Alemania. El significado común de los   préstamos acumulados que aparecen en los cuatro conjuntos y que con los años descienden en rango son términos económicos y referentes a la industria como  \textit{capital}, \textit{dollar}, \textit{invesment}, \textit{relations}, \textit{market}, \textit{company}, \textit{development}, \textit{financial},  \textit{institutions}, \textit{internet} y \textit{software}. Otra característica relevante es la aparición continua de los apellidos de los presidentes de los Estados Unidos (posteriores a la guerra) durante el periodo en el cual gobernaron.  Apoyado de la información de los préstamos nuevos, se puede confirmar que el inglés se ha beneficiado del crecimiento de los Estados Unidos para ser exportado a las demás lenguas y ser el idioma común para transmitir información.   


En los últimos cincuenta años, los idiomas mas comunes en el ingles han sido el español y el francés,  nombres de países latinoamericanos como \textit{México}, \textit{Cuba}, \textit{Chile}, \textit{Nicaragua} y \textit{Argentina}, caracterizan a los acumulados del español, mientras que en el francés en su mayoría son palabras que bien podrian catalogarse de origen inglés, entre ellas \textit{royals}, \textit{religion}, \textit{saint}, \textit{passage} o \textit{court}. Tras brevemente ver ambos conjuntos se infiere que el español ha logrado instaurarse en el inglés por la relevancia de estos países en las relaciones o conflictos que tuvieron en el siglo pasado y donde intervinieron países de habla inglesa, contrario  al francés que prevalece por las relaciones culturales y etimológicas que existen entre ambas lenguas.

Por parte de las palabras con origen alemán e italiano no se logró relacionarlas a un campo semántico común. 


% }}}
\subsection{Francés} % {{{

\begin{figure}[h!]
	\centering
	\includegraphics[scale=.36]{PF1_S2_FR.png}
	\label{fig.ST_a_FR}
	\caption{El francés en los demás idiomas. El italiano empleó más al francés durante todo el siglo XX, caracterizado por palabras comunes en la industria vitivinícola.}
\end{figure}


\begin{figure}[h!]
	\centering
	\includegraphics[scale=.36]{PF2_S2_FR.png}
	\label{fig.ST_b_FR}
	\caption{Los demás en el francés. El español ha resultado el de mayor presencia en el francés, en su mayoria son palabras con etimologías grecolatinas, comunes para ambas lenguas al provenir de la misma familia lingüística.}
\end{figure}
		


A pesar de que el idioma que más préstamos toma del francés es el inglés,  el idioma que más utiliza el francés ha sido el italiano,  aspecto que se mantuvo durante todo el siglo del análisis,  la industria vitivinícola, surgen como un conectores entre ambas lenguas al estar presente términos como  \textit{raisins}, \textit{vin}, \textit{vignoble} y \textit{recolte},  siendo una actividad común entre Francia e Italia.  Los préstamos hacia los demás idiomas son de carácter religioso o politico, destacando que tuvo el francés en estos ámbitos, a pesar de que la búsqueda se centre en el siglo XX, las mayores migraciones del francés surgen a partir de 1800, posteriores a la revolución francesa; entre las palabras que se han mantenido desde este acontecimiento están  \textit{saint}, \textit{eglise}, \textit{dime}, \textit{reine}, \textit{guerre}, \textit{imperiale}, \textit{royals} o \textit{bourgeois}.  


Para los prestamos usados en el francés el español y el inglés se muestran como los idiomas con mayor presencia, la característica común de los vocablos de ambos idiomas es que son palabras con etimología grecolatina, 
\textit{depression}, \textit{canal}, \textit{proceso}, \textit{services}, \textit{justice} entre otras,  siendo razonable la aparición de estas palabras por tener las tres lenguas una composición grecolatina. 


% }}}
\subsection{Alemán} % {{{

\begin{figure}[h!]
	\centering
	\includegraphics[scale=.36]{PF1_S2_GE.png}
	\label{fig.ST_a_GE}
	\caption{El alemán en los demás. La familiaridad por provenir de la misma rama lingüística hace posible que el inglés sea el idioma  donde los préstamos del alemán sean continuamente utilizados, siendo evidente la diferencia con las lenguas romances.}

\end{figure}


\begin{figure}[h!]
	\centering
	\includegraphics[scale=.36]{PF2_S2_GE.png}
	\label{fig.ST_b_GE}
	\caption{Los demás en el alemán. Cada lengua ha tenido un periodo de  crecimiento posterior a 1950, mostrando al alemán como un idioma susceptible (al menos en la literatura) donde los demás idiomas han impactado y perdurado.}
\end{figure}




La característica principal de los términos en alemán  en los demás idiomas  son principalmente personajes germano-parlantes que sobresalieron en algún ámbito; siendo además los encontrados en los préstamos nuevos, \textit{Hitler}, \textit{Marx}, \textit{Einstein}, \textit{Freud}, \textit{Engels}, \textit{Heidegger}, \textit{Mozart}, \textit{Hegel} y  \textit{Nietzsche}. 

Como idioma receptor, el alemán adoptó palabras de diferentes campos, tecnológicos y de desarrollo por parte del inglés,  religiosos  por el francés, históricos en el italiano y médicos por el español.  Cada idioma presentó un periodo de crecimiento posterior a 1950  o a la segunda guerra mundial, donde al ser vencido alemania, el idioma tuvo que adaptarse a las tendencias donde los demás destacaban. 

% }}}
\subsection{Italiano} % {{{

\begin{figure}
	\centering
	\includegraphics[scale=.36]{PF1_S2_IT.png}
	\label{fig.ST_a_IT}
	\caption{El italiano en los demás. A pesar de ser fonéticamente similares y de provenir de la familia de las lenguas romances,  el español persiste en ser el idioma con menor uso de italiano. La migración italiana a los Estados Unidos posterior a la primera guerra mundial coincide con el aumento en el inglés. }
\end{figure}
		
\begin{figure}[h!]
	\centering
	\includegraphics[scale=.36]{PF2_S2_IT.png}
	\label{fig.ST_b_IT}
	\caption{Los demás en el italiano. La proximidad geográfica  entre italia con países cuya lengua es el francés y el alemán ayudó a incrementar el uso de estos en el italiano. A pesar de que el inglés se difundió como un idioma universal para la comunicación, su uso en el italiano no ha mostrado un incremento considerable en todo el siglo XX. }
\end{figure}



El uso del italiano se vio caracterizado en la segunda mitad del siglo por la constante aparición y ascenso en rango de \textit{Mussolini}, siendo el personaje más relevante en el siglo pasado cuya lengua materna es el italiano.  Salvo por Mussolini, los demás prestamos italianos presentes en las otras lenguas no se lograron asociar a un único campo, sin embargo esto muestra la diversidad de temas en los cuales el italiano fue relevante, como términos políticos \textit{sociale}, \textit{liberale}, religiosos \textit{santo}, \textit{suora}, \textit{cattedrale} o  bélicos \textit{battaglia}, \textit{regime}.

El sentido inverso al considerar el tipo de palabras que utiliza el italiano de los demás idiomas, es igualmente variado por parte del ingles son conceptos ligados a la tecnología, por el francés a la industria vitivinícola, con el alemán a personajes relevantes de esta lengua,  términos que ya se han mencionado. Finalmente entre la variedad de palabras que toma del español se encuentran nombres de países o ciudades hispanohablantes \textit{México}, \textit{Chile}, \textit{Argentina}, \textit{Montevideo} o \textit{Peru}. 

% }}}
\subsection{Español} % {{{

\begin{figure}[h!] % {{{
	\centering
	\includegraphics[scale=.36]{PF1_S2_SP.png}
	\label{fig.ST_a_SP}
	\caption{El español en los demás. Los idiomas que más emplean español son aquellos con los que comparte una relación etimológica, francés e italiano por ser lenguas romances y con el ingles al tener este idioma una base de palabras grecolatinas.     }
\end{figure}


		
\begin{figure}[h!] % {{{
	\centering
	\includegraphics[scale=.36]{PF2_S2_SP.png}
	\label{fig.ST_b_SP}
	\caption{Los demás en el español.  Durante el último siglo, el contenido francés e inglés ha aumentado en el español siendo su uso equiparable, el surgimiento del inglés como el idioma universal y la relación etimológica con el francés  hacen posible los incrementos.}
\end{figure}



Entre los gráficos del apéndice 1, referentes al uso entre el español y un determinado idioma, se observo que el uso de los del español en los demás idiomas es mayor que el uso de los otros en él.  Ya se han comentado las características principales de las palabras interfieren en el uso, destacando el ámbito de la medicina, \textit{terapia}, \textit{lepra}, \textit{tumor}, \textit{syphilis}, \textit{virus} o \textit{renal}. Con ellos se infiere la productividad y la importancia de la medicina en países de lengua española antes de 1900; incluso en lenguas como el alemán se encontraron este tipo de préstamos. 


En las discusiones anteriores, se ha mencionado el tipo de palabras que los idiomas aportan,  en el caso de las contribuciones al español,   estas siguen la misma tendencia; del ingles los préstamos son de carácter tecnológico y del desarrollo industrial, del alemán son apellidos de personajes destacados en un campo especifico, mientras que del francés y el italiano son de condición religiosa. 

% }}}
% }}}
\section{Comentarios y complementos del método} % {{{


El determinar la influencia entre idiomas a través del uso de los préstamos, ha mostrado primeramente que el idioma que mas cantidad de palabras tiene en otro no siempre es el más utilizado,  radicando el mayor uso en aquel idioma cuyas préstamos tengan menores rangos en la lista de un receptor. 

En todo el siglo XX y la primer década del XXI, el inglés y el alemán han sido los idiomas más cambiantes en los papeles de origen y receptor respectivamente.
El inglés al ser el que más creció en tres idiomas (francés, alemán y español), complementando los resultados del capitulo anterior, al ser el idioma que más palabras nuevas exportó.  El alemán como el receptor donde los diferentes orígenes aumentaron su usó tras la segunda guerra mundial; el uso ha sido semejante a los préstamos nuevos, ha sido el receptor que más recibió. 

Ambos análisis se complementan,  el idioma más influyente ha aportado más palabras nuevas y aquellas que se van acumulando resultan las de mayor incremento en el uso. El idioma más influenciado recibió la mayor cantidad de palabras nuevas y el uso que han tenido los demás ha sido también el del mayor incremento. 

Por el momento sólo es posible describir que originó las variaciones en el uso o en la cantidad de nuevas palabras, no es posible predecir como se comportaran los idiomas en el futuro, ya que la principal característica que  hace fluir a las palabras entre idiomas han sido los eventos, reflejado en que las palabras de su campo semántico  se muevan a diferentes idiomas y continúen apareciendo o desapareciendo tras el suceso. 

Una mejor información de como los eventos alteran a los idiomas se podría extraer si se compararán las características de los prestamos con  datos de los países de alguna habla como lo pueden ser  el crecimiento economizo, el producto interno bruto, la alfabetización, la mortalidad, las migraciones de personas, entre otros.





% }}}

      % ~20 páginas - Explicar el problema en específico que se va a resolver, la metodología y experimentos/métodos utilizados
\chapter{Diversidad de Rango}

Los primeros análisis se enfocaron en buscar hechos relevantes que propiciaron las migraciones de palabras,  siendo oportuna la información histórica.  El capítulo anterior busco ver a cada idioma como un conjunto “universal” donde las propiedades que lo caracterizan  son comunes en cada uno y siguen siendo válidas a pesar de reducir los elementos que los componen al extraer palabras. 

En esta sección se buscará entender cómo son las variaciones de palabras a  lo largo del tiempo, para ello  se enfocara el estudio a cuantificar que tan diferentes son las listas de los préstamos de un idioma en otro,  ya que estas listas están ordenadas por rango, (donde el rango más bajo es la palabra más común en ese año y la de rango más alto la menos utilizada)  una misma palabra puede ocupar distintos cargos en diferentes años,  o  para un mismo rango existe una diversidad de palabras que lo ocupan, esta idea es más adecuada ya hay palabras que no  aparecen en todos los años del análisis,  sin embargo en todos las listas  hay palabras hasta cierta posición (rango).   

La propuesta de la diversidad de rango ha utilizada en idiomas y en deportes [XXXX], siendo útil para mostrar características comunes de los conjuntos donde es medida.  El algoritmo para llegar a la diversidad se propuso en [XXX], y se describe de la siguiente manera:


\begin{enumerate}
	
	\item Se fijan un año inicial $t_{o}$ y uno final $t_{o}$, construyendo un intervalo de años a evaluar $\Delta\,t = t_{f}- t_{o}$.
	
	\item Se toma el primer rango de todos los años en el intervalo y se cuenta el número de palabras que son distintas en ese rango. Esta cantidad será la diversidad para el rango uno.
	
	\item Se prosigue con el segundo rango y se vuelve a contar cuántas palabras son diferentes en todo el periodo de tiempo.  Con ello se obtiene la diversidad para el rango dos. 
	
	\item Ya que las listas de préstamos de un idioma en otro no son homogéneas en cantidad, el procedimiento anterior se repetirá hasta el rango mínimo que poseen todas las listas,  así se asegura tener una homogeneidad en el tamaño.
	
	\item Se normalizan  los valores dividiendo cada resultado entre el número de años comprendidos del intervalo $\Delta\,t$, obteniendo  la diversidad de rango $d(k)$.
	
	
\end{enumerate}


Antes de mostrar los resultados, se espera  que los valores de $d(k)$ sean cercanos a cero cuando  para un rango $k$, las cantidad de palabras que ocupan ese rango sea menor. En caso de que la diversidad sea cercana a uno,  significa que hay una mayor cantidad de palabras que ocupan el rango $k$. 


Tras graficar el rango contra la diversidad, se observó que en todas las combinaciones (a pesar de que algunas tuvieran más datos)  la tendencia de la diversidad  se asemeja a una función de distribución cumulativa logarítmica  normal, la cual depende del rango $k$, y la desviación estándar $\sigma$.

\begin{equation}
	\label{ec.cumulativa}
	F(k) = \Phi \left ( \frac{ln(k)}{\sigma} \right )\,\,\,\,k\geq 0; \sigma \geq 0
\end{equation}

Donde $\Phi$ es la función cumulativa de la distribución normal, que ademas del rango y la desviación estándar, depende del promedio $\mu$.

\begin{equation}
	\label{ec.distribucionnormal}
	\Phi(t) = \frac{1}{\sigma\sqrt{2\pi}} \int_{-\infty}^{t}  e^{ \frac{ - \left ( x-\mu \right )^{2}}{2\sigma^2}  } dx	
\end{equation}


\newpage
\subsubsection*{Ajuste de Datos }


Se intentó ajustar los puntos de la diversidad con esta distribución, sin embargo al ser pocos los rangos (la mayor cantidad de rangos en cualquier combinación fue de 250),  la curva descrita  no ajusta correctamente;  si se tuvieran mayor cantidad de rangos (1000 o 1000) del orden de $10^{3}$ o $10^{4}$ el ajuste es más preciso.  Para solucionar este problema se propuso hacer un ajuste lineal con la función logarítmica  de la siguiente forma:


\begin{enumerate}
	\item Se propone una función para la diversidad $d(k)$ de la forma
	
	\begin{equation}
	\label{ec.ajuste}
	y(k) =  \alpha \, ln(k) + \beta
	\end{equation}
	
	\item Al realizar los cambios de variable $\hat{Y} = y(k)$ y $X = ln(k)$, se obtiene una ecuación lineal para $k$.
	$$ \hat{Y} =  \alpha X + \beta$$
	
	\item Para encontrar los parámetros $\alpha$ y $\beta$, se utilizó el método de mínimos cuadrados, minimizando la suma de los cuadrados de los errores.  
	
	\item Conocidos los valores de $X$, la diversidad $Y$ y la cantidad de valores $n$,  se calcularon los valores muestrales de las medias ($\mu_{X}$ y $\mu_{Y}$), las varianzas ($\sigma^{2}_{X}$ $\sigma^{2}_{Y}$)  y la covarianza de las dos variables  $\sigma_{XY}$.
	
	$$ \bar{X} = \frac{1}{n} \sum_{i=1}^{n} X_{i} $$
	
	$$ \bar{Y} = \frac{1}{n} \sum_{i=1}^{n} Y_{i} $$
	
	$$ \sigma^{2}_{X} = \frac{1}{n} \sum_{i=1}^{n} \left (X_{i} -\bar{X}\right )^{2} $$
	
	$$ \sigma^{2}_{Y} = \frac{1}{n} \sum_{i=1}^{n} \left (Y_{i} -\bar{Y}\right )^{2} $$
	
	$$ \sigma_{XY} = \frac{1}{n} \sum_{i=1}^{n} \left (X_{i} - \bar{X}\right )  \left (Y_{i} - \bar{Y} \right ) $$
	
	\item Así los parámetros se expresan como
	
	$$ \alpha = \frac{\sigma_{XY}}{\sigma^{2}_{X}} $$
	
	$$ \beta = \bar{Y} - \alpha \bar{X}$$
	
	\item Calculados cada punto del ajuste (\ref{ec.ajuste}) $y_{i}$ y los valores calculados de diversidad $f_{i}$, para comprobar que tan adecuado es el ajuste se obtiene el coeficiente de determinación $R^{2}$.
	
	\begin{equation}
	\label{ec.rcuadrado}
	R^{2} = \frac{\sigma_{XY}}{\sigma^{2}_{X} \sigma^{2}_{Y}} \,\, = \,\, 1- \frac{\sum_{i=1}^{n} \left( y_{i} - f_{i}\right)^{2} }{\sum_{i=1}^{n} \left (y_{i} -\bar{Y}\right )^{2}}
	\end{equation}
	
	
	 
\end{enumerate}


Valores de $R^{2}$ próximos a 1 indicarán que existe una relación lineal ( en este caso logarítmica por el cambio de variable) exacta entre las dos variables

Las posteriores gráficas corresponden a las diferentes combinaciones entre idiomas orígenes y los receptores donde se calculó la diversidad de rango y el ajuste correspondiente.  Por cada conjunto de gráficas se muestra una tabla con los diferentes parámetros, la media $\mu$, la desviación estándar $\sigma$, el rango minimo donde se buscó la diversidad $k_{min}$, los parámetros del ajuste $\alpha$ y $\beta$  y el coeficiente de determinación $R^{2}$.
 
\newpage

\begin{figure}[h!]
	\centering
	\includegraphics[width=15cm, height=6.8cm]{Cap_6/DR_EN.png}
	\label{fig.DR_EN}
	\caption{Diversidad de rango del inglés en los demás idiomas.}
\end{figure}


\begin{table}[h!]
	\centering
	\begin{tabular}{ccccccc}
		\textbf{}                & \textbf{$\mu$} & \textbf{$\sigma$} & \textbf{$k_{min}$} & \textbf{$\alpha$} & \textbf{$\beta$} & \textbf{$R^{2}$} \\
		\textbf{inglés-francés}  & 0.61           & 0.18                & 265                   & 0.18           & -0.23         & 0.94        \\
		\textbf{inglés-alemán}   & 0.49           & 0.19                & 96                    & 0.19           & -0.23         & 0.92        \\
		\textbf{ingles-italiano} & 0.45           & 0.17                & 55                    & 0.17           & -0.10         & 0.91        \\
		\textbf{ingles-español}  & 0.38           & 0.15                & 70                    & 0.15           & -0.14         & 0.88       
	\end{tabular}
	\caption{Parámetros de la diversidad del inglés en los demás idiomas.}
	\label{tab.DR_EN}
\end{table}



\newpage

\begin{figure}[h!]
	\centering
	\includegraphics[width=1 \textwidth, scale = .38]{Cap_6/DR_FR.png}
	\label{fig.DR_FR}
	\caption{Diversidad de rango del francés en los demás idiomas.}
\end{figure}


\begin{table}[h!]
	\centering
	\begin{tabular}{ccccccc}
		\textbf{}                & \textbf{$\mu$} & \textbf{$\sigma$} & \textbf{$k_{min}$} & \textbf{$\alpha$} & \textbf{$\beta$} & \textbf{$R^{2}$} \\
		\textbf{francés-inglés}  & 0.55           & 0.18                & 269                   & 0.18           & -0.29        & 0.93        \\
		\textbf{francés-alemán}   & 0.42           & 0.16                & 67                    & 0.17           & -0.12         & 0.87        \\
		\textbf{francés-italiano} & 0.35           & 0.16                & 96                    & 0.15           & -0.21         & 0.83        \\
		\textbf{francés-español}  & 0.28           & 0.14                & 57                    & 0.15           & -0.19         & 0.86       
	\end{tabular}
	\caption{Parámetros de la diversidad del francés en los demás idiomas.}
	\label{tab.DR_FR}
\end{table}


\newpage

\begin{figure}[h!]
	\centering
	\includegraphics[width=1 \textwidth, scale = .38]{Cap_6/DR_GE.png}
	\label{fig.DR_GE}
	\caption{Diversidad de rango del alemán en los demás idiomas.}
\end{figure}


\begin{table}[h!]
	\centering
	\begin{tabular}{ccccccc}
		\textbf{}                & \textbf{$\mu$} & \textbf{$\sigma$} & \textbf{$k_{min}$} & \textbf{$\alpha$} & \textbf{$\beta$} & \textbf{$R^{2}$} \\
		\textbf{alemán-inglés}  & 0.35          & 0.17                & 60                   & 0.18           & -0.22        & 0.88        \\
		\textbf{alemán-francés}   & 0.37           & 0.17                & 44                    & 0.19           & -0.16         & 0.89        \\
		\textbf{alemán-italiano} & 0.31           & 0.15                & 28                    & 0.17           & -0.11         & 0.91        \\
		\textbf{alemán-español}  & 0.22          & 0.08                & 14                    & 0.10           & -0.04         & 0.89       
	\end{tabular}
	\caption{Parámetros de la diversidad del alemán en los demás idiomas.}
	\label{tab.DR_GE}
\end{table}


\newpage

\begin{figure}[h!]
	\centering
	\includegraphics[width=1 \textwidth, scale = .38]{Cap_6/DR_IT.png}
	\label{fig.DR_IT}
	\caption{Diversidad de rango del italiano en los demás idiomas.}
\end{figure}


\begin{table}[h!]
	\centering
	\begin{tabular}{ccccccc}
		\textbf{}                & \textbf{$\mu$} & \textbf{$\sigma$} & \textbf{$k_{min}$} & \textbf{$\alpha$} & \textbf{$\beta$} & \textbf{$R^{2}$} \\
		\textbf{italiano-inglés}  & 0.31          & 0.15                & 66                   & 0.15           & -0.18        & 0.86        \\
		\textbf{italiano-francés}   & 0.41           & 0.20                & 88                    & 0.20           & -0.28         & 0.85        \\
		\textbf{italiano-alemán} & 0.26           & 0.12                & 32                    & 0.13           & -0.08         & 0.91        \\
		\textbf{italiano-español}  & 0.22          & 0.19                & 212                    & 0.20           & -0.28         & 0.92       
	\end{tabular}
	\caption{Parámetros de la diversidad del italiano en los demás idiomas.}
	\label{tab.DR_IT}
\end{table}


\newpage

\begin{figure}[h!]
	\centering
	\includegraphics[width=1 \textwidth, scale = .38]{Cap_6/DR_SP.png}
	\label{fig.DR_SP}
	\caption{Diversidad de rango del español en los demás idiomas.}
\end{figure}


\begin{table}[h!]
	\centering
	\begin{tabular}{ccccccc}
		\textbf{}                & \textbf{$\mu$} & \textbf{$\sigma$} & \textbf{$k_{min}$} & \textbf{$\alpha$} & \textbf{$\beta$} & \textbf{$R^{2}$} \\
		\textbf{español-inglés}  & 0.41          & 0.18                & 106                   & 0.18           & -0.25        & 0.87        \\
		\textbf{español-francés}   & 0.28           & 0.14                & 75                   & 0.13           & -0.17         & 0.79        \\
		\textbf{español-alemán} & 0.21           & 0.09                & 21                    & 0.11           & -0.02         & 0.86        \\
		\textbf{español-italiano}  & 0.22          & 0.18                & 289                    & 0.18           & -0.28         & 0.95       
	\end{tabular}
	\caption{Parámetros de la diversidad del español en los demás idiomas.}
	\label{tab.DR_SP}
\end{table}            % ~5 páginas - Resumir lo que se hizo y lo que no y comentar trabajos futuros sobre el tema
\chapter{Omisión de palabras}

Anteriormente se especificó que la base de datos de los préstamos serían únicamente palabras de contenido, y a partir de ellas se realizaron los análisis anteriores.  Al trabajar con esta clasificación se están restringiendo a todas las palabras que pueden ser catalogadas como préstamos (algunos artículos o pronombres de un determinado idioma se encuentran en los demás), sin embargo los resultados obtenidos han reflejado contextos históricos en los cuáles explicar los movimientos de las palabras, por lo que las restricciones han ayudado a las deducciones. 

Por el momento ya no se tratara con la misma regularidad a la interpretación histórica de las migraciones, se centrará la siguiente parte en tratar a los prestamos como un conjunto que muestra la propiedad del uso de un idioma en otro, y ver como es afectada esta propiedad si es modificado el conjunto. 

La manera de alterar a los prestamos, se realizará al hacer restricciones en las palabras que conforman el conjunto; como antecedente se tiene el haber eliminado todas las palabras funcionales  y utilizar las de contenido.  Como todos los préstamos son palabras de contenido, las restricciones consistirán en eliminar las palabras que comiencen con ciertas letras.

\hfill\break

El proceso es sencillo y consiste en lo siguiente:
 


\begin{enumerate}
	
	\item Se eligen una pareja de idioma origen $\textit{A}$ e idioma receptor $\textit{B}$. Considerando como verdadero el uso de $\textit{A}$  en $\textit{B}$. 
	
	\item Se escogen de forma aleatoria un conjunto de letras (desde una hasta cuatro), y se eliminan del conjunto de prestamos de $\textit{A}$  en $\textit{B}$ a todas las palabras cuya primer letra sea alguna de las elegidas.  
	
	\item Se designan tres conjuntos:
	
		\begin{itemize}
			\item \textbf{Conjunto original:} Conformado por todos los prestamos de $\textit{A}$  en $\textit{B}$.
			\item \textbf{Conjunto reducido:} Conjunto original menos las palabras eliminadas. 
			\item \textbf{Conjunto residuo:} Conformado por las palabras eliminadas del conjunto original. 
		\end{itemize}
	
	\item En cada conjunto se empleo la ecuación \ref{ec.fuso} para obtener el uso de $\textit{A}$ en $\textit{B}$  con los elementos de cada conjunto. 
	
\end{enumerate}


Por la cantidad de elementos que contiene el conjunto residuo, el uso entre idiomas utilizando este conjunto es despreciable con los valores obtenidos para el conjunto original y el reducido.  Cabe decir que entre mas letras se escojan para reducir el conjunto, el residuo tendrá cada vez más elementos siendo en algun momento comparable al original.  Por ello se decidió que el máximo de restricciones fuese de cuatro letras. 

La forma de las graficas del uso del conjunto original es el mismo que se presentó en el capitulo anterior, sólo que ahora se graficaran en color negro sin importar que combinación de idiomas se este tratando.  Para el conjunto reducido cada valor de uso se marco en color rojo.  En cada grafica se especifica que letras se usaron para reducir al conjunto.


\subsubsection*{Inglés}

\begin{figure}[h!]
	\centering
	\includegraphics[width=14.5cm, height=7cm]{Cap_5/OM_EN.png}
	\label{fig.OM_EN}
	\caption{Omisiones del inglés en los demás}
\end{figure}

\newpage
\subsubsection*{Francés}

\begin{figure}[h!]
	\centering
	\includegraphics[width=14.5cm, height=7cm]{Cap_5/OM_FR.png}
	\label{fig.OM_FR}
	\caption{Omisiones del francés en los demás}
\end{figure}



\subsubsection*{Alemán}

\begin{figure}[h!]
	\centering
	\includegraphics[width=14.5cm, height=7cm]{Cap_5/OM_GE.png}
	\label{fig.OM_GE}
	\caption{Omisiones del alemán en los demás}
\end{figure}

\newpage
\subsubsection*{Italiano}

\begin{figure}[h!]
	\centering
	\includegraphics[width=14.5cm, height=7cm]{Cap_5/OM_IT.png}
	\label{fig.OM_IT}
	\caption{Omisiones del italiano en los demás}
\end{figure}


\subsubsection*{Español}

\begin{figure}[h!]
	\centering
	\includegraphics[width=14cm, height=7cm]{Cap_5/OM_SP.png}
	\label{fig.OM_SP}
	\caption{Omisiones del español en los demás}
\end{figure}
 

\chapter{Conclusiones}

En el presente trabajo, se ha tratado de estimar una forma de cuantificar la influencia que un idioma ejerce sobre otro, proceso que no resultó sencillo al no existir un antecedente en el cual basarse o tomarse como referencia, no obstante, la cuantificación llegó a través de dos métodos, con resultados importantes en ambos. 

En el primero al contar las palabras nuevas, se expuso que las palabras que migran de un idioma a otro, son parte de un mismo campo semántico, y las migraciones ocurren tras un suceso que tiene relación con el campo semántico. Con ello, se refleja la influencia que tiene un idiomas en determinadas áreas.  

El segundo método para cuantificar la influencia, por medio del uso de un idioma en otro, mostró que las palabras migrantes que continuamente son empleadas por los demás idiomas, también son parte de un mismo campo semántico. Además, estas descienden su rango (aumentan su frecuencia), en los años posteriores al evento que las hizo migrar. 

Con ambos métodos, se concluyó que las áreas donde un idioma es más influyente, y cuyas palabras son continuamente empleadas, son en el inglés la guerra, la economía, la tecnología, la política y la globalización; en el francés la guerra, la Revolución Francesa, la religión y la industria vitivinícola; en el alemán la guerra y los apellidos de personajes germano parlantes que destacaron el algún área académica; en el italiano  la guerra, la política y la religión; y en el español la medicina y la cultura de los países Latinoamericanos. 

En la parte estadística del trabajo, se destaca que las palabras migrantes de cada pareja de idiomas, siguen una regla común si estas se ordenan de acuerdo a su frecuencia de aparición,  y es que en cada año, los primeros lugares del ordenamiento, son ocupados por una menor cantidad de palabras, y conforme se avancen lugares en el ordenamiento, aumentarán las palabras distintas que los ocupen. 

En el ultimo capitulo del trabajo, se trataron de justificar los resultados, pese al escaso rigor lingüístico que se tuvo. En esta justificación, se mostró que a pesar que una palabra (o un conjunto de ellas) no pertenezca con exactitud a un idioma,  el uso de un idioma en otro no se ve alterado si está es excluida del conjunto. La eliminación de palabras, ejemplifica que mientras se considere a las palabras como un conjunto,  estas mantendrán sus características, sin importar cuantos elementos conformen el conjunto. 






%%%%%%%%%%%%%%%%%%%%%%%%%%%%%%%%%%%%%%%%%%%%%%%%%%%%%
%                   APÉNDICES                       %
%%%%%%%%%%%%%%%%%%%%%%%%%%%%%%%%%%%%%%%%%%%%%%%%%%%%%
\appendix

% this file is called up by thesis.tex
% content in this file will be fed into the main document
\chapter{Complementos}
% top level followed by section, subsection

\section{Lectura de listas}

\newpage

\section{Gráficas de palabras nuevas entre dos idiomas}

\begin{figure}[h!]
	\centering
	\includegraphics[scale=.38]{Cap_3/NC_1_S2_EN.png}
	\label{fig.NC_EF}
	\caption{Palabras nuevas entre el inglés y el francés}
\end{figure}

\begin{figure}[h!]
	\centering
	\includegraphics[scale=.38]{Cap_3/NC_2_S2_EN.png}
	\label{fig.NC_EG}
	\caption{Palabras nuevas entre el inglés y el alemán}
\end{figure}

\begin{figure}[h!]
	\centering
	\includegraphics[scale=.38]{Cap_3/NC_3_S2_EN.png}
	\label{fig.NC_EI}
	\caption{Palabras nuevas entre el inglés y el italiano}
\end{figure}


\begin{figure}[h!]
	\centering
	\includegraphics[scale=.38]{Cap_3/NC_4_S2_EN.png}
	\label{fig.NC_ES}
	\caption{Palabras nuevas entre el inglés y el español}
\end{figure}

\begin{figure}[h!]
	\centering
	\includegraphics[scale=.38]{Cap_3/NC_2_S2_FR.png}
	\label{fig.NC_FG}
	\caption{Palabras nuevas entre el francés y el alemán}
\end{figure}


\begin{figure}[h!]
	\centering
	\includegraphics[scale=.38]{Cap_3/NC_3_S2_FR.png}
	\label{fig.NC_FI}
	\caption{Palabras nuevas entre el francés y el italiano}
\end{figure}

\begin{figure}[h!]
	\centering
	\includegraphics[scale=.38]{Cap_3/NC_4_S2_FR.png}
	\label{fig.NC_FS}
	\caption{Palabras nuevas entre el francés y el español}
\end{figure}

\begin{figure}[h!]
	\centering
	\includegraphics[scale=.38]{Cap_3/NC_3_S2_GE.png}
	\label{fig.NC_GI}
	\caption{Palabras nuevas entre el alemán y el italiano}
\end{figure}

\begin{figure}[h!]
	\centering
	\includegraphics[scale=.38]{Cap_3/NC_4_S2_GE.png}
	\label{fig.NC_GS}
	\caption{Palabras nuevas entre el alemán y el español}
\end{figure}

\begin{figure}[h!]
	\centering
	\includegraphics[scale=.38]{Cap_3/NC_4_S2_IT.png}
	\label{fig.NC_IS}
	\caption{Palabras nuevas entre el italiano y el español}
\end{figure}

\newpage


               % Colocar los circuitos, manuales, código fuente, pruebas de teoremas, etc.

%%%%%%%%%%%%%%%%%%%%%%%%%%%%%%%%%%%%%%%%%%%%%%%%%%%%%
%                   REFERENCIAS                     %
%%%%%%%%%%%%%%%%%%%%%%%%%%%%%%%%%%%%%%%%%%%%%%%%%%%%%
% existen varios estilos de bilbiografía, pueden cambiarlos a placer
%\bibliographystyle{apalike} 
%\bibliographystyle{unsrt}
% otros estilos pueden ser abbrv, acm, alpha, apalike, ieeetr, plain, siam, unsrt

%El formato trae otros estilos, o pueden agregar uno que les guste:
%\bibliographystyle{Latex/Classes/PhDbiblio-case} % title forced lower case
%\bibliographystyle{Latex/Classes/PhDbiblio-bold} % title as in bibtex but bold
\bibliographystyle{Latex/Classes/PhDbiblio-url} % bold + www link if provided
%\bibliographystyle{Latex/Classes/jmb} % calls style file jmb.bst

\bibliography{Bibliografia/referencias}             % Archivo .bib

\end{document}
