% !TeX spellcheck = es_ES
\chapter{Palabras nuevas}


% Intro {{{

En el capítulo anterior, se mencionó que el algoritmo toma como  idioma
dominante a aquel que transmite palabras hacia los demás y que estas perduran en los receptores al menos dos años.  El movimiento de palabras, es una característica que puede proporcionar información sobre la influencia entre los idiomas, sin embargo  puede resultar ambiguo decir que algo es más o menos influyente, además  no existe en la literatura un método que mida la influencia.

%El empleo de estás características puede proporcionar información acerca de la influencia que ejercen los idiomas,  sin  embargo, establecer un método que brinde un resultado al cual ligar la expresión influencia  no es sencillo, no existe en la literatura tal proceso, además puede ser ambiguo decir que algo es más o menos influyente sin un valor que lo respalde


%Al momento, en la literatura no existe una serie de pasos a seguir cuyo resultado final sea una cantidad que mida la influencia. Para llegar a esta resolución se requieren establecer nuevas condiciones en la base de datos, y a partir de ellas trazar posibles caminos que lleven a relaciones y resultados  con los cuales satisfacer una interpretación de la influencia

%\fxnote{ah? Otra palabra que no cuadra para nada\ldots}, 
%\fxnote{No  entiendo la ultima frase. Ten cuidado con los cambios. Siento que lo intentas hacer un poco rococo y elegante, pero por tratar, pierdes precision. Creo qeu lo mas importante  en textos cientificos es ser preciso, ordenado y conciso.}.


% \fxnote{division? No entendi\ldots}
% \fxwarning{deje comentado el texto donde me dejaste la nota y lo cmabie por el siguiente} 
% 
% \fxwarning{Ya trate de ser más conciso en lo que quiero decir, corregi todo el parrafo}
% 



%\fxnote{Revisa redaccion} 
%\fxwarning{listo}
 

%El llegar a esta conclusión requiere el establecer condiciones y características en los elementos para poder llegar a un resultado que satisfaga una interpretación de la influencia. 

%\fxnote{Frase hueva. Que paso??? Se te bajo mucho el nivel de escritura}.
%\fxwarning{Listo, la corregi, comentare todas las frases o parrafos que vaya corrigiendo, para notar si hay un cmabiuo en la calidad}
 

% \fxnote{Ya no entiendo que quieres lograr con el siguietne parrafo. No se cual es el mensaje principal. De nuevo ``teñir un panorama'' no me suena correcto. Hablemos de la estructura de un parrafo y creo que tocara iterar de neuvo!!! Argh.}
% \fxwarning{ya Corregí todo el ,  dejo comentado el parrafo viejo}

%Aludiendo al capítulo  anterior, el tratamiento de la base de datos brinda información de los orígenes, los receptores y las palabras que migran de un lado a otro; comenzando con estos elementos, un primer paso es identificar los tiempos donde ocurren las migraciones. Antes de comprometer completamente al tiempo, es conveniente hacer una clasificación dentro de los propios préstamos para teñir un panorama más claro con el cual trabajar. 

Un primer paso para llegar a un valor que cuantifique a la influencia, es
estudiando los tiempos donde ocurren las migraciones, encontrando relaciones
entre dichos tiempos y las palabras involucradas en el movimiento. Antes de
profundizar en esta idea, es conveniente realizar una clasificación dentro los
préstamos, trabajar sobre ella y poder llegar a la cuantificación. 

Si se tiene una pareja de idioma origen \textit{A} e idioma receptor
\textit{B},  dentro de los préstamos de \textit{A} en \textit{B} se definen
como  \textbf{préstamos nuevos} a las  palabras que aparecen por primera vez en
las más usadas del idioma \textit{B}. Esta definición permite ordenar a los
préstamos nuevos por cada pareja de origen y receptor, y dentro  de este
ordenamiento, una segunda organización  por el año en el cual aparecieron. 



Dada la nueva clasificación, se determinaron tres posibilidades con las cuales
interpretar la influencia de un idioma sobre otro. 

\begin{enumerate}
	\label{proceso.nuevos}
	
\item Contar por cada año los préstamos nuevos de origen \textit{A} que están presentes en los diferentes receptores. Esto muestra en cuales idiomas \textit{A} ha sido influyente. 

\item Contar cuántos préstamos nuevos de diferentes orígenes están presentes en cada año, si se toma a \textit{A} como receptor. Con ello se obtienen los idiomas que han influenciado a \textit{A}

%Intercambiando a \textit{A}  como el idioma receptor, y contando cuántos préstamos nuevos de diferentes orígenes están presentes en cada año, obteniendo los idiomas que han influenciado a \textit{A}.  



\item Tomar fijos un idioma \textit{A} y un idioma \textit{B}, contando cuántos préstamos nuevos de \textit{A} están en \textit{B} y cuantos de \textit{B} lo están en \textit{A}.  Así se obtiene cómo ha sido la influencia entre \textit{A} y \textit{B}.

%contabilizando cuántos préstamos nuevos por año se encuentran al tomar a  \textit{A} como origen y a \textit{B} como receptor,  una vez hecho esto,  repetir el conteo intercambiando a \textit{B} como el origen y a \textit{A} como el receptor. Obteniendo las épocas donde alguno de ellos tuvo más del punto uno o del dos. 


\end{enumerate}


% }}}
\section{Eventos que ayudan a las migraciones} % {{{

Una parte complementaria de interpretar a la influencia es reconociendo causas
que originan las migraciones. Por ejemplo, sucesos como la globalización y el
acceso al desarrollo tecnológico a finales de 1980 y principios de 1990,
propiciaron la migración  de términos como \textit{internet},
\textit{computadora}, \textit{web}, \textit{email} o \textit{software}, siendo
el movimiento de este grupo de palabras, una consecuencia del desarrollo e
impacto de ambos eventos. 
 

<<<<<<< HEAD
Las causas que originan las migraciones, serán identificables a partir del significado de los préstamos involucrados en un año de migración. De acuerdo a \cite{mcgraw}, un  \textbf{campo semántico} es un conjunto de palabras asociadas que comparten parte de su significado. Entonces las palabras migrantes pueden estar relacionadas con un evento a partir de un campo semántico; esto será identificable porque las migraciones ocurren en el mismo año (o en los años alrededor) donde el evento aconteció. 






%Una parte complementaria de interpretar la influencia es reconocer posibles causas que favorecen a las migraciones,  identificables por el tiempo donde ocurren, es decir, en un año de migración el significado de las palabras puede guardar alguna relación con un evento ocurrido durante el mismo año  (o en los años al rededor de él),  ya que las palabras migrantes pueden pertenecer a un mismo campo semántico relacionado al  evento. Por ejemplo, la globalización y el acceso al desarrollo tecnológico a finales de la década de 1980 y principios de 1990 propició la migración  de términos como \textit{internet}, \textit{computadora}, \textit{web}, \textit{email} o \textit{software}, mostrando a ambos ámbitos como propagadores del movimiento entre un origen y los distintos receptores. Al tratar con idiomas, se espera que en los eventos estén comprometidos países, personajes o comunidades que hablen o utilicen las lenguas involucradas. 











=======
Las causas que originan las migraciones, serán identificables a partir del
significado de los préstamos involucrados en un año de migración. De acuerdo a
\cite{mcgraw}, un  \textbf{campo semántico} es un conjunto de palabras
asociadas que comparten parte de su significado. Entonces las palabras
migrantes pueden estar relacionadas con un evento a partir de un campo
semántico, esto será identificable porque las migraciones ocurren en el mismo
año (o en los años al rededor) donde el evento aconteció. 
>>>>>>> 5099562879fa04dc0a9e80ca7289c913bb8c01a6

% }}}
\section{Palabras nuevas de un idioma en los demás} % {{{

Descritas las características que engloban a los préstamos nuevos, la obtención
y presentación de resultados se realizó de la siguiente manera. 


\begin{itemize}

\item Se buscaron los préstamos nuevos entre idiomas,  durante los 109 años comprendidos en el conjunto de búsqueda (1900-2009).
	
\item Por cada idioma, se determinó la influencia que éste ejerció sobre los
demás, mencionada en el punto 1 de la página~\pageref{proceso.nuevos}). La
nomenclatura para la cantidad de palabras nuevas será $N_{p}$.

\item Las gráficas obtenidas, muestran la cantidad de palabras nuevas $N_{p}$
que aparecieron en los 109 años y a la vez, como se distribuye esta cantidad en
las diferentes décadas. 

\item Se proporciona en \cite{prestamos_nuevos} las listas  de préstamos nuevos entre cada pareja de idioma origen e idioma receptor, agrupados por el año de aparición.  Se especifica en \ref{lectura.listas}  la forma de leer las listas.

\item Por cada resultado y por cada par de idiomas, se agregó un contexto de los sucesos relacionados con las migraciones. Esto permite omitir los resultados del punto 2 (ver  página \pageref{proceso.nuevos}), ya que exhibe los campos donde \textit{A} influenció a  \textit{B}, y de manera implícita, los campos donde \textit{B} fue susceptible.

\item Las palabras mencionadas se carecen de signos ortográficos y su escritura es en minúsculas, ya que así provienen de la base de datos. 

\item Los comentarios realizados, son sustentados por la influencia entre \textit{A} y \textit{B} (punto 3), las graficás correspondientes se anexan en \ref{palabras.nuevas.apendice}. 

\end{itemize}

%\fxnote{Respecto al numero en la bibliografia despues de la referencia, para que lo quieres? Creo que no es estandard y a mi me confunde. Te propongo quitarlo.}

%\fxwarning{ok, veo como quitarlo por el momento no lo se}

\fxnote{No creo que amerite una subsección. La quite. Si no estas de acuerdo,  respondeme y la vuleves a poner.}

\fxwarning{de acuerdo, es más una adicion a la misma seccion}

En las gráficas se utilizaron diferentes colores y un sistema de abreviaciones para distinguir a los idiomas que intervienen, donde la primera abreviación corresponde al idioma origen y la segunda al idioma receptor. Los colores y abreviaciones se especifican  en la tabla \ref{tab.idcolor}. 

En todas las gráficas, el eje horizontal simboliza a los años del conjunto de búsqueda (1900-2009),  mientras el eje vertical representa la cantidad de prestamos nuevos $N_{p}$. 

%\fxnote{Nota como hace uno referencia a una tabla. Se deja flotante. Cuando  entiendas esta nota, borrala.}

\begin{table} % {{{
	\centering
	\begin{tabular}{ccc}
		\textbf{Idioma} & \textbf{Abreviación} & \textbf{Color} \\
		Inglés          & EN                   & Azul           \\
		Francés         & FR                   & Amarillo       \\
		Alemán          & GE                   & Violeta        \\
		Italiano        & IT                   & Verde          \\
		Español         & SP                   & Guinda        
	\end{tabular}
	\caption{Nomenclatura de los idiomas}
	\label{tab.idcolor}
\end{table} % }}}




\fxnote{Me parece que la anterior seccion y esta son una sola. Si quieres lo  discutimos}
	
\fxwarning{Cambie el nombre de la seccion por Palabras nuevas,  dentro de esta lo primero será especificar  como se presentan los resultados y las siguientes subsecciones los resultados mismos}

\fxnote{Antes de avanzar, tenemos que discutir las graficas. Creo que admiten una mejoría. Ver notas en un caption mas adelante y poner en espacio sencillo}

\fxwarning{listo}

\subsection{Inglés} % {{{

\begin{figure} % {{{
	\centering
	\includegraphics[scale=.38]{NC_EN.png}
	\label{fig.NC_EN}
	\caption{Palabras nuevas del inglés hacia los demás idiomas. El alemán se ha beneficado más del inglés, con los mayores aportes en los primeros años del siglo XXI. Las palabras que intervienen son de carácter industrial y del desarrollo tecnológico.}
\end{figure} %  }}}

%De manera general, el idioma que más se ha beneficiado del inglés ha sido el alemán, con 300 préstamos en 100 años.  Inglés y alemán forman parte de las lenguas germánicas,  posible razón de los mayores intercambios. Entre las lenguas romances, el francés fue el más favorecido, pero también es la más similar por las relación normanda entre ambas.

\subsubsection*{Inglés-Francés} % {{{

Los mayores aportes se dieron entre 1930 y 1970, periodo que engloba comienzos de
la gran depresión (1929), la segunda guerra mundial (1939-1945) y la guerra fría (1945-1991), sucesos donde participaron países de ambas lenguas. Las palabras nuevas en el francés en este periodo que hacen referencia a estos eventos son \textit{churchill} (1944), \textit{territories} (1944), \textit{nazis} (1945), \textit{catastrophe}(1945), \textit{dollar}(1950), \textit{nixon}(1968) y \textit{johnson}(1970); siendo las ultimas,  apellidos de los presidentes de los Estados Unidos  Lyndon B. Johnson y Richard Nixon, cuyos periodos de gobierno fueron  entre 1963-1969 y 1969-1974 respectivamente.

% }}}
\subsubsection*{Inglés-Alemán} % {{{

Sólo  en dos épocas (1900  y 1980), el alemán no fue el idioma que más prestamos recibió  del ingles. Por el año de aparición, palabras como \textit{economic} (1929), \textit{depression} (1931), \textit{investment} (1933) y \textit{roosevelt} (1935), son del campo semántico de la gran depresión,  mientras que  Franklin D. Roosevelt fue el presidente de los Estados Unidos que gobernó posterior a la crisis económica y durante la segunda guerra mundial. 

La crisis económica de la gran depresión, se origino en los Estados Unidos con consecuencias en diferentes países, entre ellos  Alemania, siendo uno de los motivos que propiciaron la segunda guerra mundial.

En las ultimas dos décadas, surgen términos referentes a la globalización y al desarrollo tecnológico, entre ellas \textit{standards} (1983), \textit{market} (1994), \textit{internet} (1996), \textit{economy} (1996), \textit{online} (1998), \textit{value} (2001), \textit{financial} (2003) y \textit{customer} (2007). 
% }}}

\subsubsection*{Inglés-Italiano} % {{{


Las palabras hacia el italiano, identificables en la primera mitad de siglo son \textit{roosevelt} (1941) y \textit{stalin} (1949), apellidos de personajes involucrados en la segunda guerra mundial. En el caso de Joseph Stalin, a pesar de que su nacionalidad no es de algún país de habla inglesa, en el inglés su apellido tomó notoriedad para exportarse a los otros idiomas, siendo un ejemplo de palabras que se hacen populares en idiomas distintos al idioma origen. 

En los últimos años, nuevamente la globalización y la economía, son áreas comunes para los préstamos del inglés, \textit{internet} (1996), \textit{bussines} (2000) y \textit{marketing} (2001), son ejemplos de ellas. 


\subsubsection*{Inglés-Español} 

El español ha sido en cada década,  el idioma que menos prestamos toma del ingles, sin embargo las relaciones encontradas han sido las más bastas.  Nombres de organizaciones y empresas,  \textit{standard} (1933) (y \textit{oil} (1931) aludiendo a la extinta Standard Oil) y \textit{unesco} (1955);  presidentes de los Estados Unidos,  \textit{roosevelt} (1941), \textit{kennedy} (1961), \textit{johnson} 1966),  \textit{nixon} (1972) y \textit{bush} (1990); y la globalizacion, \textit{internet} (196), \textit{mail} (1999), \textit{marketing} (2001) y \textit{software} (2004).   

%A pesar de no ser el idioma más favorecido es al que en más areas ha impregnado el ingles, siendo este un factor que también puede indicar una mayor influencia,  en cuantas áreas esta presente un idioma y que tanto se utiliza. 

% }}}
% }}}
\subsection{Francés} % {{{

\begin{figure}[h!]
	\centering
	\includegraphics[scale=.38]{NC_FR.png}
	\label{fig.NC_FR}
	\caption{Palabras nuevas del francés hacia los demás idiomas. Entre las lenguas romances, el aporte del francés ha sido minoritario siendo las lenguas germánicas donde el francés llevó más elementos, principalmente en el alemán, durante la primera (1920) y segunda guerra mundial (1940).}
	
	
\end{figure}

\subsubsection*{Francés-Inglés}% {{{

Durante el siglo XX, los préstamos en este sentido, se caracterizan por ser palabras comunes e identificables de origen inglés,  por ejemplo  \textit{diagnostic,} \textit{clients,} \textit{placement,} \textit{adaptation,} \textit{diffusion,} \textit{amplitude,}.  A pesar de ser errores en los resultados, este tipo de palabras si surgieron primero en el francés, por lo que el algoritmo designó a esta lengua como el origen. 

% }}}
\subsubsection*{Francés-Alemán}% {{{

El alemán tuvo dos décadas donde  fue e idiomas que más palabras adopto del francés, entre 1920 y 1940  (años cercanos a las dos guerras mundiales), se ubicaron a  textit{diplomatie} (1917), \textit{bourgeoisie} (1919),  \textit{guerre} (1925), \textit{allemagne} (1925), \textit{russie} (1925) y \textit{empire} (1937); palabras que en conjunto son referentes a temas políticos y diplomáticos, acordes a ambas contiendas. 


% }}}
\subsubsection*{Francés-Italiano}% {{{

Las clasificaciones para esta pareja de origen y receptor son escasas, entre las pocas realizadas se encuentra un término del campo científico como \textit{poincare} (1924) (apellido del matemático francés Henri Poincaré);  y nombres de ciudades y países referentes tras las dos guerras mundiales, \textit{versailles} (1924), \textit{vietnam} (1966)  y \textit{urss} (1975).


% }}}
\subsubsection*{Francés-Español}% {{{

Al igual que el italiano, en el español hay pocas palabras cuyo contenido fue ligado a un evento. La palabra más destacada es \textit{euros} (2002),
nombre y primer año de circulación de la moneda de la unión europea, organización donde son miembros España y Francia. 

%El hecho de no poder enlazar palabras a eventos, no significa que el francés no es importante para el español (o el italiano), sino que el periodo donde los hechos tuvieron mayor impacto no esta dentro del periodo de búsqueda,  por ejemplo hechos como la revolución francesa, o la invasión napoleónica a España, propiciaron a un mayor intercambio en este sentido, pero al ocurrir antes de 1900 no permite tener una conclusión de ello. 


% }}}
% }}}
\subsection{Alemán}% {{{

\begin{figure}[h!]
	\centering
	\includegraphics[scale=.38]{NC_GE.png}
	\label{fig.NC_GE}
	\caption{Palabras nuevas del alemán hacia los demás idiomas. El francés destaca  entre 1930-1940 por ser el idioma con más préstamos del alemán. Se distingue al español por tener periodos donde no recibió alguna palabra.}  
\end{figure}




\subsubsection*{Alemán-Inglés}% {{{

En los años posteriores a la segunda guerra mundial, se encontraron palabras ligadas al evento, entre ellas \textit{lenin} (1931), \textit{hitler} (1934) y \textit{reich} (1939).  Otras palabras relevantes son \textit{marx} (1934) y \textit{freud} (1934), apellidos de personajes destacados en la filosofía y psicología. 


% }}}
\subsubsection*{Alemán-Francés}% {{{

%El francés ha recibido más palabras del alemán que cualquier otro. A pesar de que la mayor cantidad de aportes se dio en la primera mitad de siglo, las relaciones que se encontraron han sido a lo largo de todo el periodo y en diferentes áreas. 

Se distinguieron dos campos comunes para los préstamos nuevos,  el primero referente a la  la historia del alemán en las guerras, \textit{kaiser} (1915), \textit{reich} (1921), \textit{hitler} (1933),  \textit{regierung}, \textit{deutschen},\textit{minister} y  \textit{bestimmungen} (traducciones de gobierno, alemán, ministro y reglamentos) las ultimas surgiendo en 1944.  

El segundo campo, son apellidos de  personajes destacados en la historia, la filosofía, la música y la psicología, además de Hitler, se encuentra  \textit{nietzsche} (1905),  \textit{marx} (1923),  \textit{mozart} (1956), \textit{freud} (1965), \textit{engels} (1970) y \textit{heidegger} (1987); todos ellos  nacidos en países germanohablantes.

% }}}
\subsubsection*{Alemán-Italiano}% {{{

Nuevamente los préstamos que aporta el alemán,  son apellidos de personajes,  además de los ya mencionados, el único apellido exclusivo en el italiano fue \textit{berchtold} en 1943, referente a Leopold Berchtold, ministro de exteriores del Imperio Austro-Húngaro de 1912 a 1915, cuyo fallecimiento ocurrió en 1942.


  
% }}}
\subsubsection*{Alemán-Español}% {{{

Las palabras que van en este sentido,  presentaron  décadas  con pocas migraciones. Entre los préstamos encontrados estan \textit{marx} (1932), \textit{kaiser} (1938), \textit{hitler} (1940), \textit{lenin} (1970), \textit{hegel} (1971),  \textit{nietzsche} (2000) y \textit{freud} (2002). Estas palabras a pesar de haber sido mencionadas,  las migraciones hacia el español ocurren después que en los demás, por ejemplo en el francés,  Nietzsche apareció 1905 y Freud en 1934. 

La diferencia en los años de migración puede ser un indicio de la adaptabilidad de una lengua en otra, aunque en este trabajo no se ha desarrollado esta idea. 



% }}}
% }}}
\subsection{Italiano}% {{{

\begin{figure}[h!]
	\centering
	\includegraphics[scale=.38]{NC_IT.png}
	\label{fig.NC_IT}
	\caption{Palabras nuevas del italiano hacia los demás idiomas. Al provenir de la misma familia grecolatina, siendo fonética y etimológicamente similares, el español y el francés han adoptado la mayor cantidad de palabras provenientes del italiano.} 
\end{figure}




\subsubsection*{Italiano-Inglés}% {{{

A pesar de que en cada década existen términos nuevos en el inglés, sólo fue posible relacionar \textit{mussolini} (1935),  alusivo al político y militar Benito Mussolini, posiblemente el personaje italiano más relevante en la historia del siglo XX .

% }}}
\subsubsection*{Italiano-Francés}% {{{



En las migraciones sólo se asoció \textit{mussolini} (1935), la cual ya se había mencionado. Aunque en 1940 migraron la mayor cantidad de préstamos, ninguno de ellos ha tenido contexto con los sucesos de esa época. 

Tras revisar las listas de los préstamos nuevos con origen italiano  en los demás idiomas, Mussolini se encuentra en todas ellas, surgiendo siempre en 1935.





% }}}
\subsubsection*{Italiano-Alemán}% {{{

En esta dirección existen relaciones con el contexto bélico,  \textit{regime} (1938), \textit{panzer} (1941), \textit{duce} (1942),  traducciones de régimen, blindado y líder.

%además de \textit{Mussolini} (1935). 



% }}}
\subsubsection*{Italiano-Español}% {{{

En el español, además de los términos de la guerra, se encontraron nombres de términos políticos y sociales que tuvieron un auge en el siglo pasado,  \textit{socialista} (1914), \textit{comunista} (1932), \textit{capitalismo} (1935), \textit{fascismo} (1937),  \textit{marxismo} (1963) y \textit{terrorismo} (1986). 

% }}}
% }}}
\subsection{Español}% {{{

\begin{figure} % {{{
	\centering
	\includegraphics[scale=.38]{NC_SP.png}
	\label{fig.NC_SP}
	\caption{Palabras nuevas del español hacia los demás idiomas. El italiano es el idioma  que más palabras recibió del español, seguido del francés, proviniendo los tres idiomas, de la misma familia lingüística. }
\end{figure} % }}}

\subsubsection*{Español-Inglés}% {{{

Contrario a la tendencia en las migraciones anteriores, la guerra no es un campo semántico común en el contenido de los préstamos. La mayor parte de los términos son habituales en la medicina,  en 1943  aparecieron  \textit{virus} y \textit{anemia};  años antes en 1934 George Richards Minot, Parry Murphy y George Hoiyt Whipple, habían recibido el premio Nobel de medicina por su descubrimiento de la terapia de hígado para el tratamiento de anemias.   

Probablemente, las palabras virus y anemia ya existían en el inglés antes de 1943,  pero sólo hasta este año,  cobraron relevancia para ser parte de las cinco mil palabras más utilizadas. Esto ejemplifica que hay eventos (como un premio internacional) que retoman palabras cuyo periodo de uso  ha disminuido,  para volver a ser importantes y migrar a los demás idiomas.


\subsubsection*{Español-Francés}% {{{

El primer préstamo con contenido es \textit{panama} (1913), su importancia se debe a la inauguración del canal de Panamá en 1914, destaca que la palabra llegue al francés por ser el gobierno de Francia el que impulsó económicamente su construcción, aunque su conclusión y administración pasó a los Estados Unidos.  




% }}}
\subsubsection*{Español-Alemán}


Los préstamos son nuevamente términos médicos, además \textit{virus} y \textit{anemia} mencionados en las migraciones hacia el inglés, se encontró a \textit{lepra} (1901), anteriormente en 1874, Gerhard Armauer Hansen descubrió el bacilo de Leprae que origina la enfermedad.

%fue globalmente importante a partir de 1874,  ya que en ese año el científico noruego Gerhard Armauer Hansen descubrió el bacilo de Hansen Mycobacterium Leprae \cite{lepra} que origina la enfermedad. Por el carácter médico de la palabra, es probable que se hiciera más investigación sobre la enfermedad en diferentes idiomas, en este caso el alemán. 


% }}}
\subsubsection*{Español-Italiano}% {{{


Además de los términos medicos ya mencionados, en el italiano migraron de forma  exclusiva  \textit{virus} (1922), \textit{colesterina} (1928),  \textit{sintomatología} (1931), \textit{anestesia} (1932), \textit{vitamina} (1935), \textit{anemia} (1936), \textit{metabolismo} (1936),  \textit{gástrica} (1936)  y \textit{endovenosa} (1937).  

%El aparecer estas palabras en el español (dentro de las cinco mil más usadas)antes que en los demás,  sugiere que la medicina era un campo importante para los países de habla española, donde posiblemente se publicaron más libros de medicina en esta lengua. 






% }}}
% }}}
% }}}
\section{Comentarios del método}% {{{



Tras las múltiples combinaciones entre idiomas, la relación habitual son palabras del campo semántico de la guerra, la mayor parte de ellas, surgieron en los diferentes receptores durante y después de la segunda guerra mundial. 

Cada idioma se muestra como un exportador de palabras de ciertos campos,  el inglés en economía, tecnología y política; el español en medicina; mientras el francés, el alemán y el italiano en la guerra.  

Las áreas mencionadas, no brindan una respuesta sobre que idioma ha sido más influyente, pero si en cuales campos un idioma ha influido más que los otros, siendo esta una forma alternativa de hablar de influencia en los idiomas.
 
Una manera diferente de estudiar a los préstamos nuevos, sería midiendo el tiempo que le toma a las palabras moverse de un idioma a otro, dando una idea de la velocidad con la que migran y se adaptan a las lenguas receptoras. Aunque estos resultados pueden ser complementarios, por el momento no se han tratado. 


%El inglés se presenta en las últimas dos décadas como el idioma común para transmitir información, exportando términos comunes en ámbitos como  la globalización y el desarrollo  de la tecnología.  Destaca el rol de los Estados Unidos como un país involucrado en los principales acontecimientos que originaron las migraciones, siendo usuales los apellidos de todos sus presidentes (posteriores a la segunda guerra mundial) en los demás idiomas. 

%Salvo el inglés que fue exportador en distintas áreas, los demás idiomas se caracterizaron por brindar palabras especificas,  el alemán por apellidos de personajes, el español por términos médicos, mientras que el francés y el italiano por la historia bélica y la religión. . 

 %Estos posibles resultados ayudarían a complementar la relación entre eventos, ya que en algunos eventos las palabras asociadas a él, migraron a los demás idiomas  en el mismo periodo, por ejemplo,  las palabras que migraron tras la revolución francesa (1789-1799) aparecieron en los diferentes receptores mientras ocurría el suceso y hasta veinte años después de él; así mismo,  los términos involucrados en la globalización  posterior a 1980 migraron en los años inmediatos a su invención.  Para tales complementos se necesitaría separar a los préstamos por áreas, lo cual no se hizo en este trabajo.  












% }}}



