\chapter{Eliminación de palabras}


Los capítulos anteriores se han enfocado en tratar a las migraciones de palabras como consecuencias de eventos donde  las lenguas están involucradas. La clasificación de los $n$-grams en palabras funcionales y palabras de contenido, y la posterior eliminación de las palabras funcionales, facilitó encontrar las relaciones con los eventos y establecer una cuantificación para la influencia entre idiomas (llamada uso), pero ¿qué sucedería con esta cantidad si se realizaran otras reglas para eliminar ciertas palabras?

Esta interrogante ha llevado a la construcción de un nuevo algoritmo que limite a las palabras,  reduciendo el conjunto de las migraciones y obteniendo nuevos valores del uso entre idiomas.  Elegidos una pareja de idioma origen \textit{A} e idioma receptor \textit{B}, el proceso es el siguiente. 


\begin{enumerate}
	
	\item Se toma la lista de los préstamos acumulados de \textit{A} en \textit{B},  este conjunto se denotará como \textbf{conjunto original}.
		
	\item Se escogen de forma aleatoria un conjunto de letras (desde una hasta diez), y se descartan de los préstamos acumulados a todas las palabras cuya primer letra sea alguna de las elegidas; siendo este nuevo grupo el \textbf{conjunto reducido}.
	
	\item Se establece un tercer grupo designado como \textbf{conjunto residuo}, conformado por todas las palabras eliminadas del conjunto original.  La unión del reducido y el residuo es el original. 
	
	\item En los tres conjuntos se emplea la ecuación \ref{ec.fuso}, para encontrar el uso de \textit{A} en \textit{B}. 	
	 
\end{enumerate}

La intención de estas alteraciones no es desaparecer el conjunto de las migraciones sino el reducirlas  y comparar el uso entre el conjunto original y el reducido.  Para poder decir que tanto ha cambiado el uso entre idiomas en los dos conjuntos, se utilizará el coeficiente de determinación $R^{2}$. 

El primer criterio importante sera tomar el uso en el conjunto original como verdadero(ya que con el se establecieron los resultados del capítulo anterior) identificando sus valores a lo largo del tiempo $t$ como $O_{t}$, si los valores de uso en el conjunto reducido se denotan como  $v_{t}$ y el promedio de ellos es $\bar{v}$, entonces   el coeficiente de determinación queda definido como:
 
\begin{equation}
 \label{ec.dif_uso}
 R^{2} = 1 - \sum_{t} \frac{ \left( v_{t}- O_{t} \right)^{2}  }{ \left( v_{t} - \bar{v} \right)^{2} }
\end{equation}

Se define el concepto \textbf{conservación del uso} para aquellos pares de conjuntos donde el uso no cambie a pesar de las omisiones; la conservación es favorable si $R^{2}$ es próximo a 1.  Si la conservación no es favorable, es indicio de que para el idioma las palabras que fueron eliminadas son las más relevantes. 

\section{Características de las eliminaciones}

El proceso anterior se realizó cuatrocientas veces por cada pareja de idiomas, obteniendo en cada uno un valor de $R^{2}$. Tras los múltiples descartes, se distinguieron las siguientes características al graficar el uso del conjunto original y del reducido. 


\begin{itemize}
	
	\item Valores iguales. Punto a punto el conjunto reducido empalma al original, siendo las gráficas indistinguibles. La conservación del uso se da en todo el intervalo de tiempo. 
	
	\item Diferencia de alturas. Ambas gráficas muestran el mismo comportamiento, sin embargo existe una diferencia casi constante entre el uso original y el reducido. En este caso se dirá que el uso también se conserva ya que ambas graficas tienen los mismos valores sólo que están desfasados. 
	
	\item Alteraciones por periodos.  Presenta periodos donde el uso de ambos conjuntos son completamente diferentes. La conservación se da por periodos de tiempo e incluso puede ser inexistente.

	
\end{itemize}

Para ilustrar las caracterizaras antes mencionadas, se exponen algunas graficas obtenidas, representado el uso del conjunto original con un trazo continuo, mientras que el uso en el conjunto reducido  es una serie de puntos. En cada grafica se especifica que idiomas se están tratando así como el conjunto de letras con las cuales se hicieron las eliminaciones. 


\begin{figure}[h!]
	\centering
	\includegraphics[scale=.38]{OM1.png}
	\label{fig.OM1}
	\caption{En ambas parejas de idiomas hay conservación del uso durante todo el siglo XX, al presentar valores iguales en el inglés-francés y  diferencia de alturas en el francés-inglés.}
\end{figure}


\begin{figure}[h!]
	\centering
	\includegraphics[scale=.38]{OM2.png}
	\label{fig.OM2}
	\caption{La conservación para el español-alemán no existió durante los años de 1960 a 1990,  correspondiendo a una alteración por periodos. Para el italiano-español el uso se conservo  en la mayor parte del siglo al existir una diferencia de alturas, en los últimos años se caracteriza por una alteración.}
\end{figure}


\begin{figure}[h!]
	\centering
	\includegraphics[scale=.38]{OM3.png}
	\label{fig.OM3}
	\caption{ En el italiano-francés la diferencia de alturas comienza a reducirse tras avanzar en los años, por lo que hay conservación. El alemán-francés  presenta las tres características,  valores iguales  en 1918, alrededor de 1980 y antes del 2000,  alteraciones entre 1920 y 1940, y deferencia de alturas  en los años restantes; en promedio durante todo el siglo XX el uso fue conservado, aunque en este caso la conservación dependerá del periodo tratado.}
\end{figure}


La característica del uso no siempre será la misma tras una nueva elección de letras, sin embargo  es posible decir de manera general si un idioma se conserva.  Para todas las repeticiones, se promedio el valor de $R^{2}$ por cada pareja de idiomas, y a la vez por cada idioma origen.  Los resultados de la tabla \ref{tab.conservacion} muestran de manera general si el idioma origen se conserva, así como con cuales receptores existieron más y menos cambios. 



\begin{table}[h!]
	\centering
	\begin{tabular}{cccc}
		\textbf{}         & \textbf{$R^{2}$} & \textbf{\begin{tabular}[c]{@{}c@{}}más \\ cambios\end{tabular}} & \textbf{\begin{tabular}[c]{@{}c@{}}menos \\ cambios\end{tabular}} \\
		\textbf{inglés}   & 0           & 0                                                               & 0                                                                 \\
		\textbf{francés}  & 0           & 0                                                               & 0                                                                 \\
		\textbf{alemán}   & 0           & 0                                                               & 0                                                                 \\
		\textbf{italiano} & 0           & 0                                                               & 0                                                                 \\
		\textbf{español}  & 0           & 0                                                               & 0                                                                
	\end{tabular}
	\caption{xxxx}
	\label{tab.conservacion}
\end{table}



\section{Comentarios del método}

El realizar diferentes elecciones para restringir a las palabras que conforman los prestamos de un idioma en otro, mostró desde el punto de vista estadístico que no importan cuales elementos conforman el corpus, la propiedad del uso es la misma, ya que individualmente los valores de uso de una única palabra pueden variar en los años del análisis y ser distintos a los de otra palabra, sin embargo al tratar a todo el conjunto, el uso se comporta de la misma  manera, sin importar los valores individuales de los elementos que lo conforman. 







