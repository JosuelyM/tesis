
\begin{abstracts}        

En los últimos años, el desarrollo científico y tecnológico,  y el crecimiento económico de países como los Estados Unidos,  han hecho del idioma inglés el lenguaje común para la comunicación y para la difusión de información. Esto ha provocado que algunos vocablos del inglés comiencen a surgir en los demás idiomas,  mezclándose con las palabras típicas del idioma, y en ocasiones desplazándolas.  

Está tendencia no ha sido una característica única del inglés, a lo largo del tiempo, diferentes idiomas han aportado y modificado el vocabulario de otras lenguas. En este trabajo se estudia la forma en que los idiomas inglés, francés, alemán, italiano y español, se han relacionado durante el siglo XX a través de las palabras que son comunes entre ellos, llamadas  \textit{palabras migrantes};  y a partir de ellas, se han propuesto dos formas para cuantificar la influencia que un idioma ha tenido en otro. 

También se presenta un estudio estadístico llamado \textit{diversidad de rango}, con el que se cuantifican las distintas palabras migrantes que pueden ocupar lugar en un listado, si estas se ordenan  a partir de su frecuencia.  

Finalmente se justifica la falta de un rigor lingüístico, al eliminar cierta cantidad de palabras migrantes, para volver a obtener la influencia entre idiomas, y medir su similitud con los resultados previos. 



 
\end{abstracts}
%\end{abstractlongs}


% ----------------------------------------------------------------------