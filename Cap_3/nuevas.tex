% !TeX spellcheck = es_ES
\chapter{Palabras nuevas}
% Intro {{{
En el capítulo anterior se mencionó que el algoritmo toma como  idioma
influyente a aquel que transmite palabras hacia los demás y que estas perduren
en los receptores al menos dos años.

El objetivo ahora es identificar qué idioma es más influyente sobre los demás. Sin embargo establecer un método que proporcione este resultado no es sencillo, puede ser ambiguo decir que algo es más o menos influyente, no existe una serie de pasos a seguir cuyo resultado final sea una cantidad medible para cuantificar la influencia.  Para poder llegar a esta conclusión habrá que
especificar las condiciones y características que deben de tener los elementos en los cuales se quieren conocer estas cantidades.
 
\fxnote{Por favor separa  en frases. Esta muy largo y dificil de leer.}
\fxwarning{Ya la corte, quedo mejor}

La base de datos proporciona información de los orígenes, los receptores y las palabras que migran de un lado a otro; comenzando con estos elementos una primer manera de establecer la influencia es  identificar los tiempos donde ocurren las migraciones.  Para ello conviene hacer una clasificación dentro de los propios préstamos y a partir de ella  continuar con una cuantificación de la influencia.  Si se tiene una pareja de idioma origen \textit{A} e idioma receptor \textit{B} entonces se clasifican a los préstamos de \textit{A} en \textit{B} como:

\textbf{Préstamos Nuevos}: Son palabras que aparecen por primera vez en las más
usadas del idioma \textit{B}.  Únicamente serán nuevos en el primer año de
aparición.  Se recuerda que todos los préstamos nuevos, volvieron a aparecer en
alguna lista del idioma B posterior al primer año de la migración, omitiendo a
aquellos que solo aparecieron una vez. 

Dentro de los años comprendidos en el conjunto de búsqueda
(1901-2009), existen múltiples posibilidades con las cuáles trabajar a los
cinco idiomas. La manera en que se tratará de llegar a un resultado que muestre la influencia puede ser de tres formas:
\begin{enumerate}
	
\item Contando por cada año los préstamos de origen \textit{A} que están presentes en los diferentes receptores. 

\item Tomar ahora a \textit{A}  como el idioma receptor, y contando cuántos préstamos nuevos de diferentes orígenes están presentes en cada año.  

\fxnote{El caso 1 y 2 no son equivalentes Mas bien el primero mide la influencia, y el punto dos la suceptibilidad.}
\fxwarning{Hablé de las diferencias entre los puntos al final de los item}

\item Tomando fijos un idioma \textit{A} y un idioma \textit{B}, contabilizar cuántos préstamos nuevos por año se encuentran si se toma a \textit{A} como origen y a \textit{B} como receptor,  una vez hecho esto  repetir el conteo intercambiando a \textit{B} como el origen y a \textit{A} como el receptor. 

\end{enumerate}


En los tres casos la influencia estará ligada a la cantidad de préstamos
nuevos.  Las tres formas dan resultados diferentes,  el primer punto reflejara en que idiomas \textit{A} ha sido influyente, mientras que el segundo punto mostrara la susceptibilidad  de \textit{A} ante los demás idiomas,  finalmente el tercero combina la influencia con la susceptibilidad ya que al tratar únicamente a dos idiomas se verá en que épocas alguno de ellos ha tenido más del punto uno o del dos. 

\fxnote{Siento que las tres maneras miden cosas diferentes. Seria bueno poner una frase interpretando las medidas en cada explicacion. Esta padre! Sin embargo, luego en los gráficos que veo, solo usas una medida, no? Como esta la cosa?}

\fxwarning{Ya lo comente en donde introduzco las graficas}
% }}}
\section{Eventos que ayudan a las migraciones} % {{{

Una parte complementaria de interpretar la influencia es responder qué es lo que propicia el haber migraciones de un lado hacia otro.  El tener los préstamos nuevos clasificados por el año de la primera migración ayuda a identificar posibles causas de carácter histórico que propiciaron las migraciones en ese año o alrededor de él.  Por ejemplo, la globalización y el acceso de más población a la tecnología en los años recientes ha propiciado que una ola de palabras de su campo semántico hayan migrado entre orígenes y receptores,  \textit{internet}, \textit{computadora}, \textit{web}, \textit{email},  \textit{software}, son ejemplos de palabras que surgieron tras el fenómeno de globalización y desarrollo tecnológico. 

\fxnote{Cuando ocurrio la globalizacion y como se define? Es curioso, porque todos los ejemplos son cosas tecnollógicas asociadas con internet\ldots}

\fxwarning{Todas las palabras que mencione surgieron a partir de 1990, y puse estas porque para una mayoria de personas estas palabras si las considera como nuevas (provenientes de otro origen)y si aceptan que se han mezclado con el español}

Los eventos pueden ser de diferentes tipos y al tratar con idiomas, se espera que en tales eventos participen países, personajes o  comunidades que hablan las lenguas involucradas.  El asociar sucesos con palabras y a la vez con determinados grupos ayudará a mostrar el alcance que tuvo un lenguaje y la forma en la que surgió la influencia y esta se propagó.

Al contar con datos que abarcan mas de cien años, se resalta que la propagación no es inmediata; si un evento propicia una migración de palabras, el tiempo que le lleva a un termino migrar a otro conjunto dependerá de la época en la que se origino el evento y de la propia relevancia del mismo, existiendo palabras que brotan en los diferentes receptores hasta diez o veinte años después de surgir en los orígenes.
\fxnote{Frase hiperlarga. Corrige.}
\fxwarning{Listo}

Para cada caso estudiado, se hizo un análisis histórico de las palabras que
conforman los préstamos nuevos entre idiomas.  En la mayor parte de los casos
se ubicaron sucesos para explicar el por qué ciertas palabras migraron y la
importancia del año o años en que lo hicieron. 


% }}}
\section{Palabras nuevas de un idioma en los demás}

 % {{{
 

Ya establecida la forma en que se buscarán  palabras nuevas en los diferentes idiomas,  los resultados encontrados se presentarán de la siguiente manera:

\begin{itemize}
	
\item Por cada idioma se determinó la influencia que esté ejerció en los demás, mencionada en el punto 1 al comienzo del capítulo. Se denotará con la letra $N_{p}$  a la cantidad de palabras nuevas.

\fxnote{Creo que hay forma de citar la pagina en cuestión. Debe ser un comando como ref o algo asi}. 
\fxwarning{No encontre la forma, sino al final de editarlo agrego " a mano" la pagina donde quedo el punto 1}

\fxnote{Si realmente las estas usando quiza valga la pena ponerle nombres a las cantidades, por ejemplo $F^{1954}_{\text{en} \to \text{fr}}$ ojala usando letras que tengan un significado, por ejemplo $S$ para susceptibilidad o como quieras llamarlo. } 
\fxwarning{listo}

\item El conteo de palabras nuevas se realizó en un periodo de 109 años (1900 - 2009) y por cada década comprendida en este tiempo, es decir  se mostrarán la cantidad de palabras que aparecieron en los 109 años y a la vez como se distribuye esta cantidad en las diferentes décadas.  La intención de este punto es observar en qué periodos un idioma aportó o recibió más palabras. 

\item Se proporcionará en \cite{prestamos_nuevos} las listas  de préstamos nuevos entre cada idioma, agrupados por el año de aparición en el receptor.  Se especificará en el Anexo 1  la forma de leer e interpretar estas listas.

\item En cada resultado y por cada par de idiomas se agregará un contexto enfatizando los sucesos históricos de las palabras involucradas, así como los posibles detonantes que propiciaron los movimientos.  Esto permite omitir los resultados del punto 2 (ver  página [XXX] ), ya que al especificar los eventos y las palabras involucradas en el movimiento de  \textit{A} a \textit{B}, se esta cubriendo de manera implícita los términos  a los que \textit{B} fue susceptible.

\item Los coemntarios realizados estarán sustentados por la influencia entre \textit{A} y \textit{B}, (punto 3), las graficás correspondientes se anexaran en el Apéndice 1.  

\end{itemize}


\fxnote{Respecto al numero en la bibliografia despues de la referencia, para que lo quieres? Creo que no es estandard y a mi me confunde. Te propongo quitarlo.}

\fxwarning{ok, veo como quitarlo por el momento no lo se}

% \subsection*{Lectura de las gráficas} 
\fxnote{No creo que amerite una subsección. La quite. Si no estas de acuerdo,  respondeme y la vuleves a poner.}

\fxwarning{de acuerdo, es más una adicion a la misma seccion}

Para las gráficas del trabajo se utilizaron diferentes colores para
caracterizar las influencias de un idioma, además en cada gráfica se especifica los idiomas que intervienen y para su mejor lectura se utilizaron abreviaciones. Los colores y abreviaciones de cada idioma se presentan  en la tabla \ref{tab.idcolor}.
%\fxnote{Nota como hace uno referencia a una tabla. Se deja flotante. Cuando  entiendas esta nota, borrala.}

\begin{table} % {{{
	\centering
	\begin{tabular}{ccc}
		\textbf{Idioma} & \textbf{Abreviación} & \textbf{Color} \\
		Inglés          & EN                   & Azul           \\
		Francés         & FR                   & Amarillo       \\
		Alemán          & GE                   & Violeta        \\
		Italiano        & IT                   & Verde          \\
		Español         & SP                   & Guinda        
	\end{tabular}
	\caption{Nomenclatura de los idiomas}
	\label{tab.idcolor}
\end{table} % }}}

En cada gráfica se especificará por medio de  abreviaciones y  colores a los
idiomas tratados, todo ello en una leyenda; la primera abreviación que será el
idioma origen de los préstamos, y la segunda corresponderá al idioma receptor,
por ejemplo  la leyenda EN-FR serán los préstamos que van del inglés al
francés, en cambio FR-EN son los préstamos que van en sentido contrario.
En todas las gráficas, el eje horizontal está representado por los años del
conjunto de búsqueda 1900-2009,  mientras que en el eje vertical se presenta la
cantidad de prestamos nuevos. 
% }}}


%\section{Palabras nuevas de un idioma en los demás} % {{{
\fxnote{Me parece que la anterior seccion y esta son una sola. Si quieres lo  discutimos}
	
\fxwarning{Cambie el nombre de la seccion por Palabras nuevas,  dentro de esta lo primero será especificar  como se presentan los resultados y las siguientes subsecciones los resultados mismos}

\fxnote{Antes de avanzar, tenemos que discutir las graficas. Creo que admiten una mejoría. Ver notas en un caption mas adelante y poner en espacio sencillo}

\fxwarning{listo}

\clearpage

\subsection{Inglés}

\begin{figure}[h!]
	\centering
	\includegraphics[scale=.38]{Cap_3/NC_EN.png}
	\label{fig.NC_EN}
	\caption{Palabras nuevas del inglés hacia los demás. El alemán se ha beneficado más del inglés, teniendo los mayores aportes en los primeros años del siglo XXI, caracterizado por palabras de tipo industrial y del desarrollo tecnológico.}
\end{figure} % 



De manera general, el idioma que más se ha beneficiado del inglés ha sido el
alemán, con 300 préstamos en 100 años.  Inglés y alemán forman parte de la
misma familia lingüística de las lenguas germánicas,  posible razón de los
mayores intercambios. Entre las lenguas romances, el francés fue el más
favorecido, pero también es la más similar por las relaciones normandas entre
ambas.
\subsubsection*{Inglés-Francés} % {{{

Los mayores aportes se dieron de 1930 a 1970, periodo que engloba comienzos de
la gran depresión, la segunda guerra mundial y la guerra fria, sucesos donde
participaron países de ambas lenguas. Las palabras involucradas en este periodo
hacen referencia a estos eventos, entre 1944 y 1945 surgieron en el francés los
términos \textit{Churchill}, \textit{territories}, \textit{nazis} y
\textit{catastrophe},  mientras que en las décadas entre 1950 y 1970
aparecieron \textit{Nixon}, \textit{dollar} y \textit{Johnson}; dos de estas
palabras aluden a apellidos de presidentes de los Estados Unidos,  Lyndon B.
Johnson y Richard Nixon, cuyos periodos de gobierno fueron  entre 1963-1969 y
1969-1974 respectivamente.
% }}}
\subsubsection*{Inglés-Alemán} % {{{
Entre todas las décadas, solo en dos de ellas (1900  y 1980) el alemán no fue
el idioma que mas prestamos recibió  del ingles. Entre las palabras encontradas
en las demás épocas están \textit{economic} (1929), \textit{depression} (1931),
\textit{investment} (1933), \textit{Roosevelt} (1935), que pertenecen al campo
semántico de la gran depresión mientras que el presidente Franklin D. Roosevelt
gobernó posterior a la crisis económica y durante la segunda guerra mundial. La
gran depresión se origino en los Estados Unidos con consecuencias en la
economía de diferentes países, entre ellos  Alemania, y fue uno de los motivos
que propiciaron la segunda guerra mundial.

En las ultimas dos décadas, la globalización y  el desarrollo tecnológico son
responsables del mayor crecimiento, palabras como \textit{standards} (1983),
\textit{market} (1994), \textit{internet} (1996), \textit{economy} (1996),
\textit{online} (1998), \textit{value} (2001), \textit{financial} (2003) y
\textit{customer} (2007). 
% }}}
\subsubsection*{Inglés-Italiano} % {{{
La principal característica de las palabras hacia el italiano son apellidos de
personajes involucrados en la segunda guerra mundial, \textit{Roosevelt} (1941)
o \textit{Stalin} (1949), el el caso de Joseph Stalin a pesar de que su
nacionalidad no es la de algún pais de habla inglesa, en el ingles su apellido
tomo notoriedad para exportarse a los otros idiomas; este es un ejemplo de que
no todas las palabras encontradas nacieron en el idioma origen,  solamente en
el se hicieron populares.  En los últimos años nuevamente la globalización y el
poder economico de los Estados Unidos han hecho favorable el crecimiento del
ingles, \textit{internet} (1996), \textit{bussines} (2000), \textit{marketing}
(2001) y \textit{bush} (2002), son terminos que abarcan estos acontecimientos. 
% }}}
\subsubsection*{Inglés-Español} % {{{
El español ha sido en la mayor parte de los periodos de diez años,  el idioma
que menos prestamos ha adoptado del ingles, sin embargo los hechos a los que se
han ligado han sido en más areas que en las otras combinaciones.  Nombres de
organizaciones y empresas,  \textit{standard} (1933) (y \textit{oil} (1931)
aludiendo a la extinta Standard Oil) y \textit{unesco} (1955);  personajes de
la historia de los Estados Unidos,  \textit{Roosevelt} (1941), \textit{Kennedy}
(1961), \textit{Johnson} (1966),  \textit{Nixon} (1972) y \textit{Bush} (1990);
y a la globalizacion en los últimos 20 años, \textit{internet} (1996),
\textit{mail} (1999), \textit{marketing} (2001) y \textit{software} (2004).   

A pesar de no ser el idioma más favorecido es al que en más areas ha impregnado
el ingles, siendo este un factor que también puede indicar una mayor
influencia,  en cuantas áreas esta presente un idioma y que tanto se utiliza. 

% }}}
% }}}

\clearpage

\subsection{Francés} % {{{

\begin{figure}[h!]
	\centering
	\includegraphics[scale=.38]{Cap_3/NC_FR.png}
	\label{fig.NC_FR}
	\caption{Palabras nuevas del francés hacia los demás. Entre las demás lenguas romances, el aporte del francés ha sido minoritario, siendo las lenguas germánicas donde el francés asentó más elementos. Se destaca el mayor beneficio del alemán en las décadas durante la primera (1920) y segunda guerra mundial (1940).}
	
	
\end{figure}

\subsubsection*{Francés-Inglés}% {{{

La característica de los prestamos, es que son principalmente palabras que son comunes en el ingles, y que a simple vista serian catalogados como errores de clasificación, por ejemplo  \textit{diagnostic,} \textit{clients,} \textit{placement,} \textit{adaptation,} \textit{diffusion,} \textit{amplitude,} no pareciese lógico ver que migraron del francés hacia el inglés, sin embargo el algoritmo designo este origen por ser al principio de la base de datos (1740)  donde las palabras eran más utilizadas.  Antes de considerarse un error, se puede inferir que antes las obras en francés eran bastas de vocablos de otros idiomas, destacando el papel que tuvo el francés como idioma "común" para transmitir información. 


% }}}
\subsubsection*{Francés-Alemán}% {{{

El alemán tuvo dos décadas donde la diferencia de palabras nuevas que llegaron a él es mas significativa que en los otros idiomas, en los periodos de 1920 y 1940  (posteriores a las dos guerras mundiales), entre el contenido se ubicaron a  textit{diplomatie} (1917), \textit{bourgeoisie} (1919),  \textit{guerre} (1925), \textit{Allemagne} (1925), \textit{Russie} (1925) y \textit{empire} (1937); siendo un conjunto de palabras utilizables al referise a temas políticos  o diplomáticos, y por donde participaron países hablantes de las dos lenguas. 


% }}}
\subsubsection*{Francés-Italiano}% {{{

El italiano se vio mas susceptible al francés en los primeros años del siglo, aunque no fue posible ligar a las palabras a un hecho relevante en estos años. Entre las pocas clasificaciones se encuentra la terna cientica con \textit{Poincaré} (1924), apellido del matemático francés Henri Poincaré, y la bélica,  entre los terminos encontrados estan \textit{Versailles} (1924), textit{Vietnam} (1966)  y \textit{URSS} (1975), mostrando que la  La primera y segunda guerra mundial fueron detonantes para el flujo de palabras.


% }}}
\subsubsection*{Francés-Español}% {{{

Al igual que la tendencia en el italiano, en el español no llegaron gran cantidad de palabras cuyo contenido sea ligado a un evento, a pesar de que en cien años migraron alrededor de veinte. La mas destacada fue \textit{euros} (2002) por ser el año de circulación de la moneda de la unión europea, organización donde son miembros países importantes de las dos lenguas. 

El hecho de no poder enlazar palabras a eventos, no significa que el francés no es importante para el español (o el italiano), sino que el periodo donde los hechos tuvieron mayor impacto no esta dentro del periodo de búsqueda,  por ejemplo hechos como la revolución francesa, o la invasión napoleónica a España, propiciaron a un mayor intercambio en este sentido, pero al ocurrir antes de 1900 no permite tener una conclusión de ello. 


% }}}
% }}}

\clearpage

\subsection{Alemán}% {{{

\begin{figure}[h!]
	\centering
	\includegraphics[scale=.38]{Cap_3/NC_GE.png}
	\label{fig.NC_GE}
	\caption{Palabras nuevas del alemán hacia los demás.El francés destaca durante la primera mitad del siglo como el idioma donde tiene más presencia el alemán, logrando las mayores migraciones en los años entre las guerras mundiales (1930-1940). Se distingue al español como el idioma que menos préstamos obtuvo  existiendo décadas donde el aporte fue nulo.  Campos como el desarrollo científico, la filosofía  y la política donde los germanoparlantes tuvieron papeles destacados en el siglo pasado ha hecho posible que las demás lenguas se impregnan del alemán.}
\end{figure}




\subsubsection*{Alemán-Inglés}% {{{

Como en la relación en sentido inverso, en los años posteriores a la segunda guerra mundial,  fueron donde mayores relaciones de palabras se encontraron  \textit{Lenin} (1931), \textit{Hitler} (1934) y \textit{reich} (1939) forman parte  de este contexto histórico.  Otras palabras relevantes son \textit{Marx} (1934) y \textit{Freud}, apellidos de dos personaje sy autores destacados en la filosofía y psicología. 


% }}}
\subsubsection*{Alemán-Francés}% {{{

Como se comento de la grafica por siglo, el francés ha recibido más palabras del alemán que cualquier otro. A pesar de que la mayor cantidad de aportes se dio en la primera mitad de siglo, las relaciones que se encontraron han sido a lo largo de todo el periodo y en diferentes áreas. 

Destaca la década de 1940, con palabras como \textit{regierung},  \textit{deutschen},\textit{minister} y  \textit{bestimmungen}(traducciones de gobierno, alemán, ministro y reglamentos) todas apareciendo en 1944;  si se añaden palabras en años previos como \textit{Hitler} (1933), \textit{kaiser} (1915) y \textit{reich} (1921), son conceptos que muestran parte de la historia del alemán en las guerras. 

El objetivo no es solo identificar sucesos de carácter militar en esta dirección de los préstamos, el identificar un nombre o apellido facilita encontrar las relaciones con un ámbito,  además de Hitler se encontraron los siguientes apellidos:  \textit{Nietzsche} (1905),  \textit{Marx} (1923), \textit{Heidegger} (1987),  \textit{Mozart} (1956), \textit{Freud} (1965) y \textit{Engels} (1970), enlazados a la filosofía, la música y la medicina,  además todos ellos de personajes nacidos en países germanohablantes.


% }}}
\subsubsection*{Alemán-Italiano}% {{{

La proximidad geográfica de Italia con países con lengua oficial en el alemán, hace posible la entrada de palabras, algunas no tan comunes.  Los apellidos de los personajes mencionados anteriormente, también se encuentran en el lenguaje italiano, apareciendo en años próximos a los ya citados.  

El único apellido que se asentó exclusivamente en el  italiano fue \textit{Berchtold} en 1943, aludiendo a Leopold Berchtold ministro de exteriores del Imperio Austro-Húngaro de 1912 a 1915, cuyo fallecimiento ocurrió en 1942.  Se hablo de la proximidad geográfica, ya que este puede ser un factor (ademas del contexto histórico) que detone las migraciones de palabras, la proximidad de Italia con el imperio austro-húngaro puede inferir en la existencia de  palabras que migren entre  dos idiomas y sólo entre ellos. 

  
% }}}
\subsubsection*{Alemán-Español}% {{{

Las palabras que van en este sentido,  presentaron años (o una décadas)  con pocas migraciones o sin alguna, el incremento de palabras se dio posterior a 1950 donde los años sin intercambio disminuyeron.  \textit{Marx} (1932), \textit{kaiser} (1938), \textit{Hitler} (1940), \textit{Lenin} (1970), \textit{Hegel} (1971),  \textit{Nietzsche} (2000) y \textit{Freud} (2002) son parte de los términos que llegaron al español,  sin embargo, es peculiar que las dos últimas hayan aparecido en el español muchos años después que en los demás idiomas, por ejemplo en el francés,  Nietzsche apareció 1905 y Freud en 1965. El letargo de años puede ser una característica del tiempo que les lleva  a las  palabras del alemán pasar hacia el español, al adaptarse a una lengua de una familia distinta y donde históricamente no ha existido un evento conjunto entre países. 



% }}}
% }}}

\clearpage
\subsection{Italiano}% {{{

\begin{figure}[h!]
	\centering
	\includegraphics[scale=.38]{Cap_3/NC_IT.png}
	\label{fig.NC_IT}
	\caption{Palabras nuevas del inglés hacia los demás.Al provenir de la misma familia grecolatina y ser fonética y etimológicamente similares, el español ha adoptado la mayor cantidad de palabras provenientes del italiano,  seguido del francés otra lengua romance.} 
\end{figure}




\subsubsection*{Italiano-Inglés}% {{{

A pesar de que en cada década existen términos nuevos en el inglés, solo ha sido posible asociar \textit{mussolini} (1935) al político y militar Benito Mussolini, tal vez el personaje italiano más relevante para la historia en el siglo XX 

% }}}
\subsubsection*{Italiano-Francés}% {{{



En las migraciones sólo se asoció \textit{Mussolini} (1935), la cual ya se había mencionado en las migraciones del italiano al inglés, tras revisar las listas de migraciones con origen italiano  a los demás idiomas, Mussolini siempre se encuentra en todas las migraciones y en el mismo año ratificando la importancia de este término en la historia. 

Aunque en 1940 migraron la mayor cantidad de préstamos, ninguno de ellos ha tenido contexto con los sucesos de esa época. 

% }}}
\subsubsection*{Italiano-Alemán}% {{{

En esta dirección de las palabras del italiano, si fue posible relacionarlas con el contexto bélico,  \textit{regime} (1938), \textit{panzer} (1941), \textit{duce} (1942),  traducciones de régimen, blindado y líder, además de \textit{Mussolini} (1935). 



% }}}
\subsubsection*{Italiano-Español}% {{{

En el español, las tendencias no fueron sólo hacia la guerra, también a idelogías políticas como \textit{socialista} (1914), \textit{comunista} (1932), \textit{capitalismo} (1935), \textit{fascismo} (1937),  \textit{marxismo} (1963) y \textit{terrorismo} (1986). 



% }}}
% }}}

\clearpage
\subsection{Español}% {{{

\begin{figure}[h!]
	\centering
	\includegraphics[scale=.38]{Cap_3/NC_SP.png}
	\label{fig.NC_SP}
	\caption{Palabras nuevas del español hacia los demás. El español y el italiano, es la única reciproca, donde  el italiano es ahora el que más palabras recibió del español, descantando en la década de 1930; el francés la otra lengua romance es el segundo idioma con mayor presencia del español, seguido de las lenguas germánicas. El provenir de una misma familia es un factor para que existan migraciones.}
\end{figure}

\subsubsection*{Español-Inglés}% {{{

Resalta que hay décadas donde el aporte al inglés es mínimo (o no existe). Contrario a la tendencia en las anteriores migraciones donde la guerra era una constante para que se diese el flujo de palabras, en este sentido se encontraron términos médicos en el año de 1943 (la decadas mas prolifera del español en el inglés) aparecieron las palabras \textit{virus} y \textit{anemia}, años antes en 1934 George Richards Minot, Parry Murphy y George Hoiyt Whipple, habían recibido el premio nobel de medicina por su descubrimiento de la terapia de hígado para el tratamiento de anemias.   

Probablemente las palabras virus y anemia ya existían en el inglés años antes de 1943,  pero sólo hasta este año tuvieron la importancia para estar dentro de las cinco mil mas utilizadas. Con este ejemplo, se trata se ver que hay eventos (como un premio internacional) que retoman palabras cuo periodo de uso  ha disminuido y las vuelve a impulsar para distribuirse en los demás idiomas.   No siempre serán las palabras mas recientes en el origen las que realicen las migraciones. 

% }}}
\subsubsection*{Español-Francés}% {{{

El primer préstamo que resultó importante del español hacia el francés es \textit{Panamá} (1913), su trascendencia se liga al año de inauguración del canal de Panamá en 1914, siendo una obra importante para el comercio de la época al conectar los océanos pacífico y atlántico, además el primer gobierno que impulsó económicamente la construcción del canal fue el francés,  aunque su conclusión y administración pasó a los Estados Unidos.  




% }}}
\subsubsection*{Español-Alemán}% {{{


La investigación hecha para ligar muestran nuevamente términos médicos, la palabra \textit{lepra} (1901) fue globalmente importante a partir de 1874,  ya que en ese año el científico noruego Gerhard Armauer Hansen descubrió el bacilo de Hansen Mycobacterium Leprae \cite{lepra} que origina la enfermedad. Por el carácter médico de la palabra es probable que se hiciera más investigación sobre la enfermedad en diferentes idiomas, en este caso el alemán.

Como en las migraciones hacia el ingles, son términos que vuelven a ser importantes,  y esto les permite migrar a otros idiomas. 


% }}}
\subsubsection*{Español-Italiano}% {{{

Los términos médicos han sido una constante en las migraciones del español, en el italiano se encontraron \textit{virus} (1922), \textit{colesterina} (1928),  \textit{sintomatología} (1931), \textit{anestesia} (1932), \textit{vitamina} (1935), \textit{anemia} (1936), \textit{metabolismo} (1936),  \textit{gástrica} (1936)  y \textit{endovenosa} (1937).  

El aparecer estas palabras en el español (dentro de las cinco mil más usadas)antes que en los demás propone que la medicina era un campo importante para los países de habla española, donde posiblemente en los años del conjunto base (1740-1900) se publicaron más libros de medicina en lenguaje español. 






% }}}
% }}}
% }}}
\section{Comentarios del método}% {{{


Tras las múltiples combinaciones entre idiomas, se  encontró que el principal
evento que origino las migraciones ha sido la segunda guerra mundial donde
todos los lenguajes recibieron un termino asociado a ella durante el siglo XX.

Se destaca el papel del inglés en las ultimas dos décadas como el idioma común
para transmitir información; además de ser el exportador de palabras en campos
como la tecnología, disciplina que ha hecho posible la fluidez con la que se
mueve a información. El poderío económico ha hecho de países como los Estados
Unidos una fuente de nueva información.

El alemán también ha acaparado las migraciones, se han distinguido dos causas
de la importancia de este idioma tanto como origen como receptor, la primera
por el papel representado en la historia del siglo XX principalmente por
Alemania donde la guerra permitió la comunicación de varias lenguas con el
alemán, incluso después de que estas terminara, así mismo el protagonismo de
personajes germanoparlantes en diferentes áreas ha hecho que el alemán se
expanda incluso en lenguajes que no son de su propia familia lingüística. 

Finalmente entre las tres lenguas romances, aunque no hayan sido tan
protagonistas (no aporten palabras con un contenido afín),  su periodo de
apogeo e influencia hacia los demás no se tiene registrado en este trabajo.
Destacan los conceptos médicos en la lengua española, mostrando la importancia
que tenia la medicina y la cantidad de libros que se publicaban en esta área en
este idioma. 

Entre las principales deficiencias del método, es que aun es ambiguo decir
quien ha influido mas en los otros; por el momento sólo se puede decir la
manera en que se ha reflejado la influencia, cada conjunto de palabras que
migraron de un idioma a otro, tienen relación a algún ámbito, la guerra, la
economía, la tecnología, las ciencias, las artes y la medicina; cada
combinación tendrá mas elementos de alguna de estas áreas.  Se realizara otro
método para cuantificar la influencia de unos sobre otros. 

% }}}

